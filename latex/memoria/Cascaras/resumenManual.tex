%---------------------------------------------------------------------
%
%                      resumenManual.tex
%
%---------------------------------------------------------------------
%
% resumenManual.tex
% Copyright 2009 Marco Antonio Gomez-Martin, Pedro Pablo Gomez-Martin
%
% This file belongs to the TeXiS manual, a LaTeX template for writting
% Thesis and other documents. The complete last TeXiS package can
% be obtained from http://gaia.fdi.ucm.es/projects/texis/
%
% Although the TeXiS template itself is distributed under the 
% conditions of the LaTeX Project Public License
% (http://www.latex-project.org/lppl.txt), the manual content
% uses the CC-BY-SA license that stays that you are free:
%
%    - to share & to copy, distribute and transmit the work
%    - to remix and to adapt the work
%
% under the following conditions:
%
%    - Attribution: you must attribute the work in the manner
%      specified by the author or licensor (but not in any way that
%      suggests that they endorse you or your use of the work).
%    - Share Alike: if you alter, transform, or build upon this
%      work, you may distribute the resulting work only under the
%      same, similar or a compatible license.
%
% The complete license is available in
% http://creativecommons.org/licenses/by-sa/3.0/legalcode
%
%---------------------------------------------------------------------
%
% Contiene el cap�tulo del resumen.
%
% Se crea como un cap�tulo sin numeraci�n.
%
%---------------------------------------------------------------------

\chapter{Resumen}
\cabeceraEspecial{Resumen}

\begin{FraseCelebre}
\begin{Frase}
  Desocupado lector,  sin juramento me  podr�s creer que  quisiera que
  este  libro  [...] fuera  el  m�s hermoso,  el  m�s  gallardo y  m�s
  discreto que pudiera imaginarse.
\end{Frase}
\begin{Fuente}
  Miguel de Cervantes, Don Quijote de la Mancha
\end{Fuente}
\end{FraseCelebre}

\texis\ es un conjunto de ficheros \LaTeX\ que pueden servir para
escribir tesis doctorales, trabajos de fin de master, de fin de
carrera y otros documentos del mismo estilo. El documento que tienes
en tus manos es un manual que explica las distintas caracter�sticas de
la plantilla. En los distintos cap�tulos iremos explicando los
ficheros existentes en \texis\ as� como su funci�n. Tambi�n se
explican algunas de las caracter�sticas, como por ejemplo ciertos
comandos que facilitan la escritura de los documentos.

Aunque el c�digo \LaTeX\ utilizado en \texis\ est� muy comentado
para su uso f�cil, creemos que las explicaciones que aqu� se
proporcionan pueden ser �tiles.

Hay dos distribuciones distintas de \texis: el c�digo fuente completo
de este manual (de forma que \texis\ es ``\emph{su propio
  manual}''\footnote{Los expertos en l�gica seguro que tendr�an algo
  que decir al respecto...}), o una distribuci�n casi ``vac�a de
contenido'', que tiene un �nico cap�tulo y ap�ndice vac�o, pero
mantiene la portada, dedicatoria, agradecimientos y bibliograf�a del
manual.

Dependiendo, pues, de qu� distribuci�n escojas, partir�s directamente
de los ficheros \texttt{.tex} de este manual y eliminar�s su texto
para a�dir el tuyo, o de un conjunto de ficheros sin apenas contenido
que rellenar�s. Aconsejamos esta �ltima aproximaci�n por ser m�s
c�moda. Sin embargo, hacemos disponible los ficheros \texttt{.tex} del
manual como referencia.

Para facilitar las cosas, hemos intentado que su estructura sea
parecida a la de una posible tesis. De esta forma el c�digo fuente del
propio manual puede servir como punto de partida para la escritura de
este tipo de documentos. Como podr�s comprobar, en alg�n momento nos
ha sido dif�cil justificar la existencia de ciertos elementos pues no
eran realmente relevantes para el manual. En esos casos, piensa que
est�n ah� no porque sean importantes desde el punto de vista de
\emph{este} documento, sino porque muy posiblemente estar�an en el
tipo de textos para los que \texis\ es �til.

\medskip

Al estar compuesto por varios tipos de ficheros, \texis\ se rige por varias licencias:

\medskip

\begin{center}
\begin{tabular}[h]{m{3.5cm}m{8cm}}
& La plantilla (ficheros en el directorio \texttt{TeXiS}) se distribuye bajo la \emph{\LaTeX\ Project Public License} (Licencia P�blica del Proyecto \LaTeX). \\
& \\
& \\
\includegraphics[width=3cm]%
                     {Imagenes/Bitmap/GPLv3-logo-red} & Los ficheros \texttt{Makefile} y scripts de apoyo a la generaci�n del documento, se distribuyen bajo licencia GPLv3. \\
& \\
& \\
\includegraphics[width=3cm]%
                     {Imagenes/Vectorial/by-sa} & El \emph{manual} de \texis\ se distribuye con una licencia Creative Commons (CC-BY-SA). \\
\end{tabular}
\end{center}


\endinput
% Variable local para emacs, para  que encuentre el fichero maestro de
% compilaci�n y funcionen mejor algunas teclas r�pidas de AucTeX
%%%
%%% Local Variables:
%%% mode: latex
%%% TeX-master: "../ManualTeXiS.tex"
%%% End:
