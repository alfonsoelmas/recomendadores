%----------------------------------------------------------------------
%
%                          TFG.tex
%
%----------------------------------------------------------------------
%
% TFG.tex
% Copyright 2009 Marco Antonio Gomez-Martin, Pedro Pablo Gomez-Martin
%
% This file belongs to the TeXiS manual, a LaTeX template for writting
% Thesis and other documents. The complete last TeXiS package can
% be obtained from http://gaia.fdi.ucm.es/projects/texis/
%
% Although the TeXiS template itself is distributed under the 
% conditions of the LaTeX Project Public License
% (http://www.latex-project.org/lppl.txt), the manual content
% uses the CC-BY-SA license that stays that you are free:
%
%    - to share & to copy, distribute and transmit the work
%    - to remix and to adapt the work
%
% under the following conditions:
%
%    - Attribution: you must attribute the work in the manner
%      specified by the author or licensor (but not in any way that
%      suggests that they endorse you or your use of the work).
%    - Share Alike: if you alter, transform, or build upon this
%      work, you may distribute the resulting work only under the
%      same, similar or a compatible license.
%
% The complete license is available in
% http://creativecommons.org/licenses/by-sa/3.0/legalcode
%
%---------------------------------------------------------------------
%
% Este fichero contiene el "documento maestro" del manual. Lo ?nico
% que hace es configurar el entorno LaTeX e incluir los ficheros .tex
% que contienen cada secci?n.
%
%----------------------------------------------------------------------
%
% Los ficheros necesarios para este documento son:
%
%       TeXiS/* : ficheros de la plantilla TeXiS.
%       Cascaras/* : ficheros con las partes del documento que no
%          son cap?tulos ni ap?ndices (portada, agradecimientos, etc.)
%       Capitulos/*.tex : cap?tulos de la tesis
%       Apendices/*.tex: ap?ndices de la tesis
%       constantes.tex: constantes LaTeX
%       config.tex : configuraci?n de la "compilaci?n" del documento
%       guionado.tex : palabras con guiones
%
% Para la bibliograf?a, adem?s, se necesitan:
%
%       *.bib : ficheros con la informaci?n de las referencias
%
% ---------------------------------------------------------------------

\documentclass[11pt,a4paper,twoside]{book}

%
% Definimos  el   comando  \compilaCapitulo,  que   luego  se  utiliza
% (opcionalmente) en config.tex. Quedar?a  mejor si tambi?n se definiera
% en  ese fichero,  pero por  el modo  en el  que funciona  eso  no es
% posible. Puedes consultar la documentaci?n de ese fichero para tener
% m?s  informaci?n. Definimos tambi?n  \compilaApendice, que  tiene el
% mismo  cometido, pero  que se  utiliza para  compilar  ?nicamente un
% ap?ndice.
%
%
% Si  queremos   compilar  solo   una  parte  del   documento  podemos
% especificar mediante  \includeonly{...} qu? ficheros  son los ?nicos
% que queremos  que se incluyan.  Esto  es ?til por  ejemplo para s?lo
% compilar un cap?tulo.
%
% El problema es que todos aquellos  ficheros que NO est?n en la lista
% NO   se  incluir?n...  y   eso  tambi?n   afecta  a   ficheros  de
% la plantilla...
%
% Total,  que definimos  una constante  con los  ficheros  que siempre
% vamos a querer compilar  (aquellos relacionados con configuraci?n) y
% luego definimos \compilaCapitulo.
\newcommand{\ficherosBasicosTeXiS}{%
TeXiS/TeXiS_pream,TeXiS/TeXiS_cab,TeXiS/TeXiS_bib,TeXiS/TeXiS_cover,%
TeXiS/TeXiS_part%
}
\newcommand{\ficherosBasicosTexto}{%
constantes,guionado,Cascaras/bibliografia,config%
}
\newcommand{\compilaCapitulo}[1]{%
\includeonly{\ficherosBasicosTeXiS,\ficherosBasicosTexto,Capitulos/#1}%
}

\newcommand{\compilaApendice}[1]{%
\includeonly{\ficherosBasicosTeXiS,\ficherosBasicosTexto,Apendices/#1}%
}

%- - - - - - - - - - - - - - - - - - - - - - - - - - - - - - - - - - -
%            Pre?mbulo del documento. Configuraciones varias
%- - - - - - - - - - - - - - - - - - - - - - - - - - - - - - - - - - -

% Define  el  tipo  de  compilaci?n que  estamos  haciendo.   Contiene
% definiciones  de  constantes que  cambian  el  comportamiento de  la
% compilaci?n. Debe incluirse antes del paquete TeXiS/TeXiS.sty
%---------------------------------------------------------------------
%
%                          config.tex
%
%---------------------------------------------------------------------
%
% Contiene la  definici�n de constantes  que determinan el modo  en el
% que se compilar� el documento.
%
%---------------------------------------------------------------------
%
% En concreto, podemos  indicar si queremos "modo release",  en el que
% no  aparecer�n  los  comentarios  (creados  mediante  \com{Texto}  o
% \comp{Texto}) ni los "por  hacer" (creados mediante \todo{Texto}), y
% s� aparecer�n los �ndices. El modo "debug" (o mejor dicho en modo no
% "release" muestra los �ndices  (construirlos lleva tiempo y son poco
% �tiles  salvo  para   la  versi�n  final),  pero  s�   el  resto  de
% anotaciones.
%
% Si se compila con LaTeX (no  con pdflatex) en modo Debug, tambi�n se
% muestran en una esquina de cada p�gina las entradas (en el �ndice de
% palabras) que referencian  a dicha p�gina (consulta TeXiS_pream.tex,
% en la parte referente a show).
%
% El soporte para  el �ndice de palabras en  TeXiS es embrionario, por
% lo  que no  asumas que  esto funcionar�  correctamente.  Consulta la
% documentaci�n al respecto en TeXiS_pream.tex.
%
%
% Tambi�n  aqu� configuramos  si queremos  o  no que  se incluyan  los
% acr�nimos  en el  documento final  en la  versi�n release.  Para eso
% define (o no) la constante \acronimosEnRelease.
%
% Utilizando \compilaCapitulo{nombre}  podemos tambi�n especificar qu�
% cap�tulo(s) queremos que se compilen. Si no se pone nada, se compila
% el documento  completo.  Si se pone, por  ejemplo, 01Introduccion se
% compilar� �nicamente el fichero Capitulos/01Introduccion.tex
%
% Para compilar varios  cap�tulos, se separan sus nombres  con comas y
% no se ponen espacios de separaci�n.
%
% En realidad  la macro \compilaCapitulo  est� definida en  el fichero
% principal tesis.tex.
%
%---------------------------------------------------------------------


% Comentar la l�nea si no se compila en modo release.
% TeXiS har� el resto.
% ���Si cambias esto, haz un make clean antes de recompilar!!!
\def\release{1}


% Descomentar la linea si se quieren incluir los
% acr�nimos en modo release (en modo debug
% no se incluir�n nunca).
% ���Si cambias esto, haz un make clean antes de recompilar!!!
\def\acronimosEnRelease{1}


% Descomentar la l�nea para establecer el cap�tulo que queremos
% compilar

% \compilaCapitulo{01Introduccion}
% \compilaCapitulo{02EstructuraYGeneracion}
% \compilaCapitulo{03Edicion}
% \compilaCapitulo{04Imagenes}
% \compilaCapitulo{05Bibliografia}
% \compilaCapitulo{06Makefile}

% \compilaApendice{01AsiSeHizo}

% Variable local para emacs, para  que encuentre el fichero maestro de
% compilaci�n y funcionen mejor algunas teclas r�pidas de AucTeX
%%%
%%% Local Variables:
%%% mode: latex
%%% TeX-master: "./Tesis.tex"
%%% End:


% Paquete de la plantilla
\usepackage{TeXiS/TeXiS}

% Incluimos el fichero con comandos de constantes
%---------------------------------------------------------------------
%
%                          constantes.tex
%
%---------------------------------------------------------------------
%
% Fichero que  declara nuevos comandos LaTeX  sencillos realizados por
% comodidad en la escritura de determinadas palabras
%
%---------------------------------------------------------------------

%%%%%%%%%%%%%%%%%%%%%%%%%%%%%%%%%%%%%%%%%%%%%%%%%%%%%%%%%%%%%%%%%%%%%%
% Comando: 
%
%       \titulo
%
% Resultado: 
%
% Escribe el t�tulo del documento.
%%%%%%%%%%%%%%%%%%%%%%%%%%%%%%%%%%%%%%%%%%%%%%%%%%%%%%%%%%%%%%%%%%%%%%
\def\titulo{Sistemas de recomendaci�n enfocados a Jueces en l�nea}

%%%%%%%%%%%%%%%%%%%%%%%%%%%%%%%%%%%%%%%%%%%%%%%%%%%%%%%%%%%%%%%%%%%%%%
% Comando: 
%
%       \autor
%
% Resultado: 
%
% Escribe el autor del documento.
%%%%%%%%%%%%%%%%%%%%%%%%%%%%%%%%%%%%%%%%%%%%%%%%%%%%%%%%%%%%%%%%%%%%%%
\def\autor{Alfonso Soria Mu�oz\\Pedro Domenech}

%%%%%%%%%%%%%%%%%%%%%%%%%%%%%%%%%%%%%%%%%%%%%%%%%%%%%%%%%%%%%%%%%%%%%%
% Comando:
%		\director
% Resultado:
% 
% Escribe los directores del documento.
%%%%%%%%%%%%%%%%%%%%%%%%%%%%%%%%%%%%%%%%%%%%%%%%%%%%%%%%%%%%%%%%%%%%%%
\def\director{Pedro Pablo G�mez Mart�n y Marco Antonio G�mez Mart�n}
% Variable local para emacs, para  que encuentre el fichero maestro de
% compilaci�n y funcionen mejor algunas teclas r�pidas de AucTeX

%%%
%%% Local Variables:
%%% mode: latex
%%% TeX-master: "tesis.tex"
%%% End:


% Sacamos en el log de la compilaci?n el copyright
\typeout{Copyright Marco Antonio and Pedro Pablo Gomez Martin}

%
% "Metadatos" para el PDF
%
\ifpdf\hypersetup{%
    pdftitle = {\titulo},
    pdfsubject = {Plantilla de Tesis},
    pdfkeywords = {Plantilla, LaTeX, tesis, trabajo de
      investigación, trabajo de Fin de Grado, Alfonso Soria Muñoz, Pedro Domenech},
    pdfauthor = {\textcopyright\ \autor},
    pdfcreator = {\LaTeX\ con el paquete \flqq hyperref\frqq},
    pdfproducer = {pdfeTeX-0.\the\pdftexversion\pdftexrevision},
    }
    \pdfinfo{/CreationDate (\today)}
\fi


%- - - - - - - - - - - - - - - - - - - - - - - - - - - - - - - - - - -
%                        Documento
%- - - - - - - - - - - - - - - - - - - - - - - - - - - - - - - - - - -
\begin{document}

% Incluimos el  fichero de definici?n de guionado  de algunas palabras
% que LaTeX no ha dividido como deber?a
%----------------------------------------------------------------
%
%                          guionado.tex
%
%----------------------------------------------------------------
%
% Fichero con algunas divisiones de palabras que LaTeX no
% hace correctamente si no se le da alguna ayuda.
%
%----------------------------------------------------------------

\hyphenation{
% a
abs-trac-to
abs-trac-tos
abs-trac-ta
abs-trac-tas
ac-tua-do-res
a-gra-de-ci-mien-tos
ana-li-za-dor
an-te-rio-res
an-te-rior-men-te
apa-rien-cia
a-pro-pia-do
a-pro-pia-dos
a-pro-pia-da
a-pro-pia-das
a-pro-ve-cha-mien-to
a-que-llo
a-que-llos
a-que-lla
a-que-llas
a-sig-na-tu-ra
a-sig-na-tu-ras
a-so-cia-da
a-so-cia-das
a-so-cia-do
a-so-cia-dos
au-to-ma-ti-za-do
% b
batch
bi-blio-gra-f�a
bi-blio-gr�-fi-cas
bien
bo-rra-dor
boo-l-ean-expr
% c
ca-be-ce-ra
call-me-thod-ins-truc-tion
cas-te-lla-no
cir-cuns-tan-cia
cir-cuns-tan-cias
co-he-ren-te
co-he-ren-tes
co-he-ren-cia
co-li-bri
co-men-ta-rio
co-mer-cia-les
co-no-ci-mien-to
cons-cien-te
con-si-de-ra-ba
con-si-de-ra-mos
con-si-de-rar-se
cons-tan-te
cons-trucci�n
cons-tru-ye
cons-tru-ir-se
con-tro-le
co-rrec-ta-men-te
co-rres-pon-den
co-rres-pon-dien-te
co-rres-pon-dien-tes
co-ti-dia-na
co-ti-dia-no
crean
cris-ta-li-zan
cu-rri-cu-la
cu-rri-cu-lum
cu-rri-cu-lar
cu-rri-cu-la-res
% d
de-di-ca-do
de-di-ca-dos
de-di-ca-da
de-di-ca-das
de-rro-te-ro
de-rro-te-ros
de-sa-rro-llo
de-sa-rro-llos
de-sa-rro-lla-do
de-sa-rro-lla-dos
de-sa-rro-lla-da
de-sa-rro-lla-das
de-sa-rro-lla-dor
de-sa-rro-llar
des-cri-bi-re-mos
des-crip-ci�n
des-crip-cio-nes
des-cri-to
des-pu�s
de-ta-lla-do
de-ta-lla-dos
de-ta-lla-da
de-ta-lla-das
di-a-gra-ma
di-a-gra-mas
di-se-�os
dis-po-ner
dis-po-ni-bi-li-dad
do-cu-men-ta-da
do-cu-men-to
do-cu-men-tos
% e
edi-ta-do
e-du-ca-ti-vo
e-du-ca-ti-vos
e-du-ca-ti-va
e-du-ca-ti-vas
e-la-bo-ra-do
e-la-bo-ra-dos
e-la-bo-ra-da
e-la-bo-ra-das
es-co-llo
es-co-llos
es-tu-dia-do
es-tu-dia-dos
es-tu-dia-da
es-tu-dia-das
es-tu-dian-te
e-va-lua-cio-nes
e-va-lua-do-res
exis-ten-tes
exhaus-ti-va
ex-pe-rien-cia
ex-pe-rien-cias
% f
for-ma-li-za-do
% g
ge-ne-ra-ci�n
ge-ne-ra-dor
ge-ne-ra-do-res
ge-ne-ran
% h
he-rra-mien-ta
he-rra-mien-tas
% i
i-dio-ma
i-dio-mas
im-pres-cin-di-ble
im-pres-cin-di-bles
in-de-xa-do
in-de-xa-dos
in-de-xa-da
in-de-xa-das
in-di-vi-dual
in-fe-ren-cia
in-fe-ren-cias
in-for-ma-ti-ca
in-gre-dien-te
in-gre-dien-tes
in-me-dia-ta-men-te
ins-ta-la-do
ins-tan-cias
% j
% k
% l
len-gua-je
li-be-ra-to-rio
li-be-ra-to-rios
li-be-ra-to-ria
li-be-ra-to-rias
li-mi-ta-do
li-te-ra-rio
li-te-ra-rios
li-te-ra-ria
li-te-ra-rias
lo-tes
% m
ma-ne-ra
ma-nual
mas-que-ra-de
ma-yor
me-mo-ria
mi-nis-te-rio
mi-nis-te-rios
mo-de-lo
mo-de-los
mo-de-la-do
mo-du-la-ri-dad
mo-vi-mien-to
% n
na-tu-ral
ni-vel
nues-tro
% o
obs-tan-te
o-rien-ta-do
o-rien-ta-dos
o-rien-ta-da
o-rien-ta-das
% p
pa-ra-le-lo
pa-ra-le-la
par-ti-cu-lar
par-ti-cu-lar-men-te
pe-da-g�-gi-ca
pe-da-g�-gi-cas
pe-da-g�-gi-co
pe-da-g�-gi-cos
pe-rio-di-ci-dad
per-so-na-je
plan-te-a-mien-to
plan-te-a-mien-tos
po-si-ci�n
pre-fe-ren-cia
pre-fe-ren-cias
pres-cin-di-ble
pres-cin-di-bles
pri-me-ra
pro-ble-ma
pro-ble-mas
pr�-xi-mo
pu-bli-ca-cio-nes
pu-bli-ca-do
% q
% r
r�-pi-da
r�-pi-do
ra-zo-na-mien-to
ra-zo-na-mien-tos
re-a-li-zan-do
re-fe-ren-cia
re-fe-ren-cias
re-fe-ren-cia-da
re-fe-ren-cian
re-le-van-tes
re-pre-sen-ta-do
re-pre-sen-ta-dos
re-pre-sen-ta-da
re-pre-sen-ta-das
re-pre-sen-tar-lo
re-qui-si-to
re-qui-si-tos
res-pon-der
res-pon-sa-ble
% s
se-pa-ra-do
si-guien-do
si-guien-te
si-guien-tes
si-guie-ron
si-mi-lar
si-mi-la-res
si-tua-ci�n
% t
tem-pe-ra-ments
te-ner
trans-fe-ren-cia
trans-fe-ren-cias
% u
u-sua-rio
Unreal-Ed
% v
va-lor
va-lo-res
va-rian-te
ver-da-de-ro
ver-da-de-ros
ver-da-de-ra
ver-da-de-ras
ver-da-de-ra-men-te
ve-ri-fi-ca
% w
% x
% y
% z
}
% Variable local para emacs, para que encuentre el fichero
% maestro de compilaci�n
%%%
%%% Local Variables:
%%% mode: latex
%%% TeX-master: "./Tesis.tex"
%%% End:


% Marcamos  el inicio  del  documento para  la  numeraci?n de  p?ginas
% (usando n?meros romanos para esta primera fase).
\frontmatter

%---------------------------------------------------------------------
%
%                          configCover.tex
%
%---------------------------------------------------------------------
%
% cover.tex
% Copyright 2009 Marco Antonio Gomez-Martin, Pedro Pablo Gomez-Martin
%
% This file belongs to the TeXiS manual, a LaTeX template for writting
% Thesis and other documents. The complete last TeXiS package can
% be obtained from http://gaia.fdi.ucm.es/projects/texis/
%
% Although the TeXiS template itself is distributed under the 
% conditions of the LaTeX Project Public License
% (http://www.latex-project.org/lppl.txt), the manual content
% uses the CC-BY-SA license that stays that you are free:
%
%    - to share & to copy, distribute and transmit the work
%    - to remix and to adapt the work
%
% under the following conditions:
%
%    - Attribution: you must attribute the work in the manner
%      specified by the author or licensor (but not in any way that
%      suggests that they endorse you or your use of the work).
%    - Share Alike: if you alter, transform, or build upon this
%      work, you may distribute the resulting work only under the
%      same, similar or a compatible license.
%
% The complete license is available in
% http://creativecommons.org/licenses/by-sa/3.0/legalcode
%
%---------------------------------------------------------------------
%
% Fichero que contiene la configuraci�n de la portada y de la 
% primera hoja del documento.
%
%---------------------------------------------------------------------


% Pueden configurarse todos los elementos del contenido de la portada
% utilizando comandos.

%%%%%%%%%%%%%%%%%%%%%%%%%%%%%%%%%%%%%%%%%%%%%%%%%%%%%%%%%%%%%%%%%%%%%%
% T�tulo del documento:
% \tituloPortada{titulo}
% Nota:
% Si no se define se utiliza el del \titulo. Este comando permite
% cambiar el t�tulo de forma que se especifiquen d�nde se quieren
% los retornos de carro cuando se utilizan fuentes grandes.
%%%%%%%%%%%%%%%%%%%%%%%%%%%%%%%%%%%%%%%%%%%%%%%%%%%%%%%%%%%%%%%%%%%%%%
\tituloPortada{%
Sistemas de recomendaci�n enfocados\\a Jueces en l�nea
}

%%%%%%%%%%%%%%%%%%%%%%%%%%%%%%%%%%%%%%%%%%%%%%%%%%%%%%%%%%%%%%%%%%%%%%
% Autor del documento:
% \autorPortada{Nombre}
% Se utiliza en la portada y en el valor por defecto del
% primer subt�tulo de la segunda portada.
%%%%%%%%%%%%%%%%%%%%%%%%%%%%%%%%%%%%%%%%%%%%%%%%%%%%%%%%%%%%%%%%%%%%%%
\autorPortada{Alfonso Soria Mu�oz\\Pedro Domenech}

%%%%%%%%%%%%%%%%%%%%%%%%%%%%%%%%%%%%%%%%%%%%%%%%%%%%%%%%%%%%%%%%%%%%%%
% Fecha de publicaci�n:
% \fechaPublicacion{Fecha}
% Puede ser vac�o. Aparece en la �ltima l�nea de ambas portadas
%%%%%%%%%%%%%%%%%%%%%%%%%%%%%%%%%%%%%%%%%%%%%%%%%%%%%%%%%%%%%%%%%%%%%%
\fechaPublicacion{Mayo 2019}

%%%%%%%%%%%%%%%%%%%%%%%%%%%%%%%%%%%%%%%%%%%%%%%%%%%%%%%%%%%%%%%%%%%%%%
% Imagen de la portada (y escala)
% \imagenPortada{Fichero}
% \escalaImagenPortada{Numero}
% Si no se especifica, se utiliza la imagen TODO.pdf
%%%%%%%%%%%%%%%%%%%%%%%%%%%%%%%%%%%%%%%%%%%%%%%%%%%%%%%%%%%%%%%%%%%%%%
\imagenPortada{Imagenes/Vectorial/escudoUCM}
\escalaImagenPortada{.2}

%%%%%%%%%%%%%%%%%%%%%%%%%%%%%%%%%%%%%%%%%%%%%%%%%%%%%%%%%%%%%%%%%%%%%%
% Tipo de documento.
% \tipoDocumento{Tipo}
% Para el texto justo debajo del escudo.
% Si no se indica, se utiliza "TESIS DOCTORAL".
%%%%%%%%%%%%%%%%%%%%%%%%%%%%%%%%%%%%%%%%%%%%%%%%%%%%%%%%%%%%%%%%%%%%%%
\tipoDocumento{TRABAJO DE FIN DE GRADO}

%%%%%%%%%%%%%%%%%%%%%%%%%%%%%%%%%%%%%%%%%%%%%%%%%%%%%%%%%%%%%%%%%%%%%%
% Instituci�n/departamento asociado al documento.
% \institucion{Nombre}
% Puede tener varias l�neas. Se utiliza en las dos portadas.
% Si no se indica aparecer� vac�o.
%%%%%%%%%%%%%%%%%%%%%%%%%%%%%%%%%%%%%%%%%%%%%%%%%%%%%%%%%%%%%%%%%%%%%%
\institucion{%
Grado de Ingenier�a Inform�tica\\[0.2em]
Facultad de Inform�tica\\[0.2em]
Universidad Complutense de Madrid
}

%%%%%%%%%%%%%%%%%%%%%%%%%%%%%%%%%%%%%%%%%%%%%%%%%%%%%%%%%%%%%%%%%%%%%%
% Director del trabajo.
% \directorPortada{Nombre}
% Se utiliza para el valor por defecto del segundo subt�tulo, donde
% se indica qui�n es el director del trabajo.
% Si se fuerza un subt�tulo distinto, no hace falta definirlo.
%%%%%%%%%%%%%%%%%%%%%%%%%%%%%%%%%%%%%%%%%%%%%%%%%%%%%%%%%%%%%%%%%%%%%%
%\directorPortada{Walterio Malatesta}

%%%%%%%%%%%%%%%%%%%%%%%%%%%%%%%%%%%%%%%%%%%%%%%%%%%%%%%%%%%%%%%%%%%%%%
% Texto del primer subt�tulo de la segunda portada.
% \textoPrimerSubtituloPortada{Texto}
% Para configurar el primer "texto libre" de la segunda portada.
% Si no se especifica se indica "Memoria que presenta para optar al
% t�tulo de Doctor en Inform�tica" seguido del \autorPortada.
%%%%%%%%%%%%%%%%%%%%%%%%%%%%%%%%%%%%%%%%%%%%%%%%%%%%%%%%%%%%%%%%%%%%%%
\textoPrimerSubtituloPortada{%
\textit{Memoria que presenta para optar al t�tulo de grado en Ingenier�a Inform�tica}  \\ [0.3em]
\textbf{Grado en Ingenier�a Inform�tica} \\ [0.3em]
\textbf{IT/2019/05}
}

%%%%%%%%%%%%%%%%%%%%%%%%%%%%%%%%%%%%%%%%%%%%%%%%%%%%%%%%%%%%%%%%%%%%%%
% Texto del segundo subt�tulo de la segunda portada.
% \textoSegundoSubtituloPortada{Texto}
% Para configurar el segundo "texto libre" de la segunda portada.
% Si no se especifica se indica "Dirigida por el Doctor" seguido
% del \directorPortada.
%%%%%%%%%%%%%%%%%%%%%%%%%%%%%%%%%%%%%%%%%%%%%%%%%%%%%%%%%%%%%%%%%%%%%%
\textoSegundoSubtituloPortada{%
\textit{Dirigida por los Doctores \director}
}

%%%%%%%%%%%%%%%%%%%%%%%%%%%%%%%%%%%%%%%%%%%%%%%%%%%%%%%%%%%%%%%%%%%%%%
% \explicacionDobleCara
% Si se utiliza, se aclara que el documento est� preparado para la
% impresi�n a doble cara.
%%%%%%%%%%%%%%%%%%%%%%%%%%%%%%%%%%%%%%%%%%%%%%%%%%%%%%%%%%%%%%%%%%%%%%
\explicacionDobleCara

%%%%%%%%%%%%%%%%%%%%%%%%%%%%%%%%%%%%%%%%%%%%%%%%%%%%%%%%%%%%%%%%%%%%%%
% \isbn
% Si se utiliza, aparecer� el ISBN detr�s de la segunda portada.
%%%%%%%%%%%%%%%%%%%%%%%%%%%%%%%%%%%%%%%%%%%%%%%%%%%%%%%%%%%%%%%%%%%%%%
%\isbn{978-84-692-7109-4}


%%%%%%%%%%%%%%%%%%%%%%%%%%%%%%%%%%%%%%%%%%%%%%%%%%%%%%%%%%%%%%%%%%%%%%
% \copyrightInfo
% Si se utiliza, aparecer� informaci�n de los derechos de copyright
% detr�s de la segunda portada.
%%%%%%%%%%%%%%%%%%%%%%%%%%%%%%%%%%%%%%%%%%%%%%%%%%%%%%%%%%%%%%%%%%%%%%
\copyrightInfo{\autor}


%%
%% Creamos las portadas
%%
\makeCover

% Variable local para emacs, para que encuentre el fichero
% maestro de compilaci�n
%%%
%%% Local Variables:
%%% mode: latex
%%% TeX-master: "../Tesis.tex"
%%% End:


%---------------------------------------------------------------------
%
%                      dedicatoria.tex
%
%---------------------------------------------------------------------
%
% dedicatoria.tex
% Copyright 2009 Marco Antonio Gomez-Martin, Pedro Pablo Gomez-Martin
%
% This file belongs to the TeXiS manual, a LaTeX template for writting
% Thesis and other documents. The complete last TeXiS package can
% be obtained from http://gaia.fdi.ucm.es/projects/texis/
%
% Although the TeXiS template itself is distributed under the 
% conditions of the LaTeX Project Public License
% (http://www.latex-project.org/lppl.txt), the manual content
% uses the CC-BY-SA license that stays that you are free:
%
%    - to share & to copy, distribute and transmit the work
%    - to remix and to adapt the work
%
% under the following conditions:
%
%    - Attribution: you must attribute the work in the manner
%      specified by the author or licensor (but not in any way that
%      suggests that they endorse you or your use of the work).
%    - Share Alike: if you alter, transform, or build upon this
%      work, you may distribute the resulting work only under the
%      same, similar or a compatible license.
%
% The complete license is available in
% http://creativecommons.org/licenses/by-sa/3.0/legalcode
%
%---------------------------------------------------------------------
%
% Contiene la p�gina de dedicatorias.
%
%---------------------------------------------------------------------

\dedicatoriaUno{%
\emph{
Al duque de B�jar\\
y\hspace*{10ex} \\
a t�, lector car�simo%
}%
}

\dedicatoriaDos{%
\emph{%
I can't go to a restaurant and\\%
order food because I keep looking\\%
at the fonts on the menu.\\%
Donald Knuth%
}%
}

\makeDedicatorias

% Variable local para emacs, para que encuentre el fichero
% maestro de compilaci�n
%%%
%%% Local Variables:
%%% mode: latex
%%% TeX-master: "../Tesis.tex"
%%% End:


%---------------------------------------------------------------------
%
%                      agradecimientos.tex
%
%---------------------------------------------------------------------
%
% agradecimientos.tex
% Copyright 2009 Marco Antonio Gomez-Martin, Pedro Pablo Gomez-Martin
%
% This file belongs to the TeXiS manual, a LaTeX template for writting
% Thesis and other documents. The complete last TeXiS package can
% be obtained from http://gaia.fdi.ucm.es/projects/texis/
%
% Although the TeXiS template itself is distributed under the 
% conditions of the LaTeX Project Public License
% (http://www.latex-project.org/lppl.txt), the manual content
% uses the CC-BY-SA license that stays that you are free:
%
%    - to share & to copy, distribute and transmit the work
%    - to remix and to adapt the work
%
% under the following conditions:
%
%    - Attribution: you must attribute the work in the manner
%      specified by the author or licensor (but not in any way that
%      suggests that they endorse you or your use of the work).
%    - Share Alike: if you alter, transform, or build upon this
%      work, you may distribute the resulting work only under the
%      same, similar or a compatible license.
%
% The complete license is available in
% http://creativecommons.org/licenses/by-sa/3.0/legalcode
%
%---------------------------------------------------------------------
%
% Contiene la p�gina de agradecimientos.
%
% Se crea como un cap�tulo sin numeraci�n.
%
%---------------------------------------------------------------------

\chapter{Agradecimientos}

\cabeceraEspecial{Agradecimientos}

\begin{FraseCelebre}
\begin{Frase}
A todos los que la presente vieren y entendieren.
\end{Frase}
\begin{Fuente}
Inicio de las Leyes Org�nicas. Juan Carlos I
\end{Fuente}
\end{FraseCelebre}

Groucho Marx dec�a que encontraba a la televisi�n muy educativa porque
cada vez que alguien la encend�a, �l se iba a otra habitaci�n a leer
un libro. Utilizando un esquema similar, nosotros queremos agradecer
al Word de Microsoft el habernos forzado a utilizar \LaTeX. Cualquiera
que haya intentado escribir un documento de m�s de 150 p�ginas con
esta aplicaci�n entender� a qu� nos referimos. Y lo decimos porque
nuestra andadura con \LaTeX\ comenz�, precisamente, despu�s de
escribir un documento de algo m�s de 200 p�ginas. Una vez terminado
decidimos que nunca m�s pasar�amos por ah�. Y entonces ca�mos en
\LaTeX.

Es muy posible que hub�eramos llegado al mismo sitio de todas formas,
ya que en el mundo acad�mico a la hora de escribir art�culos y
contribuciones a congresos lo m�s extendido es \LaTeX. Sin embargo,
tambi�n es cierto que cuando intentas escribir un documento grande
en \LaTeX\ por tu cuenta y riesgo sin un enlace del tipo ``\emph{Author
  instructions}'', se hace cuesta arriba, pues uno no sabe por donde
empezar.

Y ah� es donde debemos agradecer tanto a Pablo Gerv�s como a Miguel
Palomino su ayuda. El primero nos ofreci� el c�digo fuente de una
programaci�n docente que hab�a hecho unos a�os atr�s y que nos sirvi�
de inspiraci�n (por ejemplo, el fichero \texttt{guionado.tex} de
\texis\ tiene una estructura casi exacta a la suya e incluso puede
que el nombre sea el mismo). El segundo nos dej� husmear en el c�digo
fuente de su propia tesis donde, adem�s de otras cosas m�s
interesantes pero menos curiosas, descubrimos que a�n hay gente que
escribe los acentos espa�oles con el \verb+\'{\i}+.

No podemos tampoco olvidar a los numerosos autores de los libros y
tutoriales de \LaTeX\ que no s�lo permiten descargar esos manuales sin
coste adicional, sino que tambi�n dejan disponible el c�digo fuente.
Estamos pensando en Tobias Oetiker, Hubert Partl, Irene Hyna y
Elisabeth Schlegl, autores del famoso ``The Not So Short Introduction
to \LaTeXe'' y en Tom�s Bautista, autor de la traducci�n al espa�ol. De
ellos es, entre otras muchas cosas, el entorno \texttt{example}
utilizado en algunos momentos en este manual.

Tambi�n estamos en deuda con Joaqu�n Ataz L�pez, autor del libro
``Creaci�n de ficheros \LaTeX\ con {GNU} Emacs''. Gracias a �l dejamos
de lado a WinEdt y a Kile, los editores que por entonces utiliz�bamos
en entornos Windows y Linux respectivamente, y nos pasamos a emacs. El
tiempo de escritura que nos ahorramos por no mover las manos del
teclado para desplazar el cursor o por no tener que escribir
\verb+\emph+ una y otra vez se lo debemos a �l; nuestro ocio y vida
social se lo agradecen.

Por �ltimo, gracias a toda esa gente creadora de manuales, tutoriales,
documentaci�n de paquetes o respuestas en foros que hemos utilizado y
seguiremos utilizando en nuestro quehacer como usuarios de
\LaTeX. Sab�is un mont�n.

Y para terminar, a Donal Knuth, Leslie Lamport y todos los que hacen y
han hecho posible que hoy puedas estar leyendo estas l�neas.

\endinput
% Variable local para emacs, para  que encuentre el fichero maestro de
% compilaci�n y funcionen mejor algunas teclas r�pidas de AucTeX
%%%
%%% Local Variables:
%%% mode: latex
%%% TeX-master: "../Tesis.tex"
%%% End:


%---------------------------------------------------------------------
%
%                      resumenManual.tex
%
%---------------------------------------------------------------------
%
% resumenManual.tex
% Copyright 2009 Marco Antonio Gomez-Martin, Pedro Pablo Gomez-Martin
%
% This file belongs to the TeXiS manual, a LaTeX template for writting
% Thesis and other documents. The complete last TeXiS package can
% be obtained from http://gaia.fdi.ucm.es/projects/texis/
%
% Although the TeXiS template itself is distributed under the 
% conditions of the LaTeX Project Public License
% (http://www.latex-project.org/lppl.txt), the manual content
% uses the CC-BY-SA license that stays that you are free:
%
%    - to share & to copy, distribute and transmit the work
%    - to remix and to adapt the work
%
% under the following conditions:
%
%    - Attribution: you must attribute the work in the manner
%      specified by the author or licensor (but not in any way that
%      suggests that they endorse you or your use of the work).
%    - Share Alike: if you alter, transform, or build upon this
%      work, you may distribute the resulting work only under the
%      same, similar or a compatible license.
%
% The complete license is available in
% http://creativecommons.org/licenses/by-sa/3.0/legalcode
%
%---------------------------------------------------------------------
%
% Contiene el cap�tulo del resumen.
%
% Se crea como un cap�tulo sin numeraci�n.
%
%---------------------------------------------------------------------

\chapter{Resumen}
\cabeceraEspecial{Resumen}

\begin{FraseCelebre}
\begin{Frase}
  Desocupado lector,  sin juramento me  podr�s creer que  quisiera que
  este  libro  [...] fuera  el  m�s hermoso,  el  m�s  gallardo y  m�s
  discreto que pudiera imaginarse.
\end{Frase}
\begin{Fuente}
  Miguel de Cervantes, Don Quijote de la Mancha
\end{Fuente}
\end{FraseCelebre}

\texis\ es un conjunto de ficheros \LaTeX\ que pueden servir para
escribir tesis doctorales, trabajos de fin de master, de fin de
carrera y otros documentos del mismo estilo. El documento que tienes
en tus manos es un manual que explica las distintas caracter�sticas de
la plantilla. En los distintos cap�tulos iremos explicando los
ficheros existentes en \texis\ as� como su funci�n. Tambi�n se
explican algunas de las caracter�sticas, como por ejemplo ciertos
comandos que facilitan la escritura de los documentos.

Aunque el c�digo \LaTeX\ utilizado en \texis\ est� muy comentado
para su uso f�cil, creemos que las explicaciones que aqu� se
proporcionan pueden ser �tiles.

Hay dos distribuciones distintas de \texis: el c�digo fuente completo
de este manual (de forma que \texis\ es ``\emph{su propio
  manual}''\footnote{Los expertos en l�gica seguro que tendr�an algo
  que decir al respecto...}), o una distribuci�n casi ``vac�a de
contenido'', que tiene un �nico cap�tulo y ap�ndice vac�o, pero
mantiene la portada, dedicatoria, agradecimientos y bibliograf�a del
manual.

Dependiendo, pues, de qu� distribuci�n escojas, partir�s directamente
de los ficheros \texttt{.tex} de este manual y eliminar�s su texto
para a�dir el tuyo, o de un conjunto de ficheros sin apenas contenido
que rellenar�s. Aconsejamos esta �ltima aproximaci�n por ser m�s
c�moda. Sin embargo, hacemos disponible los ficheros \texttt{.tex} del
manual como referencia.

Para facilitar las cosas, hemos intentado que su estructura sea
parecida a la de una posible tesis. De esta forma el c�digo fuente del
propio manual puede servir como punto de partida para la escritura de
este tipo de documentos. Como podr�s comprobar, en alg�n momento nos
ha sido dif�cil justificar la existencia de ciertos elementos pues no
eran realmente relevantes para el manual. En esos casos, piensa que
est�n ah� no porque sean importantes desde el punto de vista de
\emph{este} documento, sino porque muy posiblemente estar�an en el
tipo de textos para los que \texis\ es �til.

\medskip

Al estar compuesto por varios tipos de ficheros, \texis\ se rige por varias licencias:

\medskip

\begin{center}
\begin{tabular}[h]{m{3.5cm}m{8cm}}
& La plantilla (ficheros en el directorio \texttt{TeXiS}) se distribuye bajo la \emph{\LaTeX\ Project Public License} (Licencia P�blica del Proyecto \LaTeX). \\
& \\
& \\
\includegraphics[width=3cm]%
                     {Imagenes/Bitmap/GPLv3-logo-red} & Los ficheros \texttt{Makefile} y scripts de apoyo a la generaci�n del documento, se distribuyen bajo licencia GPLv3. \\
& \\
& \\
\includegraphics[width=3cm]%
                     {Imagenes/Vectorial/by-sa} & El \emph{manual} de \texis\ se distribuye con una licencia Creative Commons (CC-BY-SA). \\
\end{tabular}
\end{center}


\endinput
% Variable local para emacs, para  que encuentre el fichero maestro de
% compilaci�n y funcionen mejor algunas teclas r�pidas de AucTeX
%%%
%%% Local Variables:
%%% mode: latex
%%% TeX-master: "../ManualTeXiS.tex"
%%% End:


\ifx\generatoc\undefined
\else
%---------------------------------------------------------------------
%
%                          TeXiS_toc.tex
%
%---------------------------------------------------------------------
%
% TeXiS_toc.tex
% Copyright 2009 Marco Antonio Gomez-Martin, Pedro Pablo Gomez-Martin
%
% This file belongs to TeXiS, a LaTeX template for writting
% Thesis and other documents. The complete last TeXiS package can
% be obtained from http://gaia.fdi.ucm.es/projects/texis/
%
% This work may be distributed and/or modified under the
% conditions of the LaTeX Project Public License, either version 1.3
% of this license or (at your option) any later version.
% The latest version of this license is in
%   http://www.latex-project.org/lppl.txt
% and version 1.3 or later is part of all distributions of LaTeX
% version 2005/12/01 or later.
%
% This work has the LPPL maintenance status `maintained'.
% 
% The Current Maintainers of this work are Marco Antonio Gomez-Martin
% and Pedro Pablo Gomez-Martin
%
%---------------------------------------------------------------------
%
% Contiene  los  comandos  para  generar los  �ndices  del  documento,
% entendiendo por �ndices las tablas de contenidos.
%
% Genera  el  �ndice normal  ("tabla  de  contenidos"),  el �ndice  de
% figuras y el de tablas. Tambi�n  crea "marcadores" en el caso de que
% se est� compilando con pdflatex para que aparezcan en el PDF.
%
%---------------------------------------------------------------------


% Primero un poquito de configuraci�n...


% Pedimos que inserte todos los ep�grafes hasta el nivel \subsection en
% la tabla de contenidos.
\setcounter{tocdepth}{2} 

% Le  pedimos  que nos  numere  todos  los  ep�grafes hasta  el  nivel
% \subsubsection en el cuerpo del documento.
\setcounter{secnumdepth}{3} 


% Creamos los diferentes �ndices.

% Lo primero un  poco de trabajo en los marcadores  del PDF. No quiero
% que  salga una  entrada  por cada  �ndice  a nivel  0...  si no  que
% aparezca un marcador "�ndices", que  tenga dentro los otros tipos de
% �ndices.  Total, que creamos el marcador "�ndices".
% Antes de  la creaci�n  de los �ndices,  se a�aden los  marcadores de
% nivel 1.

\ifpdf
   \pdfbookmark{�ndices}{indices}
\fi

% Tabla de contenidos.
%
% La  inclusi�n  de '\tableofcontents'  significa  que  en la  primera
% pasada  de  LaTeX  se  crea   un  fichero  con  extensi�n  .toc  con
% informaci�n sobre la tabla de contenidos (es conceptualmente similar
% al  .bbl de  BibTeX, creo).  En la  segunda ejecuci�n  de  LaTeX ese
% documento se utiliza para  generar la verdadera p�gina de contenidos
% usando la  informaci�n sobre los  cap�tulos y dem�s guardadas  en el
% .toc
\ifpdf
   \pdfbookmark[1]{Tabla de contenidos}{tabla de contenidos}
\fi

\cabeceraEspecial{\'Indice}

\tableofcontents

\newpage 

% �ndice de figuras
%
% La idea es semejante que para  el .toc del �ndice, pero ahora se usa
% extensi�n .lof (List Of Figures) con la informaci�n de las figuras.

\cabeceraEspecial{\'Indice de figuras}

\ifpdf
   \pdfbookmark[1]{�ndice de figuras}{indice de figuras}
\fi

\listoffigures

\newpage

% �ndice de tablas
% Como antes, pero ahora .lot (List Of Tables)

\ifpdf
   \pdfbookmark[1]{�ndice de tablas}{indice de tablas}
\fi

\cabeceraEspecial{\'Indice de tablas}

\listoftables

\newpage

% Variable local para emacs, para  que encuentre el fichero maestro de
% compilaci�n y funcionen mejor algunas teclas r�pidas de AucTeX

%%%
%%% Local Variables:
%%% mode: latex
%%% TeX-master: "../Tesis.tex"
%%% End:

\fi

% Marcamos el  comienzo de  los cap?tulos (para  la numeraci?n  de las
% p?ginas) y ponemos la cabecera normal
\mainmatter
\restauraCabecera

%---------------------------------------------------------------------
%
%                          Capítulo 1
%
%---------------------------------------------------------------------
%
% 00IntroduccionP.tex
% Copyright 2019 Alfonso Soria Muñoz, Pedro Domenech
%
% This file belongs to the TeXiS manual, a LaTeX template for writting
% Thesis and other documents. The complete last TeXiS package can
% be obtained from http://gaia.fdi.ucm.es/projects/texis/
%
% Although the TeXiS template itself is distributed under the 
% conditions of the LaTeX Project Public License
% (http://www.latex-project.org/lppl.txt), the manual content
% uses the CC-BY-SA license that stays that you are free:
%
%    - to share & to copy, distribute and transmit the work
%    - to remix and to adapt the work
%
% under the following conditions:
%
%    - Attribution: you must attribute the work in the manner
%      specified by the author or licensor (but not in any way that
%      suggests that they endorse you or your use of the work).
%    - Share Alike: if you alter, transform, or build upon this
%      work, you may distribute the resulting work only under the
%      same, similar or a compatible license.
%
% The complete license is available in
% http://creativecommons.org/licenses/by-sa/3.0/legalcode
%
%---------------------------------------------------------------------

\chapter{Introducción}

\begin{FraseCelebre}
\begin{Frase}
Púsose don Quijote delante de dicho carro, y haciendo en su fantasía
uno de los más desvariados discursos que jamás había hecho, dijo en
alta voz:
\end{Frase}
\begin{Fuente}
  Alonso Fernández de Avellaneda, El Ingenioso Hidalgo Don Quijote de
  la Mancha
\end{Fuente}
\end{FraseCelebre}

\begin{resumen}
  Este capítulo presenta una breve introducción a \texis.  El
  lector podrá hacerse una idea de qué es y para qué sirve. También se
  encuentra aquí una descripción del resto de capítulos del manual.
\end{resumen}


%-------------------------------------------------------------------
\section{Introducción}
%-------------------------------------------------------------------
\label{cap1:sec:introduccion}


Si estás leyendo estas líneas es muy posible que haya llegado la hora
de ponerte a escribir la tesis, después de mucho tiempo dando vueltas
al área de investigación concreta en el que estás inmerso. O puede que
estés a punto de empezar a escribir la memoria del proyecto de fin de
carrera, fin de master, o cualquier otro documento de cierta
envergadura.

Sea lo que sea lo que te traes entre manos, lo más probable es que no
sea fácil hacerlo. Muy posiblemente no tengas aún muy claro qué vas a
escribir, pero tu tutor/director/profesor te ha dicho que vayas
empezando a plasmar esas ideas sobre el papel para tener algo firme, y
sentir que vas avanzando.

Y entonces viene el problema de cómo escribirlo. Muy posiblemente
habrás escrito algún artículo en \LaTeX\ y estés convencido de que esa
es la vía a seguir para hacer un documento que superará las 10 páginas
y que tendrá bibliografía. O puede, simplemente, que alguien te haya
dicho que lo mejor es que escribas el proyecto en \LaTeX\ porque la
apariencia final es mejor, porque es más cómodo, o cualquier otra
razón.

Sea como fuere, parece que estás más o menos decidido a escribir tu
documento en \LaTeX. Bien hecho. Pero, ¿cómo?. Al contrario de lo que
suele ocurrir en congresos y en revistas, no tienes disponible ninguna
página en la que descargarte las ``instrucciones para los autores'',
con la cómoda plantilla en \LaTeX\ que tú, sufrido autor, simplemente
tienes que rellenar. No. Ahora las cosas son más complicadas.

Así que te vas a la guía de \LaTeX\ con la que empezaste (apostamos
que es la misma con la que hemos empezado todos), y ves las distintas
posibilidades que te ofrece en su ``\texttt{documentclass}'':
\texttt{article}, \texttt{report}, \texttt{book}, ... Y te quedas con
la última. Pero te asaltan muchas preguntas. ¿Cómo organizo todo esto?
o ¿cómo hago la portada? o incluso ¿qué hago para que no ponga
``Chapter'', sino ``Capítulo''?. En ese punto, es de suponer, has
pedido ayuda a la gente de alrededor y/o a tu buscador de Internet
favorito. Y de alguna forma, te has encontrado leyendo estas líneas.

Tenemos que decir que exactamente esa fue nuestra situación cuando por
fin nos decidimos a escribir nuestras tesis. Desgraciadamente, ni la
gente que teníamos alrededor ni nuestro buscador favorito supieron
contestarnos de forma satisfactoria, por lo que tuvimos que invertir
\emph{mucho tiempo} hasta conseguir que el resultado que salía de
nuestros \texttt{.tex} nos gustara, hasta que nos sentimos cómodos con
la estructura de los ficheros, con las macros disponibles y con el
modo de compilación.

Y para que nadie más pueda utilizar como excusa el no saber cómo
personalizar la clase \texttt{book} para retrasar el comienzo de su
tesis, para que nadie más se decida por Word u otro paquete ofimático
en vez de \LaTeX\ porque lo ve mucho más sencillo, en definitiva, para
que nadie pierda tanto tiempo como perdimos nosotros creando la
estructura, decidimos hacer público el esqueleto básico que
construimos nosotros para hacerlas. Ese esqueleto básico o plantilla
es \texis.

En vez de hacer disponible la plantilla o ficheros \texttt{.tex} sin
ningún contenido, proporcionamos un manual en formato PDF que (a no
ser que estés leyendo directamente el código \LaTeX), será lo que
estás leyendo. Este manual ha sido creado \emph{con la propia
  plantilla}. Por lo tanto, la distribución de \texis\ es en
realidad el código fuente de \emph{su propio manual}. Con su código
fuente entre tus manos, lo único que tienes que hacer es borrar su
contenido (\emph{este texto}), y rellenarlo con tu gran contribución
al mundo.  Como podrás comprobar, la estructura del propio manual
sigue el esquema de lo que podría ser una tesis, trabajo de
investigación o proyecto de fin de carrera, precisamente para que sea
fácil quitar el contenido textual y sustituirlo por el nuevo.

En los capítulos que siguen encontrarás toda la información necesaria
para poder utilizar los ficheros \LaTeX\ para crear tus propios
documentos. Además, el propio código fuente está lleno de comentarios
(especialmente en los ficheros que definen el estilo), por lo que
también en ellos encontrarás una buena fuente de información. Eso es
especialmente importante en caso de que quieras modificar en algo el
aspecto final de tu documento.

Esperemos que te sea de utilidad. Si es así, nos gustaría que lo
reconocieras en la sección de agradecimientos. Si durante tu proceso
de escritura has añadido algún aspecto que crees que puede ser
interesante para otros, no dudes en decírnoslo para intentar incluirlo
en siguientes versiones de la propia plantilla; tampoco dudes en
enviarnos sugerencias sobre las explicaciones de este manual para
poder mejorarlo con el tiempo. Por último, también puedes enviarnos el
resultado final para poner una referencia a él en la página de
descarga, donde, por cierto, puedes ver otros documentos creados con
la plantilla, lo que te permitirá coger ideas de cosas que puedes
variar. Recuerda que la versión más reciente de \texis\ está
disponible en \url{http://gaia.fdi.ucm.es/projects/texis/}.

%-------------------------------------------------------------------
\section{Qué es \texis}
%-------------------------------------------------------------------
\label{cap1:sec:que-es}

La plantilla que tienes entre las manos es, como hemos dicho, el
esqueleto del código fuente de las Tesis Doctorales de los dos autores
\citep{GomezMartinMA2008PhD, GomezMartinPP2008PhD}. Por tanto, sirve
para escribir otras Tesis Doctorales u otros documentos con estructura
similar de forma fácil.

\texis\ te permite además generar el fichero utilizando tanto el
comando \texttt{latex} (que genera de forma nativa ficheros
\texttt{dvi} que luego se convierten a ficheros \texttt{ps} o
\texttt{pdf}), como \texttt{pdflatex}. De esta forma el usuario final
puede elegir entre cualquiera de las dos herramientas\footnote{Esto es
  útil por ejemplo cuando quieres utilizar \texttt{pdflatex} pero
  finalmente el servicio de publicaciones sólo admite el uso de
  \texttt{latex}.}.  Aconsejamos, no obstante, la utilización de este
último, debido a que \texis\ contiene ciertos comandos para dotar al
PDF final de marcadores que permiten una navegación cómoda por el
fichero utilizando los visores tradicionales.

\medskip

Como explicaremos en el capítulo siguiente, la plantilla se aprovecha
mejor en sistemas GNU/Linux. Nota que hemos dicho que la plantilla
``\emph{se aprovecha mejor}'' en sistemas GNU/Linux, no que \emph{no
pueda utilizarse} en Windows o Mac; es evidente que \LaTeX\ es
multiplataforma, y por lo tanto puede compilarse en cualquier sistema
que tenga instalada una distribución del mismo.

La razón por esta ``desviación positiva'' hacia Linux estriba en que
para hacer más cómodo el proceso de edición y compilación, \texis\
proporciona ficheros que facilitan el proceso de generación del
fichero PDF final, tal y como se describe en el
capítulo~\ref{cap:makefile}.  Esos ficheros adicionales sólo funcionan
correctamente si son ejecutados en Linux.

%-------------------------------------------------------------------
\section{Qué no es}
%-------------------------------------------------------------------
\label{cap1:sec:que-no-es}

Esta plantilla \emph{no} es un manual de \LaTeX, ni una guía de
referencia, ni un compendio de preguntas frecuentes. De hecho, no nos
consideramos expertos en \LaTeX, por lo que no tendríamos fuerzas para
escribir algo así. Si necesitas un manual de \LaTeX, puedes encontrar
muchos y muy buenos en Internet. Al final de este capítulo aparece una
lista con algunos de ellos.

La plantilla tampoco es \emph{una clase} de \LaTeX. Si miras el código
fuente podrás comprobar que el documento comienza con
\verb+\documentclass{book}+\footnote{Personalizado, eso sí, para que
  utilice DIN A-4, a doble cara y con letra de 11 puntos.}, por lo que
se basa en la clase \texttt{book}.

La plantilla tampoco te ayudará a gestionar tu bibliografía. Los
\texttt{.bib} los tendrás que crear y organizar tú ya sea de forma
manual o con alguna herramienta diseñada para ello.

\medskip

Queremos una vez más insistir antes de terminar que no somos expertos
en \LaTeX.  Durante el proceso de escritura de nuestras Tesis nos
tuvimos que enfrentar a problemas de formato que tuvimos que
solucionar buscando en Internet o preguntando a personas cercanas. Y
podemos decir que prácticamente todos los problemas a los que nos
hemos enfrentado en nuestra vida como usuarios de \LaTeX\ están
resueltos aquí, pues sendas Tesis han sido los documentos más extensos
que hemos escrito.

Por lo tanto, si tienes alguna duda concreta de \LaTeX, en vez de
preguntarnos a nosotros, busca en foros de Internet o en la
documentación del paquete que estás utilizando. A buen seguro
encontrarás ahí la respuesta. Si la duda que tienes es relativa a la
plantilla, revisa los comentarios que encontrarás en el código fuente,
hay ciertas cosas de demasiado bajo nivel que hemos preferido no
contar en el texto. Y sólo como último recurso, preguntanos a
nosotros, aunque ya te advertimos que puede que no sepamos
responderte. Querríamos poder animarte a escribirnos tus dudas, pero
preferimos no hacerlo para no decepcionarte.


%-------------------------------------------------------------------
\section{Estructura de capítulos}
%-------------------------------------------------------------------
\label{cap1:sec:estructura}

El manual está estructurado en los siguientes capítulos:

\begin{itemize}
\item El capítulo~\ref{cap2} describe a vista de pájaro los distintos
  ficheros que forman \texis. Además da una primera aproximación
  a cómo generar el documento final (\texttt{.pdf}).

\item El capítulo~\ref{cap3} se centra en el proceso de
  edición. Aunque aparentemente la tarea de escribir el texto es
  trivial, \texis\ proporciona una serie de comandos que pueden
  ser útiles durante la escritura (al menos a nosotros nos lo
  parecieron). Este capítulo se centra en la explicación de esos
  comandos.

\item El capítulo~\ref{cap4} pasa a describir cómo se estructuran las
  imágenes en \texis. Igual que antes, esto puede parecer
  superfluo a un usuario medio de \LaTeX, pero \texis\ contiene
  algunos comandos que esperan esa estructura. Es el usuario el último
  que decide si utiliza esos comandos (y por lo tanto esa estructura)
  u opta por otra completamente distinta.

\item El capítulo~\ref{cap5} aborda la bibliografía y la gesión de los
  acrónimos. Como se verá, \texis\ dispone de algunas opciones de
  personalización que merecen un pequeño capítulo.

\item El capítulo~\ref{cap6} pone fin al manual, detallando las
  opciones del fichero \texttt{Makefile} que permiten una generación
  cómoda del documento final en entornos Linux.
\end{itemize}

El manual tiene, por último, un apéndice que, si bien no es
interesante desde el punto de vista del usuario, nos sirve de excusa
para proporcionar el código \LaTeX\ necesario para su creación: a modo
de ``así se hizo'', comenta brevemente cómo fue el proceso de
escritura de nuestras tesis.


%-------------------------------------------------------------------
\section*{\NotasBibliograficas}
%-------------------------------------------------------------------
\TocNotasBibliograficas

El ``libro'' por el que la mayoría de la gente empieza sus andaduras
con \LaTeX\ es \cite{ldesc2e} pues es relativamente corto, fácil de
leer y de acceso público (licencia GPL), por lo que se puede
conseguir la versión electrónica fácilmente. Un libro algo más
completo que éste y que suele ser el segundo en orden de preferencia
es \cite{notsoshort} con la misma licencia. Dentro de los libros
dedicados a \LaTeX\ de libre distribución, también se puede contar con
\cite{latexAPrimer}.

No obstante, los libros de \LaTeX\ más conocidos son ``The \LaTeX\
Companion'' \citep{latexCompanion} y ``\LaTeX: A Document Preparation
System'' \citep{LaTeXLamport}.

%-------------------------------------------------------------------
\section*{\ProximoCapitulo}
%-------------------------------------------------------------------
\TocProximoCapitulo

Una vez hecha una descripción de \texis, el próximo capítulo
describe los ficheros que componen tanto la plantilla como el manual
que estás leyendo. También se explicará cómo se puede generar o
compilar el manual a partir de los \texttt{.tex} proporcionados. Por
lo tanto, el capítulo sirve como una primera aproximación rápida al
trabajo con \texis; al final del mismo seremos capaces de entender
la estructura de directorios propuesta y dónde se encuentran los
ficheros que hay que editar para cambiar el contenido del documento
final.

No obstante, el capítulo siguiente debe verse únicamente como una
primera aproximación. El capítulo~\ref{cap:edicion} da más detalles
sobre el proceso de edición del documento, y el
capítulo~\ref{cap:makefile} dará una alternativa al modo de
compilación explicado.

% Variable local para emacs, para  que encuentre el fichero maestro de
% compilación y funcionen mejor algunas teclas rápidas de AucTeX
%%%
%%% Local Variables:
%%% mode: latex
%%% TeX-master: "../ManualTeXiS.tex"
%%% End:

\include{Capitulos/01EstadoDelArte}
\include{Capitulos/02RecomendadorBayesiano}
%---------------------------------------------------------------------
%
%                          Capítulo 3
%
%---------------------------------------------------------------------
%
% 03RecomendadorKVecinosCercanos.tex
% Copyright 2009 Marco Antonio Gomez-Martin, Pedro Pablo Gomez-Martin
%
% This file belongs to the TeXiS manual, a LaTeX template for writting
% Thesis and other documents. The complete last TeXiS package can
% be obtained from http://gaia.fdi.ucm.es/projects/texis/
%
% Although the TeXiS template itself is distributed under the 
% conditions of the LaTeX Project Public License
% (http://www.latex-project.org/lppl.txt), the manual content
% uses the CC-BY-SA license that stays that you are free:
%
%    - to share & to copy, distribute and transmit the work
%    - to remix and to adapt the work
%
% under the following conditions:
%
%    - Attribution: you must attribute the work in the manner
%      specified by the author or licensor (but not in any way that
%      suggests that they endorse you or your use of the work).
%    - Share Alike: if you alter, transform, or build upon this
%      work, you may distribute the resulting work only under the
%      same, similar or a compatible license.
%
% The complete license is available in
% http://creativecommons.org/licenses/by-sa/3.0/legalcode
%
%---------------------------------------------------------------------

\chapter{Recomendador por K-Vecinos más similares}
\label{cap3}

\begin{FraseCelebre}
\begin{Frase}
  La inteligencia artificial será la última versión de Google, el motor de búsqueda que entenderá todo en la web. Comprenderá exactamente lo que quiera el usuario y le dará lo correcto. No estamos cerca de lograrlo ahora pero podemos acercarnos cada vez más y es básicamente en lo que trabajamos.
\end{Frase}
\begin{Fuente}
Larry Page, fundador de Google
\end{Fuente}
\end{FraseCelebre}

\begin{resumen}
  Este capítulo tratará en profundidad el recomendador basado en los K vecinos más cercanos. Se comentarán las diferentes implementaciones y formas de funcionamientos que tiene este concepto adaptado a los jueces en línea, su evaluación y su proceso de optimización. Las comparativas y evaluaciones en conjunto con el otro recomendador se tratarán en otro capítulo más adelante.~\ref{cap:comparaciones}.
\end{resumen}

%-------------------------------------------------------------------
\section{Introducción recomendador k-vecinos}
%-------------------------------------------------------------------
\label{cap3:sec:introduccion}

El recomendador de vecinos, o recomendador basado en correlaciones de usuarios, es un recomendador cuyo algoritmo recomienda calculando la correlación entre usuarios y posteriormente se le da un peso a cada problema que aún no ha hecho el usuario A, respecto al resto de usuarios Bi teniendo en cuenta ese grado de correlación de cada Bi. La finalidad y motivación para realizar este recomendador ha sido la necesidad de poder implementar un “algoritmo” más común en el mundo de los recomendadores, pero adaptado al caso de los \texttt{jueces en línea} y optimizado para este, teniendo en cuenta que no tenemos información más que usuarios y problemas anonimizados sin “propiedades” solo \texttt{relaciones} entre ellos considerando que un usuario ha “intentado y resuelto/intentado y no resuelto/no intentado”  un problema. De esta manera podemos tratar esto como un grafo donde los nodos son los usuarios y las relaciones son los problemas en común entre ellos. Con este recomendador también vamos a poder contrastar los resultados con el otro recomendador que se va a realizar (Para tener más diversidad en la recomendación y poder tener una api sobre la que decidir qué recomendador usar según qué momentos o qué situaciones, etc). Estos conceptos e ideas de trabajarán con más detalle en los capítulos a posterior donde se habla como funciona la api y la comunicación del recomendador con un juez en línea. ~\ref{cap:implementacion}.

Para este recomendador se han estudiado también sus diferentes situaciones límite, por ejemplo para casos donde tenemos usuarios que aún no han realizado problemas (El recomendador devuelve una lista de todos los problemas con un peso de 0 para todos, y por lo tando, un orden “aleatorio” que no sirve para recomendar), o casos donde le cueste recomendar un problema nuevo que se ha añadido por falta de AC’s/Entregas que tenga (Como otros recomendadores, si falta información no podrá realizar recomendaciones buenas hasta tener suficiente información para ello), indirectamente el algoritmo se ve afectado por estos parámetros y por ello discrimina en ciertos casos problemas que en un futuro podría no discriminar… Para ello se pueden contemplar modificaciones en el algoritmo que tengan en cuenta estos casos en que la base del algoritmo no llega a lograr cubrir y discrimine en ciertos casos problemas que quizás otros recomendadores no discriminen en estas mismas situaciones (Sean más eficaces para situaciones límites). De todas formas se puede implementar el recomendador, teniendo en cuenta las situaciones límites para hacer recomendaciones sobre usuarios que vaya a tener recomendación eficaz, que llevándolo a casos prácticos reales, será la gran mayoría de las veces. Por lo general este algoritmo de recomendación recomendará el problema más fácil que un usuario podría hacer (O un orden a seguir del que menos le pueda costar, frente al que más le pueda costar). Esto generará que muchos problemas tengan más peso respecto a otros, aunque si falta información (como podremos observar más adelante), muchos problemas tendrán los mismos pesos y podrán “dificultar” la precisión de la recomendación, esto se dará para un número de K-vecinos muy bajo (K <= 10) como veremos en la evaluación, y se podrá hacer un estudio y cálculo en base a estas recomendaciones, obteniendo un listado ordenado de los problemas más fáciles a los más difíciles. Se podría considerar que también sean recomendaciones por gustos, ya que nos fijamos en los usuarios similares y les damos un peso a los problemas que nos faltan, pero más adelante en los resultados analizaremos mejor cómo ha actuado el algoritmo y cómo, por lo tanto, recomienda este recomendador. Hay que tener en cuenta, que una vez implementado el recomendador, quizás perdemos fiabilidad a medida que pasa el tiempo, con las nuevas recomendaciones, ya que si los usuarios siguen a rajatabla los ejercicios recomendados, se crearán patrones de recomendación para los nuevos usuarios o aquellos que tengan pocos problemas resueltos.

%-------------------------------------------------------------------
\section{Funcionamiento recomendador k-vecinos}
%-------------------------------------------------------------------
\label{cap3:sec:funcionamiento}

Este recomendador tiene similitudes en algunos aspectos con los Nearest Neighborhood ya hablados en el capítulo de estado del arte /*TODO INSERTAR REFERENCIA*/, con la diferencia de que una vez obtenemos esos vecinos más cercanos, en vez de clasificar en clases al usuario y generar una recomendación en base a la clase donde se ha clasificado, usamos esas correlaciones o similitudes entre usuarios calculadas para asignarle pesos a problemas a modo de sumatorio.







%-------------------------------------------------------------------
\section*{\NotasBibliograficas}
%-------------------------------------------------------------------
\TocNotasBibliograficas

En este capítulo hemos descrito simplemente la estructura de
directorio de \texis, por lo que no existe ninguna fuente
relacionada adicional de consulta. Se mantiene este apartado por
simetría con el resto de capítulos. En un documento normal (tesis,
trabajo de investigación) lo más probable es que todos los capítulos
puedan extenderse con notas de este tipo.

%-------------------------------------------------------------------
\section*{\ProximoCapitulo}
%-------------------------------------------------------------------
\TocProximoCapitulo

Una vez que se han descrito a vista de pájaro los ficheros que
componen la plantilla y una primera aproximación al proceso de
generación del documento final (en PDF), el siguiente capítulo pasa a
describir el proceso de edición.

Eso cubre aspectos tales como los ficheros que deben modificarse para
añadir nuevos capítulos o los comandos que \texis\ hace
disponibles para escribir ciertas partes de los mismos. El capítulo
describe también los dos modos de generación del documento final que
pueden ser de utilidad durante el largo proceso de escritura. Por
último, el capítulo terminará con ciertas consideraciones relativas a
los editores de \LaTeX\ utilizados así como sobre la posibilidad de
utilizar un control de versiones.

% Variable local para emacs, para  que encuentre el fichero maestro de
% compilación y funcionen mejor algunas teclas rápidas de AucTeX
%%%
%%% Local Variables:
%%% mode: latex
%%% TeX-master: "../ManualTeXiS.tex"
%%% End:

%%---------------------------------------------------------------------
%
%                          Parte 1
%
%---------------------------------------------------------------------
%
% Parte1.tex
% Copyright 2009 Marco Antonio Gomez-Martin, Pedro Pablo Gomez-Martin
%
% This file belongs to the TeXiS manual, a LaTeX template for writting
% Thesis and other documents. The complete last TeXiS package can
% be obtained from http://gaia.fdi.ucm.es/projects/texis/
%
% Although the TeXiS template itself is distributed under the 
% conditions of the LaTeX Project Public License
% (http://www.latex-project.org/lppl.txt), the manual content
% uses the CC-BY-SA license that stays that you are free:
%
%    - to share & to copy, distribute and transmit the work
%    - to remix and to adapt the work
%
% under the following conditions:
%
%    - Attribution: you must attribute the work in the manner
%      specified by the author or licensor (but not in any way that
%      suggests that they endorse you or your use of the work).
%    - Share Alike: if you alter, transform, or build upon this
%      work, you may distribute the resulting work only under the
%      same, similar or a compatible license.
%
% The complete license is available in
% http://creativecommons.org/licenses/by-sa/3.0/legalcode
%
%---------------------------------------------------------------------

% Definici�n de la primera parte del manual

\partTitle{Conceptos b�sicos}

\partDesc{Esta primera parte del manual presenta los conceptos b�sicos
  de \texis. Contiene un cap�tulo de introducci�n, seguido de una
  descripci�n de la estructura de \texis\ y c�mo se genera el
  documento final, para terminar con un cap�tulo en el que se describe
  el proceso de edici�n sugerido y los comandos que \texis\
  proporciona para facilitar dicho proceso.}

\partBackText{En realidad la divisi�n por partes del manual no aporta
  demasiado al lector; se ha dividido en varias partes debido a que,
  en la pr�ctica, el c�digo de este manual sirve como ejemplo de uso
  de \texis.

  En un contexto distinto, es posible que un manual de este tipo no
  habr�a tenido estas partes as� de diferenciadas.}

\makepart

%%---------------------------------------------------------------------
%
%                          Cap�tulo 1
%
%---------------------------------------------------------------------
%
% 01Introduccion.tex
% Copyright 2009 Marco Antonio Gomez-Martin, Pedro Pablo Gomez-Martin
%
% This file belongs to the TeXiS manual, a LaTeX template for writting
% Thesis and other documents. The complete last TeXiS package can
% be obtained from http://gaia.fdi.ucm.es/projects/texis/
%
% Although the TeXiS template itself is distributed under the 
% conditions of the LaTeX Project Public License
% (http://www.latex-project.org/lppl.txt), the manual content
% uses the CC-BY-SA license that stays that you are free:
%
%    - to share & to copy, distribute and transmit the work
%    - to remix and to adapt the work
%
% under the following conditions:
%
%    - Attribution: you must attribute the work in the manner
%      specified by the author or licensor (but not in any way that
%      suggests that they endorse you or your use of the work).
%    - Share Alike: if you alter, transform, or build upon this
%      work, you may distribute the resulting work only under the
%      same, similar or a compatible license.
%
% The complete license is available in
% http://creativecommons.org/licenses/by-sa/3.0/legalcode
%
%---------------------------------------------------------------------

\chapter{Introducci�n}

\begin{FraseCelebre}
\begin{Frase}
P�sose don Quijote delante de dicho carro, y haciendo en su fantas�a
uno de los m�s desvariados discursos que jam�s hab�a hecho, dijo en
alta voz:
\end{Frase}
\begin{Fuente}
  Alonso Fern�ndez de Avellaneda, El Ingenioso Hidalgo Don Quijote de
  la Mancha
\end{Fuente}
\end{FraseCelebre}

\begin{resumen}
  Este cap�tulo presenta una breve introducci�n a \texis.  El
  lector podr� hacerse una idea de qu� es y para qu� sirve. Tambi�n se
  encuentra aqu� una descripci�n del resto de cap�tulos del manual.
\end{resumen}


%-------------------------------------------------------------------
\section{Introducci�n}
%-------------------------------------------------------------------
\label{cap1:sec:introduccion}


Si est�s leyendo estas l�neas es muy posible que haya llegado la hora
de ponerte a escribir la tesis, despu�s de mucho tiempo dando vueltas
al �rea de investigaci�n concreta en el que est�s inmerso. O puede que
est�s a punto de empezar a escribir la memoria del proyecto de fin de
carrera, fin de master, o cualquier otro documento de cierta
envergadura.

Sea lo que sea lo que te traes entre manos, lo m�s probable es que no
sea f�cil hacerlo. Muy posiblemente no tengas a�n muy claro qu� vas a
escribir, pero tu tutor/director/profesor te ha dicho que vayas
empezando a plasmar esas ideas sobre el papel para tener algo firme, y
sentir que vas avanzando.

Y entonces viene el problema de c�mo escribirlo. Muy posiblemente
habr�s escrito alg�n art�culo en \LaTeX\ y est�s convencido de que esa
es la v�a a seguir para hacer un documento que superar� las 10 p�ginas
y que tendr� bibliograf�a. O puede, simplemente, que alguien te haya
dicho que lo mejor es que escribas el proyecto en \LaTeX\ porque la
apariencia final es mejor, porque es m�s c�modo, o cualquier otra
raz�n.

Sea como fuere, parece que est�s m�s o menos decidido a escribir tu
documento en \LaTeX. Bien hecho. Pero, �c�mo?. Al contrario de lo que
suele ocurrir en congresos y en revistas, no tienes disponible ninguna
p�gina en la que descargarte las ``instrucciones para los autores'',
con la c�moda plantilla en \LaTeX\ que t�, sufrido autor, simplemente
tienes que rellenar. No. Ahora las cosas son m�s complicadas.

As� que te vas a la gu�a de \LaTeX\ con la que empezaste (apostamos
que es la misma con la que hemos empezado todos), y ves las distintas
posibilidades que te ofrece en su ``\texttt{documentclass}'':
\texttt{article}, \texttt{report}, \texttt{book}, ... Y te quedas con
la �ltima. Pero te asaltan muchas preguntas. �C�mo organizo todo esto?
o �c�mo hago la portada? o incluso �qu� hago para que no ponga
``Chapter'', sino ``Cap�tulo''?. En ese punto, es de suponer, has
pedido ayuda a la gente de alrededor y/o a tu buscador de Internet
favorito. Y de alguna forma, te has encontrado leyendo estas l�neas.

Tenemos que decir que exactamente esa fue nuestra situaci�n cuando por
fin nos decidimos a escribir nuestras tesis. Desgraciadamente, ni la
gente que ten�amos alrededor ni nuestro buscador favorito supieron
contestarnos de forma satisfactoria, por lo que tuvimos que invertir
\emph{mucho tiempo} hasta conseguir que el resultado que sal�a de
nuestros \texttt{.tex} nos gustara, hasta que nos sentimos c�modos con
la estructura de los ficheros, con las macros disponibles y con el
modo de compilaci�n.

Y para que nadie m�s pueda utilizar como excusa el no saber c�mo
personalizar la clase \texttt{book} para retrasar el comienzo de su
tesis, para que nadie m�s se decida por Word u otro paquete ofim�tico
en vez de \LaTeX\ porque lo ve mucho m�s sencillo, en definitiva, para
que nadie pierda tanto tiempo como perdimos nosotros creando la
estructura, decidimos hacer p�blico el esqueleto b�sico que
construimos nosotros para hacerlas. Ese esqueleto b�sico o plantilla
es \texis.

En vez de hacer disponible la plantilla o ficheros \texttt{.tex} sin
ning�n contenido, proporcionamos un manual en formato PDF que (a no
ser que est�s leyendo directamente el c�digo \LaTeX), ser� lo que
est�s leyendo. Este manual ha sido creado \emph{con la propia
  plantilla}. Por lo tanto, la distribuci�n de \texis\ es en
realidad el c�digo fuente de \emph{su propio manual}. Con su c�digo
fuente entre tus manos, lo �nico que tienes que hacer es borrar su
contenido (\emph{este texto}), y rellenarlo con tu gran contribuci�n
al mundo.  Como podr�s comprobar, la estructura del propio manual
sigue el esquema de lo que podr�a ser una tesis, trabajo de
investigaci�n o proyecto de fin de carrera, precisamente para que sea
f�cil quitar el contenido textual y sustituirlo por el nuevo.

En los cap�tulos que siguen encontrar�s toda la informaci�n necesaria
para poder utilizar los ficheros \LaTeX\ para crear tus propios
documentos. Adem�s, el propio c�digo fuente est� lleno de comentarios
(especialmente en los ficheros que definen el estilo), por lo que
tambi�n en ellos encontrar�s una buena fuente de informaci�n. Eso es
especialmente importante en caso de que quieras modificar en algo el
aspecto final de tu documento.

Esperemos que te sea de utilidad. Si es as�, nos gustar�a que lo
reconocieras en la secci�n de agradecimientos. Si durante tu proceso
de escritura has a�adido alg�n aspecto que crees que puede ser
interesante para otros, no dudes en dec�rnoslo para intentar incluirlo
en siguientes versiones de la propia plantilla; tampoco dudes en
enviarnos sugerencias sobre las explicaciones de este manual para
poder mejorarlo con el tiempo. Por �ltimo, tambi�n puedes enviarnos el
resultado final para poner una referencia a �l en la p�gina de
descarga, donde, por cierto, puedes ver otros documentos creados con
la plantilla, lo que te permitir� coger ideas de cosas que puedes
variar. Recuerda que la versi�n m�s reciente de \texis\ est�
disponible en \url{http://gaia.fdi.ucm.es/projects/texis/}.

%-------------------------------------------------------------------
\section{Qu� es \texis}
%-------------------------------------------------------------------
\label{cap1:sec:que-es}

La plantilla que tienes entre las manos es, como hemos dicho, el
esqueleto del c�digo fuente de las Tesis Doctorales de los dos autores
\citep{GomezMartinMA2008PhD, GomezMartinPP2008PhD}. Por tanto, sirve
para escribir otras Tesis Doctorales u otros documentos con estructura
similar de forma f�cil.

\texis\ te permite adem�s generar el fichero utilizando tanto el
comando \texttt{latex} (que genera de forma nativa ficheros
\texttt{dvi} que luego se convierten a ficheros \texttt{ps} o
\texttt{pdf}), como \texttt{pdflatex}. De esta forma el usuario final
puede elegir entre cualquiera de las dos herramientas\footnote{Esto es
  �til por ejemplo cuando quieres utilizar \texttt{pdflatex} pero
  finalmente el servicio de publicaciones s�lo admite el uso de
  \texttt{latex}.}.  Aconsejamos, no obstante, la utilizaci�n de este
�ltimo, debido a que \texis\ contiene ciertos comandos para dotar al
PDF final de marcadores que permiten una navegaci�n c�moda por el
fichero utilizando los visores tradicionales.

\medskip

Como explicaremos en el cap�tulo siguiente, la plantilla se aprovecha
mejor en sistemas GNU/Linux. Nota que hemos dicho que la plantilla
``\emph{se aprovecha mejor}'' en sistemas GNU/Linux, no que \emph{no
pueda utilizarse} en Windows o Mac; es evidente que \LaTeX\ es
multiplataforma, y por lo tanto puede compilarse en cualquier sistema
que tenga instalada una distribuci�n del mismo.

La raz�n por esta ``desviaci�n positiva'' hacia Linux estriba en que
para hacer m�s c�modo el proceso de edici�n y compilaci�n, \texis\
proporciona ficheros que facilitan el proceso de generaci�n del
fichero PDF final, tal y como se describe en el
cap�tulo~\ref{cap:makefile}.  Esos ficheros adicionales s�lo funcionan
correctamente si son ejecutados en Linux.

%-------------------------------------------------------------------
\section{Qu� no es}
%-------------------------------------------------------------------
\label{cap1:sec:que-no-es}

Esta plantilla \emph{no} es un manual de \LaTeX, ni una gu�a de
referencia, ni un compendio de preguntas frecuentes. De hecho, no nos
consideramos expertos en \LaTeX, por lo que no tendr�amos fuerzas para
escribir algo as�. Si necesitas un manual de \LaTeX, puedes encontrar
muchos y muy buenos en Internet. Al final de este cap�tulo aparece una
lista con algunos de ellos.

La plantilla tampoco es \emph{una clase} de \LaTeX. Si miras el c�digo
fuente podr�s comprobar que el documento comienza con
\verb+\documentclass{book}+\footnote{Personalizado, eso s�, para que
  utilice DIN A-4, a doble cara y con letra de 11 puntos.}, por lo que
se basa en la clase \texttt{book}.

La plantilla tampoco te ayudar� a gestionar tu bibliograf�a. Los
\texttt{.bib} los tendr�s que crear y organizar t� ya sea de forma
manual o con alguna herramienta dise�ada para ello.

\medskip

Queremos una vez m�s insistir antes de terminar que no somos expertos
en \LaTeX.  Durante el proceso de escritura de nuestras Tesis nos
tuvimos que enfrentar a problemas de formato que tuvimos que
solucionar buscando en Internet o preguntando a personas cercanas. Y
podemos decir que pr�cticamente todos los problemas a los que nos
hemos enfrentado en nuestra vida como usuarios de \LaTeX\ est�n
resueltos aqu�, pues sendas Tesis han sido los documentos m�s extensos
que hemos escrito.

Por lo tanto, si tienes alguna duda concreta de \LaTeX, en vez de
preguntarnos a nosotros, busca en foros de Internet o en la
documentaci�n del paquete que est�s utilizando. A buen seguro
encontrar�s ah� la respuesta. Si la duda que tienes es relativa a la
plantilla, revisa los comentarios que encontrar�s en el c�digo fuente,
hay ciertas cosas de demasiado bajo nivel que hemos preferido no
contar en el texto. Y s�lo como �ltimo recurso, preguntanos a
nosotros, aunque ya te advertimos que puede que no sepamos
responderte. Querr�amos poder animarte a escribirnos tus dudas, pero
preferimos no hacerlo para no decepcionarte.


%-------------------------------------------------------------------
\section{Estructura de cap�tulos}
%-------------------------------------------------------------------
\label{cap1:sec:estructura}

El manual est� estructurado en los siguientes cap�tulos:

\begin{itemize}
\item El cap�tulo~\ref{cap2} describe a vista de p�jaro los distintos
  ficheros que forman \texis. Adem�s da una primera aproximaci�n
  a c�mo generar el documento final (\texttt{.pdf}).

\item El cap�tulo~\ref{cap3} se centra en el proceso de
  edici�n. Aunque aparentemente la tarea de escribir el texto es
  trivial, \texis\ proporciona una serie de comandos que pueden
  ser �tiles durante la escritura (al menos a nosotros nos lo
  parecieron). Este cap�tulo se centra en la explicaci�n de esos
  comandos.

\item El cap�tulo~\ref{cap4} pasa a describir c�mo se estructuran las
  im�genes en \texis. Igual que antes, esto puede parecer
  superfluo a un usuario medio de \LaTeX, pero \texis\ contiene
  algunos comandos que esperan esa estructura. Es el usuario el �ltimo
  que decide si utiliza esos comandos (y por lo tanto esa estructura)
  u opta por otra completamente distinta.

\item El cap�tulo~\ref{cap5} aborda la bibliograf�a y la gesi�n de los
  acr�nimos. Como se ver�, \texis\ dispone de algunas opciones de
  personalizaci�n que merecen un peque�o cap�tulo.

\item El cap�tulo~\ref{cap6} pone fin al manual, detallando las
  opciones del fichero \texttt{Makefile} que permiten una generaci�n
  c�moda del documento final en entornos Linux.
\end{itemize}

El manual tiene, por �ltimo, un ap�ndice que, si bien no es
interesante desde el punto de vista del usuario, nos sirve de excusa
para proporcionar el c�digo \LaTeX\ necesario para su creaci�n: a modo
de ``as� se hizo'', comenta brevemente c�mo fue el proceso de
escritura de nuestras tesis.


%-------------------------------------------------------------------
\section*{\NotasBibliograficas}
%-------------------------------------------------------------------
\TocNotasBibliograficas

El ``libro'' por el que la mayor�a de la gente empieza sus andaduras
con \LaTeX\ es \cite{ldesc2e} pues es relativamente corto, f�cil de
leer y de acceso p�blico (licencia GPL), por lo que se puede
conseguir la versi�n electr�nica f�cilmente. Un libro algo m�s
completo que �ste y que suele ser el segundo en orden de preferencia
es \cite{notsoshort} con la misma licencia. Dentro de los libros
dedicados a \LaTeX\ de libre distribuci�n, tambi�n se puede contar con
\cite{latexAPrimer}.

No obstante, los libros de \LaTeX\ m�s conocidos son ``The \LaTeX\
Companion'' \citep{latexCompanion} y ``\LaTeX: A Document Preparation
System'' \citep{LaTeXLamport}.

%-------------------------------------------------------------------
\section*{\ProximoCapitulo}
%-------------------------------------------------------------------
\TocProximoCapitulo

Una vez hecha una descripci�n de \texis, el pr�ximo cap�tulo
describe los ficheros que componen tanto la plantilla como el manual
que est�s leyendo. Tambi�n se explicar� c�mo se puede generar o
compilar el manual a partir de los \texttt{.tex} proporcionados. Por
lo tanto, el cap�tulo sirve como una primera aproximaci�n r�pida al
trabajo con \texis; al final del mismo seremos capaces de entender
la estructura de directorios propuesta y d�nde se encuentran los
ficheros que hay que editar para cambiar el contenido del documento
final.

No obstante, el cap�tulo siguiente debe verse �nicamente como una
primera aproximaci�n. El cap�tulo~\ref{cap:edicion} da m�s detalles
sobre el proceso de edici�n del documento, y el
cap�tulo~\ref{cap:makefile} dar� una alternativa al modo de
compilaci�n explicado.

% Variable local para emacs, para  que encuentre el fichero maestro de
% compilaci�n y funcionen mejor algunas teclas r�pidas de AucTeX
%%%
%%% Local Variables:
%%% mode: latex
%%% TeX-master: "../ManualTeXiS.tex"
%%% End:

%%---------------------------------------------------------------------
%
%                          Cap�tulo 2
%
%---------------------------------------------------------------------
%
% 02EstructuraYGeneracion.tex
% Copyright 2009 Marco Antonio Gomez-Martin, Pedro Pablo Gomez-Martin
%
% This file belongs to the TeXiS manual, a LaTeX template for writting
% Thesis and other documents. The complete last TeXiS package can
% be obtained from http://gaia.fdi.ucm.es/projects/texis/
%
% Although the TeXiS template itself is distributed under the 
% conditions of the LaTeX Project Public License
% (http://www.latex-project.org/lppl.txt), the manual content
% uses the CC-BY-SA license that stays that you are free:
%
%    - to share & to copy, distribute and transmit the work
%    - to remix and to adapt the work
%
% under the following conditions:
%
%    - Attribution: you must attribute the work in the manner
%      specified by the author or licensor (but not in any way that
%      suggests that they endorse you or your use of the work).
%    - Share Alike: if you alter, transform, or build upon this
%      work, you may distribute the resulting work only under the
%      same, similar or a compatible license.
%
% The complete license is available in
% http://creativecommons.org/licenses/by-sa/3.0/legalcode
%
%---------------------------------------------------------------------

\chapter{Estructura  y generaci�n}
\label{cap2}

\begin{FraseCelebre}
\begin{Frase}
  La mejor estructura no garantizar� los resultados ni el rendimiento.
  Pero la estructura equivocada es una garant�a de fracaso.
\end{Frase}
\begin{Fuente}
Peter Drucker
\end{Fuente}
\end{FraseCelebre}

\begin{resumen}
  Este cap�tulo explica la estructura de directorios de \texis\
  as� como los ficheros m�s importantes, describiendo el cometido de
  cada uno. Tambi�n hace una primera aproximaci�n al proceso de
  generaci�n (o compilaci�n) del PDF final, aunque este tema ser�
  extendido posteriormente en el
  cap�tulo~\ref{cap:makefile}.
\end{resumen}

%-------------------------------------------------------------------
\section{Estructura de directorios}
%-------------------------------------------------------------------
\label{cap2:sec:estructura}

Como habr�s podido comprobar, la plantilla contiene bastantes ficheros
organizados en varios directorios. Esta secci�n explica el contenido
de cada uno de los directorios, para que seas capaz de encontrar el
directorio en el que deber�a estar un fichero concreto.

Existen los siguientes directorios:

\begin{description}
\item[Directorio ra�z] contiene el fichero principal del documento
  (tambi�n llamado fichero \emph{maestro}), que es el que se utiliza
  como entrada a \texttt{pdflatex} (o \texttt{latex}) y cuyo nombre es
  \texttt{Tesis.tex}. Tambi�n aparecen en el directorio otros ficheros
  que si bien no generan texto en el documento final cumplen ciertas
  funciones espec�ficas descritas en la
  secci�n~\ref{cap2:sec:directorio-raiz}. Por �ltimo, el directorio
  contiene tambi�n los ficheros \texttt{.bib} con la informaci�n
  bibliogr�fica as� como el fichero para generar el documento
  utilizando la aplicaci�n \verb+make+.

\item[Directorio \texttt{./Capitulos}] contiene los \texttt{.tex} de
  cada cap�tulo del documento.

\item[Directorio \texttt{./Apendices}] contiene los \texttt{.tex} de
  cada uno de los ap�ndices.

\item[Directorio \texttt{./Cascaras}] contiene los \texttt{.tex}
  responsables del contenido del resto de p�ginas del documento: el
  texto de la portada, agradecimientos, resumen, etc. En definitiva
  son los ficheros responsables de todo aquello que precede a los
  cap�tulos y sigue a los ap�ndices.

\item[Directorio \texttt{./Imagenes}] contiene las im�genes del
  documento. Dentro de �l aparecen varios directorios distintos. La
  gesti�n de im�genes (y por lo tanto la estructura de estos
  directorios) se describir� en el cap�tulo~\ref{cap:imagenes}.

\item[Directorio \texttt{./TeXiS}] contiene todos los ficheros
  relacionados con la propia plantilla, es decir, los ficheros que
  definen la apariencia final del documento, as� como los comandos que
  facilitan la edici�n que ser�n descritos en el
  cap�tulo~\ref{cap:edicion}. La creaci�n de un documento que se
  adhiere completamente al formato de \texis\ no necesitar� tocar
  ninguno de los ficheros de este directorio.

\item[Directorio \texttt{./VersionesPrevias}] Este directorio es
  usado por el \texttt{Makefile} cuando se realiza una copia de
  seguridad del estado del documento. Describiremos esta
  caracter�stica en el cap�tulo~\ref{cap:makefile}.
\end{description}

Existen por lo tanto, tres tipos de ficheros \texttt{.tex}: los
ficheros que contienen el texto principal del documento (cap�tulos y
ap�ndices), los ficheros que definen las partes adicionales del mismo
(como portada y agradecimientos), y los ficheros que determinan la
apariencia. En las tres secciones siguientes describimos cada uno de
ellos.

%-------------------------------------------------------------------
\section{Ficheros con el texto principal del documento}
%-------------------------------------------------------------------
\label{cap2:sec:ficheros-texto}

Estos \texttt{.tex} son los que contienen el texto tanto de los
cap�tulos como de los ap�ndices, por lo tanto son los ficheros que m�s
tiempo pasar�s editando. Est�n divididos en secciones, tienen figuras,
tablas, referencias bibliogr�ficas, y cualquier otro tipo de elemento
que quieras o debas a�adir.

En principio pueden contener cualquier c�digo \LaTeX. No obstante, no
olvides que si necesitas alg�n paquete especial que no se cargue por
defecto en la plantilla, deber�s incluir el \verb+\usepackage+
correspondiente en el documento maestro o en el fichero de pre�mbulo
de \texis, \url{TeXiS/TeXiS\_pream.tex} descrito en la
Secci�n~\ref{cap2:sec:ficheros-formato}.

El cap�tulo siguiente est� enteramente dedicado al proceso de edici�n
de estos ficheros.

%-------------------------------------------------------------------
\section[Ficheros del documento auxiliares]%
{Ficheros del documento auxiliares: las c�scaras del documento}
%-------------------------------------------------------------------
\label{cap2:sec:ficheros-auxiliares}

Estos ficheros, como ya hemos dicho, son los responsables del
contenido del resto de p�ginas del documento, todo aquello que no son
cap�tulos o ap�ndices. Son los siguientes (por orden de ``aparici�n''
en el documento final)\footnote{Si crees que no necesitas alguno de
  ellos, puedes eliminar su inclusi�n en el fichero maestro,
  \texttt{Tesis.tex}.}:

\begin{itemize}
\item \texttt{cover.tex}: responsable de las dos primeras hojas del
  documento, que forman las portada. Mediante comandos se definen el
  autor y t�tulo que aparecer� en la portada, la fecha de publicaci�n,
  facultad, etc. Como podr�s ver cuando lo edites, el fichero contiene
  los datos concretos para generar este manual. Los comandos se
  describen en la secci�n~\ref{cap3:sec:tituloYmetadatas}.

\item \texttt{dedicatoria.tex}: contiene el c�digo \LaTeX\ que crea la
  ``dedicatoria'' de la Tesis. Consiste en una hoja donde aparece
  alineada a la izquierda una frase indicando a qui�n se ``dedica'' el
  documento (en los libros serios pone algo como ``A mis padres'',
  aunque tambi�n hay autores en libros m�s distendidos, como
  \citeauthor{AIbyExample} que dice textualmente ``For Mum and Dad,
  who bought me my first computer, and therefore must share some
  responisibility for turning me into the geek that I am''
  \citep{AIbyExample}). Se pueden poner todas las p�ginas de
  dedicatorias que se deseen, utilizando la macro
  \verb|\putDedicatoria|, que recibe la cita completa y crea la hoja
  completa con la misma. Lo m�s c�modo, no obstante, es utilizar la
  macro \verb|\dedicatoriaUno| y (opcionalmente)
  \verb|\dedicatoriaDos| para establecer las dos dedicatorias y a
  continuaci�n invocar \verb|\makeDedicatorias| para generarlas. As�
  lo hace este manual.

\item \texttt{agradecimientos.tex}: contiene el texto de las �nicas
  p�ginas que tu familia y amigos van a leer de la Tesis: los
  agradecimientos. As� que piensa bien lo que pones, no olvides a
  nadie\footnote{Tampoco a nosotros por quitarte la preocupaci�n del
    aspecto final... \texttt{:-)}}.

  Es importante que no borres la l�nea que aparece justo despu�s del
  \verb|\chapter|,

\begin{verbatim}
\cabeceraEspecial{Agradecimientos}
\end{verbatim}

  ya que lo que hace es modificar la cabecera de la p�gina para que no
  aparezca con el mismo formato que en los cap�tulos. Puedes consultar
  la secci�n~\ref{cap3:ssec:capitulos-especiales} para obtener m�s
  detalles sobre esto.

\item \texttt{resumen.tex}: si quieres incluir antes del �ndice un
  peque�o resumen de tu trabajo, puedes hacerlo en este fichero. Al
  igual que en los agradecimientos no debes eliminar el comando
  \LaTeX\ del principio que altera la cabecera.

  Tanto el resumen como los agradecimientos antes explicados se
  convierten en dos ``cap�tulos sin numeraci�n'' que tambi�n ser�n
  listados en el �ndice de contenidos. No obstante, al aparecer antes
  que el texto principal del documento (los cap�tulos propiamente
  dichos), sus p�ginas ser�n numeradas con notaci�n romana, en
  lugar de con la ar�biga tradicional.

\item \texttt{bibliografia.tex}: en �l se configura la
  bibliograf�a del documento. En concreto, el fichero permite indicar
  tanto qu� ficheros \texttt{.bib} contienen las entradas
  bibliogr�ficas como una frase c�lebre (seguramente, ya habr�s notado
  que \texis\ permite iniciar los cap�tulos con una frase
  c�lebre), caracter�stica descrita con m�s detalle en la secci�n
  \ref{cap3:ssec:frases}.

  El cap�tulo~\ref{cap:bibliografia} hace una descripci�n m�s
  detallada del tipo de bibliograf�a que propone utilizar la plantilla
  (y que utiliza este manual).

\item \texttt{fin.tex}: En nuestras respectivas tesis, como ``cierre''
  incluimos una �ltima p�gina parecida a la dedicatoria con un par de
  frases c�lebres. El c�digo \TeX\ responsable se encuentra en este
  fichero.
\end{itemize}

Existen otros dos ficheros que no aparecen en este directorio pero que
generan p�ginas en el documento final. Son \texttt{TeXiS\_toc.tex} y
\texttt{TeXiS\_acron.tex} del directorio \texttt{TeXiS}, descritos en
la secci�n~\ref{cap2:sec:ficheros-formato}. Aparecen en ese directorio
debido a que no permiten ning�n tipo de personalizaci�n al usuario de
\texis.

%-------------------------------------------------------------------
\section{Directorio raiz}
%-------------------------------------------------------------------
\label{cap2:sec:directorio-raiz}

En el directorio ra�z aparecen, adem�s de \texttt{Tesis.tex}, el
documento maestro, otros tres ficheros \texttt{.tex} que no son
responsables de la generaci�n de ninguna p�gina del documento. Uno de
ellos, \texttt{config.tex} se describe en la
secci�n~\ref{cap3:sec:modos-compilacion}. Los otros dos son:

\begin{itemize}
\item \texttt{guionado.tex}: contiene una lista de aquellas palabras
  que, durante la edici�n del documento, se ha podido comprobar que
  \LaTeX\ divid�a mal. En esos casos, la alternativa mala es hacer
  peque�os ajustes en el p�rrafo para que esa palabra cuyos guiones
  \LaTeX\ no sabe colocar no quede cerca del final de la l�nea. La
  alternativa buena es a�adir la palabra a este fichero, colocando los
  guiones donde van. En el fichero proporcionado aparece una lista de
  algunas palabras de ejemplo.

\item \texttt{constantes.tex}: est� pensado para la definici�n de
  constantes que aparezcan a menudo en el texto. Por ejemplo, si se
  hace un documento sobre Cruise Control~\citep{CruiseControl}, para
  evitar tener que escribir cont�nuamente las dos palabras, es buena
  idea incluir una constante en el fichero que cree un comando para
  hacerlo m�s r�pidamente:

\begin{example}
\newcommand{\cc}{Cruise Control}
La nueva versi�n de \cc\ \ldots
\end{example}

En este fichero aparece definida la constante \verb|\titulo| que
contiene el t�tulo del documento y \verb|\autor| con el autor. Ambos
son utilizados en la portada. Tambi�n aparece definido el comando
\verb|\texis| que utilizamos en este manual para evitarnos escribir el
c�digo que escribe ``\texis''\ una y otra vez:

\begin{example}
\texis\ te permite generar el
fichero final tanto como .dvi
como en un .pdf.
\end{example}
\end{itemize}

Por �ltimo indicar que en el directorio ra�z aparecen los ficheros con
extensi�n \texttt{.bib} que contienen la informaci�n bibliogr�fica y
los \texttt{.gdf} para los acr�nimos (ver
cap�tulo~\ref{cap:bibliografia}) as� como el fichero \texttt{Makefile}
para la generaci�n autom�tica del documento final
(cap�tulo~\ref{cap:makefile}).

%-------------------------------------------------------------------
\section{Ficheros de la plantilla}
%-------------------------------------------------------------------
\label{cap2:sec:ficheros-formato}

El directorio \texttt{TeXiS} contiene los ficheros que definen la
apariencia final del documento. Si el formato de este manual te gusta
tal cual, no tendr�s por qu� tocar ninguno de estos ficheros. La
explicaci�n de su contenido aparece a continuaci�n. Su c�digo fuente
contiene numerosos comentarios y enlaces, por lo que no deber�a
suponerte demasiado problema modificarlos.

\begin{itemize}
\item \texttt{TeXiS\_cab.tex}: contiene la definici�n de la apariencia
  de las cabeceras de las p�ginas utilizadas en el documento. La
  plantilla utiliza el paquete \texttt{fancyhdr}. Sin embargo, la
  cabecera por defecto se ha modificado para que aparezca el n�mero
  del cap�tulo, as� como su nombre en min�sculas, junto con alg�n otro
  cambio menor.

\item \texttt{TeXiS.sty}: contiene los comandos que la plantilla
  proporciona para facilitar el proceso de edici�n. El uso de estos
  comandos est� explicado en el cap�tulo~\ref{cap:edicion}. A pesar de
  que la extensi�n distinta a la habitual (\texttt{.sty} en vez de
  \texttt{.tex}) puede imponer cierto respeto al principio, puedes
  abrir sin miedo el fichero para edici�n, pues es un fichero de
  \LaTeX\ normal, con definiciones de comandos tradicionales.

\item \texttt{TeXiS.bst}: contiene el estilo que utiliza la plantilla
  para generar la lista de las referencias bibliogr�ficas al final del
  documento. Las posibilidades de este estilo son descritas en el
  cap�tulo~\ref{cap:bibliografia}.

\item \texttt{TeXiS\_pream.tex}: este fichero contiene la mayor parte
  del c�digo del pre�mbulo del documento (lo que va antes del
  \verb|\begin{document}|). En �l aparecen incluidos un buen n�mero de
    paquetes que pueden ser �tiles en la elaboraci�n del documento,
    junto con una explicaci�n de para qu� sirven y, en algunas
    ocasiones, algunos ejemplos de uso. Existen incluso ciertos
    paquetes cuya inclusi�n aparece comentada pero que se mantienen,
    junto con su comentario correspondiente, por si pueden venir bien
    para documentos concretos que necesiten ciertas caracter�sticas
    que ni este manual ni nuestras tesis requirieron.

  \item \texttt{TeXiS\_cover.tex}: contiene el c�digo \TeX\ que genera
    la portada, y la hoja siguiente a la misma, que vuelve a tener los
    mismos datos pero sin el escudo.

  \item \texttt{TeXiS\_dedic.tex}: contiene el c�dito \TeX\ para
    generar las hojas de dedicatorias.

  \item \texttt{TeXiS\_toc.tex}: es el responsable de la generaci�n de
    los �ndices de cap�tulos, tablas y figuras que aparece en el
    documento.

  \item \texttt{TeXiS\_bib.tex}: es el encargado de que en el
    documento aparezca bibliograf�a. Incluido desde el fichero
    maestro, lo primero que hace es leer el fichero de configuraci�n,
    \texttt{Cascaras/configBibliografia.tex}.

    Como puedes comprobar, la bibliograf�a es tambi�n referenciada en
    el �ndice como un cap�tulo sin numerar; tambi�n se preocupa de
    cambiar la cabecera para que no se utilice la habitual del resto
    de cap�tulos.

\item \texttt{TeXiS\_acron.tex}: la plantilla tambi�n permite a�adir
  una lista de acr�nimos o abreviaturas utilizadas en el texto. En
  este fichero se incluyen los comandos necesarios para que aparezca
  esta lista. No obstante, para que la lista funcione, en el momento
  de la generaci�n se debe invocar a la herramienta correspondiente
  para que se creen los ficheros auxiliares necesarios para su
  generaci�n. En la descripci�n sobre la generaci�n dada en la
  secci�n~\ref{cap2:sec:compilacion} no se describe este proceso,
  por lo que el resultado contendr� una lista de acr�nimos
  vac�a. El uso de acr�nimos se describe con detalle en la
  secci�n~\ref{capBiblio:sec:glosstex}.

\item \texttt{TeXiS\_part.tex}: contiene los comandos relacionados con
  la posibilidad de dividir en \emph{partes} el documento final. Los
  detalles de qu� posibilidades ofrece \texis\ para hacerlo est�n
  descritas en la secci�n~\ref{cap3:ssec:partes}.

\end{itemize}
  
%-------------------------------------------------------------------
\section{Generando el documento}
%-------------------------------------------------------------------
\label{cap2:sec:compilacion}

Como ya se dijo en la introducci�n, \texis\ permite compilar el
documento\footnote{Cuando hablamos de ``compilaci�n'' nos referimos,
  por analog�a con el desarrollo software, a la generaci�n del fichero
  final (un PDF) resultado de analizar los ficheros fuente en \LaTeX.}
tanto con \verb+latex+ como \verb+pdflatex+.  Si has utilizado \LaTeX\
a trav�s de editores de texto espec�ficos (como Kile o WinEdt), es
posible que no sepas de qu� estamos hablando. Tanto \texttt{latex}
como \texttt{pdflatex} son dos aplicaciones que cogen un fichero
\texttt{.tex} como entrada y generan el documento final
``renderizado''. La diferencia entre ambas radica en el fichero de
salida que generan. En el primer caso, se genera un fichero
\texttt{.dvi}\footnote{\emph{Device independent}, o ``independiente
  del dispositivo'' (en el que se mostrar� el contenido).}, mientras
que en el segundo caso se genera un fichero PDF directamente.
Tradicionalmente se ha utilizado \texttt{latex}, convirtiendo despu�s
el fichero \texttt{.dvi} al formato deseado (como \texttt{.ps} o
\texttt{.pdf}). Sin embargo, en nuestro caso, aconsejamos la
utilizaci�n de \texttt{pdflatex}, debido a que, al generar de forma
nativa ficheros PDF, aprovecha algunas de las caracter�sticas
disponibles en los mismos. En particular, \texis\ contiene algunos
comandos \LaTeX\ que \texttt{pdflatex} aprovecha para a�adir
informaci�n de \emph{copyright} al fichero, as� como enlaces a cada
uno de los cap�tulos y secciones del documento, permitiendo una
navegaci�n r�pida por el mismo cuando se utilizan visores
(figura~\ref{cap2:fig:pdf}).

\begin{figure}[t]
  \centering
  %
  \subfloat[][Propiedades del documento]{
     \includegraphics[width=0.42\textwidth]%
                     {Imagenes/Bitmap/02/PropiedadesPDF}
     \label{cap2:fig:PropiedadesPDF}
  }
  \qquad
  \subfloat[][Tabla de contenidos]{
     \includegraphics[width=0.42\textwidth]%
                     {Imagenes/Bitmap/02/IndicePDF}
     \label{cap2:fig:TocPDF}
  }
 \caption{Capturas del visor de PDF\label{cap2:fig:pdf}}
\end{figure}


La plantilla incluye un fichero \texttt{Makefile} para automatizar la
generaci�n del fichero final\footnote{Los ficheros \texttt{Makefile}
  son ampliamente utilizados en el desarrollo de software. Son
  ficheros que sirven de entrada a la utilidad \texttt{make} que
  genera autom�ticamente los ficheros de resultado a partir de los
  archivos de c�digo fuente.} que es capaz de crear el PDF utilizando
cualquiera de las dos alternativas. No obstante, en este apartado no
entraremos en los detalles de este fichero, ya que existe un cap�tulo
dedicado enteramente a �l (cap�tulo~\ref{cap:makefile}).

Para generar el documento de este manual a partir de los ficheros de
\texis\ proporcionados, la forma inmediata es seguir el proceso
tradicional de generaci�n de cualquier fichero de \LaTeX, es decir,
ejecutar \texttt{pdflatex} (o \texttt{latex}), a continuaci�n ejecutar
\texttt{bibtex} para resolver las referencias bibliogr�ficas, y
posteriormente ejecutar un par de veces m�s \texttt{pdflatex} para
resolver las referencias cruzadas y que aparezcan en el documento
final.

En l�nea de comandos eso se traduce a las siguientes
�rdenes\footnote{Tambi�n es v�lido el uso de \texttt{latex} en lugar
  de \texttt{pdflatex}, pero el fichero generado (\texttt{.dvi})
  deber� despu�s ser convertido a PDF.}:

\begin{verbatim}
$ pdflatex Tesis
$ bibtex Tesis
$ pdflatex Tesis
$ pdflatex Tesis
\end{verbatim}

Si se utiliza alg�n editor de \LaTeX\ para la edici�n, tambi�n se
pueden utilizar sus teclas r�pidas (o en su defecto, sus botones u
opciones de men�) para generarlo; encontrar�s una explicaci�n al
respecto en la secci�n~\ref{cap3:sec:editores}.

%-------------------------------------------------------------------
\section*{\NotasBibliograficas}
%-------------------------------------------------------------------
\TocNotasBibliograficas

En este cap�tulo hemos descrito simplemente la estructura de
directorio de \texis, por lo que no existe ninguna fuente
relacionada adicional de consulta. Se mantiene este apartado por
simetr�a con el resto de cap�tulos. En un documento normal (tesis,
trabajo de investigaci�n) lo m�s probable es que todos los cap�tulos
puedan extenderse con notas de este tipo.

%-------------------------------------------------------------------
\section*{\ProximoCapitulo}
%-------------------------------------------------------------------
\TocProximoCapitulo

Una vez que se han descrito a vista de p�jaro los ficheros que
componen la plantilla y una primera aproximaci�n al proceso de
generaci�n del documento final (en PDF), el siguiente cap�tulo pasa a
describir el proceso de edici�n.

Eso cubre aspectos tales como los ficheros que deben modificarse para
a�adir nuevos cap�tulos o los comandos que \texis\ hace
disponibles para escribir ciertas partes de los mismos. El cap�tulo
describe tambi�n los dos modos de generaci�n del documento final que
pueden ser de utilidad durante el largo proceso de escritura. Por
�ltimo, el cap�tulo terminar� con ciertas consideraciones relativas a
los editores de \LaTeX\ utilizados as� como sobre la posibilidad de
utilizar un control de versiones.

% Variable local para emacs, para  que encuentre el fichero maestro de
% compilaci�n y funcionen mejor algunas teclas r�pidas de AucTeX
%%%
%%% Local Variables:
%%% mode: latex
%%% TeX-master: "../ManualTeXiS.tex"
%%% End:

%%---------------------------------------------------------------------
%
%                          Cap�tulo 3
%
%---------------------------------------------------------------------
%
% 03Edicion.tex
% Copyright 2009 Marco Antonio Gomez-Martin, Pedro Pablo Gomez-Martin
%
% This file belongs to the TeXiS manual, a LaTeX template for writting
% Thesis and other documents. The complete last TeXiS package can
% be obtained from http://gaia.fdi.ucm.es/projects/texis/
%
% Although the TeXiS template itself is distributed under the 
% conditions of the LaTeX Project Public License
% (http://www.latex-project.org/lppl.txt), the manual content
% uses the CC-BY-SA license that stays that you are free:
%
%    - to share & to copy, distribute and transmit the work
%    - to remix and to adapt the work
%
% under the following conditions:
%
%    - Attribution: you must attribute the work in the manner
%      specified by the author or licensor (but not in any way that
%      suggests that they endorse you or your use of the work).
%    - Share Alike: if you alter, transform, or build upon this
%      work, you may distribute the resulting work only under the
%      same, similar or a compatible license.
%
% The complete license is available in
% http://creativecommons.org/licenses/by-sa/3.0/legalcode
%
%---------------------------------------------------------------------

\chapter{Proceso de edici�n}
\label{cap3}
\label{cap:edicion}

\begin{FraseCelebre}
\begin{Frase}
%Si quieres ser le�do m�s de una vez, no vaciles en borrar a menudo.
Rem tene, verba sequentur (Si dominas el tema, las palabras vendr�n solas)
\end{Frase}
\begin{Fuente}
%Horacio
Cat�n el Viejo
\end{Fuente}
\end{FraseCelebre}

\begin{resumen}
  Este cap�tulo se centra en el proceso de edici�n, dando detalles de
  qu� cosas deben cambiarse y qu� comandos y caracter�sticas tiene
  \texis\ que facilitan el proceso.
\end{resumen}

%-------------------------------------------------------------------
\section{Empezando a escribir}
%-------------------------------------------------------------------
\label{cap3:sec:tituloYmetadatas}

En primer lugar, es necesario destacar que los ficheros \texttt{.tex}
\emph{deben tener} codificaci�n ISO-8859-1. Esto es lo que ocurre de
manera predefinida en Windows y en algunos Linux como Debian. Una
excepci�n significativa es el caso de Ubuntu, que usa de manera
predeterminada UTF-8. En ese caso, deber�s ser cuidadoso para
asegurarte de que grabas tus ficheros con ISO-8859-1.

\com{En realidad, hay una remota posibilidad de que TeXiS se pueda
  configurar para usar UTF-8 de manera nativa, aunque \emph{nunca lo
    hemos probado} y \emph{no} lo recomendamos.  Hay informaci�n
  adicional en un comentario en \texttt{TeXiS/TeXiS\_pream.tex} (busca
  \texttt{inputenc} para encontrarlo).}

\medskip

El primer paso para la construcci�n de un nuevo documento es cambiar
el t�tulo y autores. Es posible que al principio del proceso no se
tenga muy claro cu�l es el t�tulo final del documento pero, y esto es
una opini�n personal, ver un t�tulo (aunque sea provisional) en vez de
lo que ahora aparece (``\titulo'') te ayudar� a pensar que lo que
est�s escribiendo es tuyo y no de otros. Para eso, basta con cambiar
la constante \verb|\titulo| y \verb|\autor| que aparece definida en el
fichero \texttt{constantes.tex}.

El segundo paso es crear la portada en
\texttt{Cascaras/cover.tex}. Como habr�s podido observar, \texis\
genera dos hojas de portada, al igual que hacen la mayor�a de los
libros. La primera portada es la que ir�a en la parte exterior del
documento encuadernado, mientras que la siguiente es una repetici�n
que aparece en la primera p�gina. A continuaci�n aparece una lista con
el texto que puede cambiarse usando los comandos de \texis; una vez
que se configuran, se debe invocar al comando \verb|\makeCover| para
generar las portadas:

\begin{itemize}
\item T�tulo del documento: aparece en las dos portadas. Por defecto
  se utilizar� la constante \verb|\titulo| definida en
  \texttt{constantes.tex}. No obstante, se puede indicar un t�tulo
  distinto usando \verb|\tituloPortada|. De esta forma, se pueden
  forzar saltos de l�nea artificiales si se desea.

\item Autor del documento: normalmente aparece tambi�n en las dos
  portadas. Igual que antes, si no se indica lo contrario se utiliza
  \verb|\autor|, aunque se puede cambiar con \verb|\autorPortada|.

\item Una imagen en la primera portada, normalmente el escudo
  institucional. El fichero a utilizar se define con
  \verb|\imagenPortada|. Tambi�n puede especificarse la escala a
  utilizar en el fichero si �ste es demasiado grande o peque�o con
  \verb|\escalaImagenPortada|.

\item Una fecha de publicaci�n, que aparece en la parte inferior de
  ambas portadas. Se utiliza el comando \verb|\fechaPublicacion|.

\item El ``tipo de documento'' que aparece en la primera portada. Si
  no se indica nada, ser� ``TESIS DOCTORAL''. Se puede modificar con
  \verb|\tipoDocumento|. Este manual por ejemplo lo establece en
  ``MANUAL DE USUARIO''.

\item El departamento y facultad al que est� asociado el
  documento. Aparece en ambas portadas, y se establece con
  \verb|\institucion|.

\item Un primer bloque de texto en la segunda portada, que aparece
  despu�s del t�tulo. Si no se indica lo contrario, en ese bloque
  aparecer� el texto ``Memoria que presenta para optar al t�tulo de
  Doctor en Inform�tica'' seguido del \verb|\autorPortada|. Se puede
  cambiar el contenido completo con
  \verb|\textoPrimerSubtituloPortada|.

\item Un segundo bloque de texto donde aparece ``Dirigida por el
  Doctor'' seguido del director del trabajo que se establece con
  \verb|\directorPortada|. El comando
  \verb|\textoSegundoSubtituloPortada| permite establecer otro texto
  distinto.
\end{itemize}

Las dos portadas en sus caras traseras pueden, adem�s, presentar otra
informaci�n auxiliar:

\begin{itemize}
\item Un breve recordatorio indicando que el documento est� preparado
  para su impresi�n a doble cara. Si se desea que aparezca, basta con
  llamar a \verb|\explicacionDobleCara|.

\item El ISBN del documento, en caso de poseerlo. Se define con
  \verb|\isbn|.

\item Informaci�n de copyright. Se puede indicar con
  \verb|\copyrightInfo|, y lo habitual ser� pasar como par�metro el
  \verb|\autor|.

\item Por defecto en la cara posterior de la primera portada aparecen
  unos ``cr�ditos'' a \texis, donde se indica que el documento se ha
  generado con \texis\ y la versi�n. Si no se desea que aparezca, se
  puede llamar a \verb|\noTeXiSCredits|, aunque nos gustar�a que lo
  incluyeras.
\end{itemize}

Por �ltimo, quiz� quieras cambiar la informaci�n de ``metadatos'' que
se incrustar� en el PDF generado. Los metadatos aparecen directamente
en el fichero \texttt{Tesis.tex} y, como indicamos en el cap�tulo
anterior y mostramos en la figura~\ref{cap2:fig:pdf}, son:

\begin{verbatim}
%
% "Metadatos" para el PDF
%
\ifpdf\hypersetup{%
    pdftitle = {\titulo},
    pdfsubject = {Plantilla de Tesis},
    pdfkeywords = {Plantilla, LaTeX, tesis, trabajo de
      investigaci�n, trabajo de Master},
    pdfauthor = {\textcopyright\ \autor},
    pdfcreator = {\LaTeX\ con el paquete \flqq hyperref\frqq},
    pdfproducer = {pdfeTeX-0.\the\pdftexversion\pdftexrevision},
    }
    \pdfinfo{/CreationDate (\today)}
\fi
\end{verbatim}

%Para adecuarlo a tu documento concreto, deber�s cambiar la entrada del
%tema y palabras clave. El t�tulo y autores se rellenan con las
%contantes correspondientes.

%-------------------------------------------------------------------
\section{Editando el texto}
%-------------------------------------------------------------------
\label{cap3:sec:edicion}

Una vez que se tiene el t�tulo y autores del documento puestos, el
trabajo de escritura consiste, en su mayor parte, en la creaci�n de
los correpondientes ficheros \LaTeX\ de cada uno de los cap�tulos y
ap�ndices.

%-------------------------------------------------------------------
\subsection{Nuevos cap�tulos (y ap�ndices)}
%-------------------------------------------------------------------

Seg�n la estructura de directorios vista en el cap�tulo anterior,
\texis\ te recomienda crear los cap�tulos en el directorio
\texttt{Capitulos} y los ap�ndices en \texttt{Apendices}.

Cuando crees un fichero en cualquiera de los directorios, se debe
a�adir en el fichero maestro (\texttt{Tesis.tex}) el nombre de ese
nuevo fichero para que se procese en el momento de la generaci�n:

\begin{verbatim}
\mainmatter

%---------------------------------------------------------------------
%
%                          Cap�tulo 1
%
%---------------------------------------------------------------------
%
% 01Introduccion.tex
% Copyright 2009 Marco Antonio Gomez-Martin, Pedro Pablo Gomez-Martin
%
% This file belongs to the TeXiS manual, a LaTeX template for writting
% Thesis and other documents. The complete last TeXiS package can
% be obtained from http://gaia.fdi.ucm.es/projects/texis/
%
% Although the TeXiS template itself is distributed under the 
% conditions of the LaTeX Project Public License
% (http://www.latex-project.org/lppl.txt), the manual content
% uses the CC-BY-SA license that stays that you are free:
%
%    - to share & to copy, distribute and transmit the work
%    - to remix and to adapt the work
%
% under the following conditions:
%
%    - Attribution: you must attribute the work in the manner
%      specified by the author or licensor (but not in any way that
%      suggests that they endorse you or your use of the work).
%    - Share Alike: if you alter, transform, or build upon this
%      work, you may distribute the resulting work only under the
%      same, similar or a compatible license.
%
% The complete license is available in
% http://creativecommons.org/licenses/by-sa/3.0/legalcode
%
%---------------------------------------------------------------------

\chapter{Introducci�n}

\begin{FraseCelebre}
\begin{Frase}
P�sose don Quijote delante de dicho carro, y haciendo en su fantas�a
uno de los m�s desvariados discursos que jam�s hab�a hecho, dijo en
alta voz:
\end{Frase}
\begin{Fuente}
  Alonso Fern�ndez de Avellaneda, El Ingenioso Hidalgo Don Quijote de
  la Mancha
\end{Fuente}
\end{FraseCelebre}

\begin{resumen}
  Este cap�tulo presenta una breve introducci�n a \texis.  El
  lector podr� hacerse una idea de qu� es y para qu� sirve. Tambi�n se
  encuentra aqu� una descripci�n del resto de cap�tulos del manual.
\end{resumen}


%-------------------------------------------------------------------
\section{Introducci�n}
%-------------------------------------------------------------------
\label{cap1:sec:introduccion}


Si est�s leyendo estas l�neas es muy posible que haya llegado la hora
de ponerte a escribir la tesis, despu�s de mucho tiempo dando vueltas
al �rea de investigaci�n concreta en el que est�s inmerso. O puede que
est�s a punto de empezar a escribir la memoria del proyecto de fin de
carrera, fin de master, o cualquier otro documento de cierta
envergadura.

Sea lo que sea lo que te traes entre manos, lo m�s probable es que no
sea f�cil hacerlo. Muy posiblemente no tengas a�n muy claro qu� vas a
escribir, pero tu tutor/director/profesor te ha dicho que vayas
empezando a plasmar esas ideas sobre el papel para tener algo firme, y
sentir que vas avanzando.

Y entonces viene el problema de c�mo escribirlo. Muy posiblemente
habr�s escrito alg�n art�culo en \LaTeX\ y est�s convencido de que esa
es la v�a a seguir para hacer un documento que superar� las 10 p�ginas
y que tendr� bibliograf�a. O puede, simplemente, que alguien te haya
dicho que lo mejor es que escribas el proyecto en \LaTeX\ porque la
apariencia final es mejor, porque es m�s c�modo, o cualquier otra
raz�n.

Sea como fuere, parece que est�s m�s o menos decidido a escribir tu
documento en \LaTeX. Bien hecho. Pero, �c�mo?. Al contrario de lo que
suele ocurrir en congresos y en revistas, no tienes disponible ninguna
p�gina en la que descargarte las ``instrucciones para los autores'',
con la c�moda plantilla en \LaTeX\ que t�, sufrido autor, simplemente
tienes que rellenar. No. Ahora las cosas son m�s complicadas.

As� que te vas a la gu�a de \LaTeX\ con la que empezaste (apostamos
que es la misma con la que hemos empezado todos), y ves las distintas
posibilidades que te ofrece en su ``\texttt{documentclass}'':
\texttt{article}, \texttt{report}, \texttt{book}, ... Y te quedas con
la �ltima. Pero te asaltan muchas preguntas. �C�mo organizo todo esto?
o �c�mo hago la portada? o incluso �qu� hago para que no ponga
``Chapter'', sino ``Cap�tulo''?. En ese punto, es de suponer, has
pedido ayuda a la gente de alrededor y/o a tu buscador de Internet
favorito. Y de alguna forma, te has encontrado leyendo estas l�neas.

Tenemos que decir que exactamente esa fue nuestra situaci�n cuando por
fin nos decidimos a escribir nuestras tesis. Desgraciadamente, ni la
gente que ten�amos alrededor ni nuestro buscador favorito supieron
contestarnos de forma satisfactoria, por lo que tuvimos que invertir
\emph{mucho tiempo} hasta conseguir que el resultado que sal�a de
nuestros \texttt{.tex} nos gustara, hasta que nos sentimos c�modos con
la estructura de los ficheros, con las macros disponibles y con el
modo de compilaci�n.

Y para que nadie m�s pueda utilizar como excusa el no saber c�mo
personalizar la clase \texttt{book} para retrasar el comienzo de su
tesis, para que nadie m�s se decida por Word u otro paquete ofim�tico
en vez de \LaTeX\ porque lo ve mucho m�s sencillo, en definitiva, para
que nadie pierda tanto tiempo como perdimos nosotros creando la
estructura, decidimos hacer p�blico el esqueleto b�sico que
construimos nosotros para hacerlas. Ese esqueleto b�sico o plantilla
es \texis.

En vez de hacer disponible la plantilla o ficheros \texttt{.tex} sin
ning�n contenido, proporcionamos un manual en formato PDF que (a no
ser que est�s leyendo directamente el c�digo \LaTeX), ser� lo que
est�s leyendo. Este manual ha sido creado \emph{con la propia
  plantilla}. Por lo tanto, la distribuci�n de \texis\ es en
realidad el c�digo fuente de \emph{su propio manual}. Con su c�digo
fuente entre tus manos, lo �nico que tienes que hacer es borrar su
contenido (\emph{este texto}), y rellenarlo con tu gran contribuci�n
al mundo.  Como podr�s comprobar, la estructura del propio manual
sigue el esquema de lo que podr�a ser una tesis, trabajo de
investigaci�n o proyecto de fin de carrera, precisamente para que sea
f�cil quitar el contenido textual y sustituirlo por el nuevo.

En los cap�tulos que siguen encontrar�s toda la informaci�n necesaria
para poder utilizar los ficheros \LaTeX\ para crear tus propios
documentos. Adem�s, el propio c�digo fuente est� lleno de comentarios
(especialmente en los ficheros que definen el estilo), por lo que
tambi�n en ellos encontrar�s una buena fuente de informaci�n. Eso es
especialmente importante en caso de que quieras modificar en algo el
aspecto final de tu documento.

Esperemos que te sea de utilidad. Si es as�, nos gustar�a que lo
reconocieras en la secci�n de agradecimientos. Si durante tu proceso
de escritura has a�adido alg�n aspecto que crees que puede ser
interesante para otros, no dudes en dec�rnoslo para intentar incluirlo
en siguientes versiones de la propia plantilla; tampoco dudes en
enviarnos sugerencias sobre las explicaciones de este manual para
poder mejorarlo con el tiempo. Por �ltimo, tambi�n puedes enviarnos el
resultado final para poner una referencia a �l en la p�gina de
descarga, donde, por cierto, puedes ver otros documentos creados con
la plantilla, lo que te permitir� coger ideas de cosas que puedes
variar. Recuerda que la versi�n m�s reciente de \texis\ est�
disponible en \url{http://gaia.fdi.ucm.es/projects/texis/}.

%-------------------------------------------------------------------
\section{Qu� es \texis}
%-------------------------------------------------------------------
\label{cap1:sec:que-es}

La plantilla que tienes entre las manos es, como hemos dicho, el
esqueleto del c�digo fuente de las Tesis Doctorales de los dos autores
\citep{GomezMartinMA2008PhD, GomezMartinPP2008PhD}. Por tanto, sirve
para escribir otras Tesis Doctorales u otros documentos con estructura
similar de forma f�cil.

\texis\ te permite adem�s generar el fichero utilizando tanto el
comando \texttt{latex} (que genera de forma nativa ficheros
\texttt{dvi} que luego se convierten a ficheros \texttt{ps} o
\texttt{pdf}), como \texttt{pdflatex}. De esta forma el usuario final
puede elegir entre cualquiera de las dos herramientas\footnote{Esto es
  �til por ejemplo cuando quieres utilizar \texttt{pdflatex} pero
  finalmente el servicio de publicaciones s�lo admite el uso de
  \texttt{latex}.}.  Aconsejamos, no obstante, la utilizaci�n de este
�ltimo, debido a que \texis\ contiene ciertos comandos para dotar al
PDF final de marcadores que permiten una navegaci�n c�moda por el
fichero utilizando los visores tradicionales.

\medskip

Como explicaremos en el cap�tulo siguiente, la plantilla se aprovecha
mejor en sistemas GNU/Linux. Nota que hemos dicho que la plantilla
``\emph{se aprovecha mejor}'' en sistemas GNU/Linux, no que \emph{no
pueda utilizarse} en Windows o Mac; es evidente que \LaTeX\ es
multiplataforma, y por lo tanto puede compilarse en cualquier sistema
que tenga instalada una distribuci�n del mismo.

La raz�n por esta ``desviaci�n positiva'' hacia Linux estriba en que
para hacer m�s c�modo el proceso de edici�n y compilaci�n, \texis\
proporciona ficheros que facilitan el proceso de generaci�n del
fichero PDF final, tal y como se describe en el
cap�tulo~\ref{cap:makefile}.  Esos ficheros adicionales s�lo funcionan
correctamente si son ejecutados en Linux.

%-------------------------------------------------------------------
\section{Qu� no es}
%-------------------------------------------------------------------
\label{cap1:sec:que-no-es}

Esta plantilla \emph{no} es un manual de \LaTeX, ni una gu�a de
referencia, ni un compendio de preguntas frecuentes. De hecho, no nos
consideramos expertos en \LaTeX, por lo que no tendr�amos fuerzas para
escribir algo as�. Si necesitas un manual de \LaTeX, puedes encontrar
muchos y muy buenos en Internet. Al final de este cap�tulo aparece una
lista con algunos de ellos.

La plantilla tampoco es \emph{una clase} de \LaTeX. Si miras el c�digo
fuente podr�s comprobar que el documento comienza con
\verb+\documentclass{book}+\footnote{Personalizado, eso s�, para que
  utilice DIN A-4, a doble cara y con letra de 11 puntos.}, por lo que
se basa en la clase \texttt{book}.

La plantilla tampoco te ayudar� a gestionar tu bibliograf�a. Los
\texttt{.bib} los tendr�s que crear y organizar t� ya sea de forma
manual o con alguna herramienta dise�ada para ello.

\medskip

Queremos una vez m�s insistir antes de terminar que no somos expertos
en \LaTeX.  Durante el proceso de escritura de nuestras Tesis nos
tuvimos que enfrentar a problemas de formato que tuvimos que
solucionar buscando en Internet o preguntando a personas cercanas. Y
podemos decir que pr�cticamente todos los problemas a los que nos
hemos enfrentado en nuestra vida como usuarios de \LaTeX\ est�n
resueltos aqu�, pues sendas Tesis han sido los documentos m�s extensos
que hemos escrito.

Por lo tanto, si tienes alguna duda concreta de \LaTeX, en vez de
preguntarnos a nosotros, busca en foros de Internet o en la
documentaci�n del paquete que est�s utilizando. A buen seguro
encontrar�s ah� la respuesta. Si la duda que tienes es relativa a la
plantilla, revisa los comentarios que encontrar�s en el c�digo fuente,
hay ciertas cosas de demasiado bajo nivel que hemos preferido no
contar en el texto. Y s�lo como �ltimo recurso, preguntanos a
nosotros, aunque ya te advertimos que puede que no sepamos
responderte. Querr�amos poder animarte a escribirnos tus dudas, pero
preferimos no hacerlo para no decepcionarte.


%-------------------------------------------------------------------
\section{Estructura de cap�tulos}
%-------------------------------------------------------------------
\label{cap1:sec:estructura}

El manual est� estructurado en los siguientes cap�tulos:

\begin{itemize}
\item El cap�tulo~\ref{cap2} describe a vista de p�jaro los distintos
  ficheros que forman \texis. Adem�s da una primera aproximaci�n
  a c�mo generar el documento final (\texttt{.pdf}).

\item El cap�tulo~\ref{cap3} se centra en el proceso de
  edici�n. Aunque aparentemente la tarea de escribir el texto es
  trivial, \texis\ proporciona una serie de comandos que pueden
  ser �tiles durante la escritura (al menos a nosotros nos lo
  parecieron). Este cap�tulo se centra en la explicaci�n de esos
  comandos.

\item El cap�tulo~\ref{cap4} pasa a describir c�mo se estructuran las
  im�genes en \texis. Igual que antes, esto puede parecer
  superfluo a un usuario medio de \LaTeX, pero \texis\ contiene
  algunos comandos que esperan esa estructura. Es el usuario el �ltimo
  que decide si utiliza esos comandos (y por lo tanto esa estructura)
  u opta por otra completamente distinta.

\item El cap�tulo~\ref{cap5} aborda la bibliograf�a y la gesi�n de los
  acr�nimos. Como se ver�, \texis\ dispone de algunas opciones de
  personalizaci�n que merecen un peque�o cap�tulo.

\item El cap�tulo~\ref{cap6} pone fin al manual, detallando las
  opciones del fichero \texttt{Makefile} que permiten una generaci�n
  c�moda del documento final en entornos Linux.
\end{itemize}

El manual tiene, por �ltimo, un ap�ndice que, si bien no es
interesante desde el punto de vista del usuario, nos sirve de excusa
para proporcionar el c�digo \LaTeX\ necesario para su creaci�n: a modo
de ``as� se hizo'', comenta brevemente c�mo fue el proceso de
escritura de nuestras tesis.


%-------------------------------------------------------------------
\section*{\NotasBibliograficas}
%-------------------------------------------------------------------
\TocNotasBibliograficas

El ``libro'' por el que la mayor�a de la gente empieza sus andaduras
con \LaTeX\ es \cite{ldesc2e} pues es relativamente corto, f�cil de
leer y de acceso p�blico (licencia GPL), por lo que se puede
conseguir la versi�n electr�nica f�cilmente. Un libro algo m�s
completo que �ste y que suele ser el segundo en orden de preferencia
es \cite{notsoshort} con la misma licencia. Dentro de los libros
dedicados a \LaTeX\ de libre distribuci�n, tambi�n se puede contar con
\cite{latexAPrimer}.

No obstante, los libros de \LaTeX\ m�s conocidos son ``The \LaTeX\
Companion'' \citep{latexCompanion} y ``\LaTeX: A Document Preparation
System'' \citep{LaTeXLamport}.

%-------------------------------------------------------------------
\section*{\ProximoCapitulo}
%-------------------------------------------------------------------
\TocProximoCapitulo

Una vez hecha una descripci�n de \texis, el pr�ximo cap�tulo
describe los ficheros que componen tanto la plantilla como el manual
que est�s leyendo. Tambi�n se explicar� c�mo se puede generar o
compilar el manual a partir de los \texttt{.tex} proporcionados. Por
lo tanto, el cap�tulo sirve como una primera aproximaci�n r�pida al
trabajo con \texis; al final del mismo seremos capaces de entender
la estructura de directorios propuesta y d�nde se encuentran los
ficheros que hay que editar para cambiar el contenido del documento
final.

No obstante, el cap�tulo siguiente debe verse �nicamente como una
primera aproximaci�n. El cap�tulo~\ref{cap:edicion} da m�s detalles
sobre el proceso de edici�n del documento, y el
cap�tulo~\ref{cap:makefile} dar� una alternativa al modo de
compilaci�n explicado.

% Variable local para emacs, para  que encuentre el fichero maestro de
% compilaci�n y funcionen mejor algunas teclas r�pidas de AucTeX
%%%
%%% Local Variables:
%%% mode: latex
%%% TeX-master: "../ManualTeXiS.tex"
%%% End:

%---------------------------------------------------------------------
%
%                          Cap�tulo 2
%
%---------------------------------------------------------------------
%
% 02EstructuraYGeneracion.tex
% Copyright 2009 Marco Antonio Gomez-Martin, Pedro Pablo Gomez-Martin
%
% This file belongs to the TeXiS manual, a LaTeX template for writting
% Thesis and other documents. The complete last TeXiS package can
% be obtained from http://gaia.fdi.ucm.es/projects/texis/
%
% Although the TeXiS template itself is distributed under the 
% conditions of the LaTeX Project Public License
% (http://www.latex-project.org/lppl.txt), the manual content
% uses the CC-BY-SA license that stays that you are free:
%
%    - to share & to copy, distribute and transmit the work
%    - to remix and to adapt the work
%
% under the following conditions:
%
%    - Attribution: you must attribute the work in the manner
%      specified by the author or licensor (but not in any way that
%      suggests that they endorse you or your use of the work).
%    - Share Alike: if you alter, transform, or build upon this
%      work, you may distribute the resulting work only under the
%      same, similar or a compatible license.
%
% The complete license is available in
% http://creativecommons.org/licenses/by-sa/3.0/legalcode
%
%---------------------------------------------------------------------

\chapter{Estructura  y generaci�n}
\label{cap2}

\begin{FraseCelebre}
\begin{Frase}
  La mejor estructura no garantizar� los resultados ni el rendimiento.
  Pero la estructura equivocada es una garant�a de fracaso.
\end{Frase}
\begin{Fuente}
Peter Drucker
\end{Fuente}
\end{FraseCelebre}

\begin{resumen}
  Este cap�tulo explica la estructura de directorios de \texis\
  as� como los ficheros m�s importantes, describiendo el cometido de
  cada uno. Tambi�n hace una primera aproximaci�n al proceso de
  generaci�n (o compilaci�n) del PDF final, aunque este tema ser�
  extendido posteriormente en el
  cap�tulo~\ref{cap:makefile}.
\end{resumen}

%-------------------------------------------------------------------
\section{Estructura de directorios}
%-------------------------------------------------------------------
\label{cap2:sec:estructura}

Como habr�s podido comprobar, la plantilla contiene bastantes ficheros
organizados en varios directorios. Esta secci�n explica el contenido
de cada uno de los directorios, para que seas capaz de encontrar el
directorio en el que deber�a estar un fichero concreto.

Existen los siguientes directorios:

\begin{description}
\item[Directorio ra�z] contiene el fichero principal del documento
  (tambi�n llamado fichero \emph{maestro}), que es el que se utiliza
  como entrada a \texttt{pdflatex} (o \texttt{latex}) y cuyo nombre es
  \texttt{Tesis.tex}. Tambi�n aparecen en el directorio otros ficheros
  que si bien no generan texto en el documento final cumplen ciertas
  funciones espec�ficas descritas en la
  secci�n~\ref{cap2:sec:directorio-raiz}. Por �ltimo, el directorio
  contiene tambi�n los ficheros \texttt{.bib} con la informaci�n
  bibliogr�fica as� como el fichero para generar el documento
  utilizando la aplicaci�n \verb+make+.

\item[Directorio \texttt{./Capitulos}] contiene los \texttt{.tex} de
  cada cap�tulo del documento.

\item[Directorio \texttt{./Apendices}] contiene los \texttt{.tex} de
  cada uno de los ap�ndices.

\item[Directorio \texttt{./Cascaras}] contiene los \texttt{.tex}
  responsables del contenido del resto de p�ginas del documento: el
  texto de la portada, agradecimientos, resumen, etc. En definitiva
  son los ficheros responsables de todo aquello que precede a los
  cap�tulos y sigue a los ap�ndices.

\item[Directorio \texttt{./Imagenes}] contiene las im�genes del
  documento. Dentro de �l aparecen varios directorios distintos. La
  gesti�n de im�genes (y por lo tanto la estructura de estos
  directorios) se describir� en el cap�tulo~\ref{cap:imagenes}.

\item[Directorio \texttt{./TeXiS}] contiene todos los ficheros
  relacionados con la propia plantilla, es decir, los ficheros que
  definen la apariencia final del documento, as� como los comandos que
  facilitan la edici�n que ser�n descritos en el
  cap�tulo~\ref{cap:edicion}. La creaci�n de un documento que se
  adhiere completamente al formato de \texis\ no necesitar� tocar
  ninguno de los ficheros de este directorio.

\item[Directorio \texttt{./VersionesPrevias}] Este directorio es
  usado por el \texttt{Makefile} cuando se realiza una copia de
  seguridad del estado del documento. Describiremos esta
  caracter�stica en el cap�tulo~\ref{cap:makefile}.
\end{description}

Existen por lo tanto, tres tipos de ficheros \texttt{.tex}: los
ficheros que contienen el texto principal del documento (cap�tulos y
ap�ndices), los ficheros que definen las partes adicionales del mismo
(como portada y agradecimientos), y los ficheros que determinan la
apariencia. En las tres secciones siguientes describimos cada uno de
ellos.

%-------------------------------------------------------------------
\section{Ficheros con el texto principal del documento}
%-------------------------------------------------------------------
\label{cap2:sec:ficheros-texto}

Estos \texttt{.tex} son los que contienen el texto tanto de los
cap�tulos como de los ap�ndices, por lo tanto son los ficheros que m�s
tiempo pasar�s editando. Est�n divididos en secciones, tienen figuras,
tablas, referencias bibliogr�ficas, y cualquier otro tipo de elemento
que quieras o debas a�adir.

En principio pueden contener cualquier c�digo \LaTeX. No obstante, no
olvides que si necesitas alg�n paquete especial que no se cargue por
defecto en la plantilla, deber�s incluir el \verb+\usepackage+
correspondiente en el documento maestro o en el fichero de pre�mbulo
de \texis, \url{TeXiS/TeXiS\_pream.tex} descrito en la
Secci�n~\ref{cap2:sec:ficheros-formato}.

El cap�tulo siguiente est� enteramente dedicado al proceso de edici�n
de estos ficheros.

%-------------------------------------------------------------------
\section[Ficheros del documento auxiliares]%
{Ficheros del documento auxiliares: las c�scaras del documento}
%-------------------------------------------------------------------
\label{cap2:sec:ficheros-auxiliares}

Estos ficheros, como ya hemos dicho, son los responsables del
contenido del resto de p�ginas del documento, todo aquello que no son
cap�tulos o ap�ndices. Son los siguientes (por orden de ``aparici�n''
en el documento final)\footnote{Si crees que no necesitas alguno de
  ellos, puedes eliminar su inclusi�n en el fichero maestro,
  \texttt{Tesis.tex}.}:

\begin{itemize}
\item \texttt{cover.tex}: responsable de las dos primeras hojas del
  documento, que forman las portada. Mediante comandos se definen el
  autor y t�tulo que aparecer� en la portada, la fecha de publicaci�n,
  facultad, etc. Como podr�s ver cuando lo edites, el fichero contiene
  los datos concretos para generar este manual. Los comandos se
  describen en la secci�n~\ref{cap3:sec:tituloYmetadatas}.

\item \texttt{dedicatoria.tex}: contiene el c�digo \LaTeX\ que crea la
  ``dedicatoria'' de la Tesis. Consiste en una hoja donde aparece
  alineada a la izquierda una frase indicando a qui�n se ``dedica'' el
  documento (en los libros serios pone algo como ``A mis padres'',
  aunque tambi�n hay autores en libros m�s distendidos, como
  \citeauthor{AIbyExample} que dice textualmente ``For Mum and Dad,
  who bought me my first computer, and therefore must share some
  responisibility for turning me into the geek that I am''
  \citep{AIbyExample}). Se pueden poner todas las p�ginas de
  dedicatorias que se deseen, utilizando la macro
  \verb|\putDedicatoria|, que recibe la cita completa y crea la hoja
  completa con la misma. Lo m�s c�modo, no obstante, es utilizar la
  macro \verb|\dedicatoriaUno| y (opcionalmente)
  \verb|\dedicatoriaDos| para establecer las dos dedicatorias y a
  continuaci�n invocar \verb|\makeDedicatorias| para generarlas. As�
  lo hace este manual.

\item \texttt{agradecimientos.tex}: contiene el texto de las �nicas
  p�ginas que tu familia y amigos van a leer de la Tesis: los
  agradecimientos. As� que piensa bien lo que pones, no olvides a
  nadie\footnote{Tampoco a nosotros por quitarte la preocupaci�n del
    aspecto final... \texttt{:-)}}.

  Es importante que no borres la l�nea que aparece justo despu�s del
  \verb|\chapter|,

\begin{verbatim}
\cabeceraEspecial{Agradecimientos}
\end{verbatim}

  ya que lo que hace es modificar la cabecera de la p�gina para que no
  aparezca con el mismo formato que en los cap�tulos. Puedes consultar
  la secci�n~\ref{cap3:ssec:capitulos-especiales} para obtener m�s
  detalles sobre esto.

\item \texttt{resumen.tex}: si quieres incluir antes del �ndice un
  peque�o resumen de tu trabajo, puedes hacerlo en este fichero. Al
  igual que en los agradecimientos no debes eliminar el comando
  \LaTeX\ del principio que altera la cabecera.

  Tanto el resumen como los agradecimientos antes explicados se
  convierten en dos ``cap�tulos sin numeraci�n'' que tambi�n ser�n
  listados en el �ndice de contenidos. No obstante, al aparecer antes
  que el texto principal del documento (los cap�tulos propiamente
  dichos), sus p�ginas ser�n numeradas con notaci�n romana, en
  lugar de con la ar�biga tradicional.

\item \texttt{bibliografia.tex}: en �l se configura la
  bibliograf�a del documento. En concreto, el fichero permite indicar
  tanto qu� ficheros \texttt{.bib} contienen las entradas
  bibliogr�ficas como una frase c�lebre (seguramente, ya habr�s notado
  que \texis\ permite iniciar los cap�tulos con una frase
  c�lebre), caracter�stica descrita con m�s detalle en la secci�n
  \ref{cap3:ssec:frases}.

  El cap�tulo~\ref{cap:bibliografia} hace una descripci�n m�s
  detallada del tipo de bibliograf�a que propone utilizar la plantilla
  (y que utiliza este manual).

\item \texttt{fin.tex}: En nuestras respectivas tesis, como ``cierre''
  incluimos una �ltima p�gina parecida a la dedicatoria con un par de
  frases c�lebres. El c�digo \TeX\ responsable se encuentra en este
  fichero.
\end{itemize}

Existen otros dos ficheros que no aparecen en este directorio pero que
generan p�ginas en el documento final. Son \texttt{TeXiS\_toc.tex} y
\texttt{TeXiS\_acron.tex} del directorio \texttt{TeXiS}, descritos en
la secci�n~\ref{cap2:sec:ficheros-formato}. Aparecen en ese directorio
debido a que no permiten ning�n tipo de personalizaci�n al usuario de
\texis.

%-------------------------------------------------------------------
\section{Directorio raiz}
%-------------------------------------------------------------------
\label{cap2:sec:directorio-raiz}

En el directorio ra�z aparecen, adem�s de \texttt{Tesis.tex}, el
documento maestro, otros tres ficheros \texttt{.tex} que no son
responsables de la generaci�n de ninguna p�gina del documento. Uno de
ellos, \texttt{config.tex} se describe en la
secci�n~\ref{cap3:sec:modos-compilacion}. Los otros dos son:

\begin{itemize}
\item \texttt{guionado.tex}: contiene una lista de aquellas palabras
  que, durante la edici�n del documento, se ha podido comprobar que
  \LaTeX\ divid�a mal. En esos casos, la alternativa mala es hacer
  peque�os ajustes en el p�rrafo para que esa palabra cuyos guiones
  \LaTeX\ no sabe colocar no quede cerca del final de la l�nea. La
  alternativa buena es a�adir la palabra a este fichero, colocando los
  guiones donde van. En el fichero proporcionado aparece una lista de
  algunas palabras de ejemplo.

\item \texttt{constantes.tex}: est� pensado para la definici�n de
  constantes que aparezcan a menudo en el texto. Por ejemplo, si se
  hace un documento sobre Cruise Control~\citep{CruiseControl}, para
  evitar tener que escribir cont�nuamente las dos palabras, es buena
  idea incluir una constante en el fichero que cree un comando para
  hacerlo m�s r�pidamente:

\begin{example}
\newcommand{\cc}{Cruise Control}
La nueva versi�n de \cc\ \ldots
\end{example}

En este fichero aparece definida la constante \verb|\titulo| que
contiene el t�tulo del documento y \verb|\autor| con el autor. Ambos
son utilizados en la portada. Tambi�n aparece definido el comando
\verb|\texis| que utilizamos en este manual para evitarnos escribir el
c�digo que escribe ``\texis''\ una y otra vez:

\begin{example}
\texis\ te permite generar el
fichero final tanto como .dvi
como en un .pdf.
\end{example}
\end{itemize}

Por �ltimo indicar que en el directorio ra�z aparecen los ficheros con
extensi�n \texttt{.bib} que contienen la informaci�n bibliogr�fica y
los \texttt{.gdf} para los acr�nimos (ver
cap�tulo~\ref{cap:bibliografia}) as� como el fichero \texttt{Makefile}
para la generaci�n autom�tica del documento final
(cap�tulo~\ref{cap:makefile}).

%-------------------------------------------------------------------
\section{Ficheros de la plantilla}
%-------------------------------------------------------------------
\label{cap2:sec:ficheros-formato}

El directorio \texttt{TeXiS} contiene los ficheros que definen la
apariencia final del documento. Si el formato de este manual te gusta
tal cual, no tendr�s por qu� tocar ninguno de estos ficheros. La
explicaci�n de su contenido aparece a continuaci�n. Su c�digo fuente
contiene numerosos comentarios y enlaces, por lo que no deber�a
suponerte demasiado problema modificarlos.

\begin{itemize}
\item \texttt{TeXiS\_cab.tex}: contiene la definici�n de la apariencia
  de las cabeceras de las p�ginas utilizadas en el documento. La
  plantilla utiliza el paquete \texttt{fancyhdr}. Sin embargo, la
  cabecera por defecto se ha modificado para que aparezca el n�mero
  del cap�tulo, as� como su nombre en min�sculas, junto con alg�n otro
  cambio menor.

\item \texttt{TeXiS.sty}: contiene los comandos que la plantilla
  proporciona para facilitar el proceso de edici�n. El uso de estos
  comandos est� explicado en el cap�tulo~\ref{cap:edicion}. A pesar de
  que la extensi�n distinta a la habitual (\texttt{.sty} en vez de
  \texttt{.tex}) puede imponer cierto respeto al principio, puedes
  abrir sin miedo el fichero para edici�n, pues es un fichero de
  \LaTeX\ normal, con definiciones de comandos tradicionales.

\item \texttt{TeXiS.bst}: contiene el estilo que utiliza la plantilla
  para generar la lista de las referencias bibliogr�ficas al final del
  documento. Las posibilidades de este estilo son descritas en el
  cap�tulo~\ref{cap:bibliografia}.

\item \texttt{TeXiS\_pream.tex}: este fichero contiene la mayor parte
  del c�digo del pre�mbulo del documento (lo que va antes del
  \verb|\begin{document}|). En �l aparecen incluidos un buen n�mero de
    paquetes que pueden ser �tiles en la elaboraci�n del documento,
    junto con una explicaci�n de para qu� sirven y, en algunas
    ocasiones, algunos ejemplos de uso. Existen incluso ciertos
    paquetes cuya inclusi�n aparece comentada pero que se mantienen,
    junto con su comentario correspondiente, por si pueden venir bien
    para documentos concretos que necesiten ciertas caracter�sticas
    que ni este manual ni nuestras tesis requirieron.

  \item \texttt{TeXiS\_cover.tex}: contiene el c�digo \TeX\ que genera
    la portada, y la hoja siguiente a la misma, que vuelve a tener los
    mismos datos pero sin el escudo.

  \item \texttt{TeXiS\_dedic.tex}: contiene el c�dito \TeX\ para
    generar las hojas de dedicatorias.

  \item \texttt{TeXiS\_toc.tex}: es el responsable de la generaci�n de
    los �ndices de cap�tulos, tablas y figuras que aparece en el
    documento.

  \item \texttt{TeXiS\_bib.tex}: es el encargado de que en el
    documento aparezca bibliograf�a. Incluido desde el fichero
    maestro, lo primero que hace es leer el fichero de configuraci�n,
    \texttt{Cascaras/configBibliografia.tex}.

    Como puedes comprobar, la bibliograf�a es tambi�n referenciada en
    el �ndice como un cap�tulo sin numerar; tambi�n se preocupa de
    cambiar la cabecera para que no se utilice la habitual del resto
    de cap�tulos.

\item \texttt{TeXiS\_acron.tex}: la plantilla tambi�n permite a�adir
  una lista de acr�nimos o abreviaturas utilizadas en el texto. En
  este fichero se incluyen los comandos necesarios para que aparezca
  esta lista. No obstante, para que la lista funcione, en el momento
  de la generaci�n se debe invocar a la herramienta correspondiente
  para que se creen los ficheros auxiliares necesarios para su
  generaci�n. En la descripci�n sobre la generaci�n dada en la
  secci�n~\ref{cap2:sec:compilacion} no se describe este proceso,
  por lo que el resultado contendr� una lista de acr�nimos
  vac�a. El uso de acr�nimos se describe con detalle en la
  secci�n~\ref{capBiblio:sec:glosstex}.

\item \texttt{TeXiS\_part.tex}: contiene los comandos relacionados con
  la posibilidad de dividir en \emph{partes} el documento final. Los
  detalles de qu� posibilidades ofrece \texis\ para hacerlo est�n
  descritas en la secci�n~\ref{cap3:ssec:partes}.

\end{itemize}
  
%-------------------------------------------------------------------
\section{Generando el documento}
%-------------------------------------------------------------------
\label{cap2:sec:compilacion}

Como ya se dijo en la introducci�n, \texis\ permite compilar el
documento\footnote{Cuando hablamos de ``compilaci�n'' nos referimos,
  por analog�a con el desarrollo software, a la generaci�n del fichero
  final (un PDF) resultado de analizar los ficheros fuente en \LaTeX.}
tanto con \verb+latex+ como \verb+pdflatex+.  Si has utilizado \LaTeX\
a trav�s de editores de texto espec�ficos (como Kile o WinEdt), es
posible que no sepas de qu� estamos hablando. Tanto \texttt{latex}
como \texttt{pdflatex} son dos aplicaciones que cogen un fichero
\texttt{.tex} como entrada y generan el documento final
``renderizado''. La diferencia entre ambas radica en el fichero de
salida que generan. En el primer caso, se genera un fichero
\texttt{.dvi}\footnote{\emph{Device independent}, o ``independiente
  del dispositivo'' (en el que se mostrar� el contenido).}, mientras
que en el segundo caso se genera un fichero PDF directamente.
Tradicionalmente se ha utilizado \texttt{latex}, convirtiendo despu�s
el fichero \texttt{.dvi} al formato deseado (como \texttt{.ps} o
\texttt{.pdf}). Sin embargo, en nuestro caso, aconsejamos la
utilizaci�n de \texttt{pdflatex}, debido a que, al generar de forma
nativa ficheros PDF, aprovecha algunas de las caracter�sticas
disponibles en los mismos. En particular, \texis\ contiene algunos
comandos \LaTeX\ que \texttt{pdflatex} aprovecha para a�adir
informaci�n de \emph{copyright} al fichero, as� como enlaces a cada
uno de los cap�tulos y secciones del documento, permitiendo una
navegaci�n r�pida por el mismo cuando se utilizan visores
(figura~\ref{cap2:fig:pdf}).

\begin{figure}[t]
  \centering
  %
  \subfloat[][Propiedades del documento]{
     \includegraphics[width=0.42\textwidth]%
                     {Imagenes/Bitmap/02/PropiedadesPDF}
     \label{cap2:fig:PropiedadesPDF}
  }
  \qquad
  \subfloat[][Tabla de contenidos]{
     \includegraphics[width=0.42\textwidth]%
                     {Imagenes/Bitmap/02/IndicePDF}
     \label{cap2:fig:TocPDF}
  }
 \caption{Capturas del visor de PDF\label{cap2:fig:pdf}}
\end{figure}


La plantilla incluye un fichero \texttt{Makefile} para automatizar la
generaci�n del fichero final\footnote{Los ficheros \texttt{Makefile}
  son ampliamente utilizados en el desarrollo de software. Son
  ficheros que sirven de entrada a la utilidad \texttt{make} que
  genera autom�ticamente los ficheros de resultado a partir de los
  archivos de c�digo fuente.} que es capaz de crear el PDF utilizando
cualquiera de las dos alternativas. No obstante, en este apartado no
entraremos en los detalles de este fichero, ya que existe un cap�tulo
dedicado enteramente a �l (cap�tulo~\ref{cap:makefile}).

Para generar el documento de este manual a partir de los ficheros de
\texis\ proporcionados, la forma inmediata es seguir el proceso
tradicional de generaci�n de cualquier fichero de \LaTeX, es decir,
ejecutar \texttt{pdflatex} (o \texttt{latex}), a continuaci�n ejecutar
\texttt{bibtex} para resolver las referencias bibliogr�ficas, y
posteriormente ejecutar un par de veces m�s \texttt{pdflatex} para
resolver las referencias cruzadas y que aparezcan en el documento
final.

En l�nea de comandos eso se traduce a las siguientes
�rdenes\footnote{Tambi�n es v�lido el uso de \texttt{latex} en lugar
  de \texttt{pdflatex}, pero el fichero generado (\texttt{.dvi})
  deber� despu�s ser convertido a PDF.}:

\begin{verbatim}
$ pdflatex Tesis
$ bibtex Tesis
$ pdflatex Tesis
$ pdflatex Tesis
\end{verbatim}

Si se utiliza alg�n editor de \LaTeX\ para la edici�n, tambi�n se
pueden utilizar sus teclas r�pidas (o en su defecto, sus botones u
opciones de men�) para generarlo; encontrar�s una explicaci�n al
respecto en la secci�n~\ref{cap3:sec:editores}.

%-------------------------------------------------------------------
\section*{\NotasBibliograficas}
%-------------------------------------------------------------------
\TocNotasBibliograficas

En este cap�tulo hemos descrito simplemente la estructura de
directorio de \texis, por lo que no existe ninguna fuente
relacionada adicional de consulta. Se mantiene este apartado por
simetr�a con el resto de cap�tulos. En un documento normal (tesis,
trabajo de investigaci�n) lo m�s probable es que todos los cap�tulos
puedan extenderse con notas de este tipo.

%-------------------------------------------------------------------
\section*{\ProximoCapitulo}
%-------------------------------------------------------------------
\TocProximoCapitulo

Una vez que se han descrito a vista de p�jaro los ficheros que
componen la plantilla y una primera aproximaci�n al proceso de
generaci�n del documento final (en PDF), el siguiente cap�tulo pasa a
describir el proceso de edici�n.

Eso cubre aspectos tales como los ficheros que deben modificarse para
a�adir nuevos cap�tulos o los comandos que \texis\ hace
disponibles para escribir ciertas partes de los mismos. El cap�tulo
describe tambi�n los dos modos de generaci�n del documento final que
pueden ser de utilidad durante el largo proceso de escritura. Por
�ltimo, el cap�tulo terminar� con ciertas consideraciones relativas a
los editores de \LaTeX\ utilizados as� como sobre la posibilidad de
utilizar un control de versiones.

% Variable local para emacs, para  que encuentre el fichero maestro de
% compilaci�n y funcionen mejor algunas teclas r�pidas de AucTeX
%%%
%%% Local Variables:
%%% mode: latex
%%% TeX-master: "../ManualTeXiS.tex"
%%% End:

...

% Ap�ndices
\appendix
%---------------------------------------------------------------------
%
%                          Ap�ndice 1
%
%---------------------------------------------------------------------
%
% 01AsiSeHizo.tex
% Copyright 2009 Marco Antonio Gomez-Martin, Pedro Pablo Gomez-Martin
%
% This file belongs to the TeXiS manual, a LaTeX template for writting
% Thesis and other documents. The complete last TeXiS package can
% be obtained from http://gaia.fdi.ucm.es/projects/texis/
%
% Although the TeXiS template itself is distributed under the 
% conditions of the LaTeX Project Public License
% (http://www.latex-project.org/lppl.txt), the manual content
% uses the CC-BY-SA license that stays that you are free:
%
%    - to share & to copy, distribute and transmit the work
%    - to remix and to adapt the work
%
% under the following conditions:
%
%    - Attribution: you must attribute the work in the manner
%      specified by the author or licensor (but not in any way that
%      suggests that they endorse you or your use of the work).
%    - Share Alike: if you alter, transform, or build upon this
%      work, you may distribute the resulting work only under the
%      same, similar or a compatible license.
%
% The complete license is available in
% http://creativecommons.org/licenses/by-sa/3.0/legalcode
%
%---------------------------------------------------------------------

\chapter{As� se hizo...}
\label{ap1:AsiSeHizo}

\begin{FraseCelebre}
\begin{Frase}
Pones tu pie en el camino y si no cuidas tus pasos, nunca sabes a donde te pueden llevar.
\end{Frase}
\begin{Fuente}
John Ronald Reuel Tolkien, El Se�or de los Anillos
\end{Fuente}
\end{FraseCelebre}

\begin{resumen}
Este ap�ndice cuenta algunos aspectos pr�cticos que nos planteamos en
su momento durante la redacci�n de la tesis (a modo de ``as� se hizo
nuestra tesis''). En realidad no es m�s que una excusa para que �ste
manual tenga un ap�ndice que sirva de ejemplo en la plantilla.
\end{resumen}

%-------------------------------------------------------------------
\section{Edici�n}
%-------------------------------------------------------------------
\label{ap1:edicion}

Ya indicamos en la secci�n~\ref{cap3:sec:editores} (p�gina
\pageref{cap3:sec:editores}) que \texis\ est� preparada para
integrarse bien con emacs, en particular con el modo Auc\TeX.

Eso era en realidad un s�ntoma indicativo de que en nuestro trabajo
cotidiano utilizamos emacs para editar ficheros \LaTeX. Es cierto que
inicialmente utilizamos otros editores creados expresamente para la
edici�n de ficheros en \LaTeX, pero descubrimos emacs y ha llegado
para quedarse (la figura~\ref{cap3:fig:emacs} mostraba una captura del
mismo mientras cre�bamos este manual). Ten en cuenta que si utilizas
Windows, tambi�n puedes usar emacs para editar; no lo consideres como
algo que s�lo se utiliza en el mundo Unix. Nosotros lo usamos a diario
tanto en Linux como en Windows.

No obstante, hay que reconocer que emacs \emph{no} es f�cil de
utilizar al principio (el manual de referencia de \cite{emacsStallman}
tiene m�s de 550 p�ginas); su curva de aprendizaje es empinada,
especialmente si quieres sacarle el m�ximo partido, o al menos
beneficiarte de algunas de sus combinaciones de teclas. Pero una vez
que consigues \emph{no} mover las manos para desplazar el cursor sobre
el documento, manejas las teclas r�pidas para a�adir los comandos
\LaTeX\ m�s utilizados y conoces las combinaciones de Auc\TeX\ para
moverte por el documento o buscar las entradas de la bibliograf�a, no
cambiar�s f�cilmente a otro editor.

Si quieres aprovechar emacs, no debes dejar de leer el documento que
nos introdujo a nosotros en el modo Auc\TeX, ``\emph{Creaci�n de
  ficheros \LaTeX\ con GNU Emacs}'' \citep{AtazLopezEmacs}.

%-------------------------------------------------------------------
\section{Encuadernaci�n}
%-------------------------------------------------------------------
\label{ap1:encuadernacion}

Si has mirado con un poco de atenci�n este manual, habr�s visto que
los m�rgenes que tiene son bastante grandes. \texis\ no configura
los m�rgenes a unos valores concretos sino que, directamente, utiliza
los que se establecen por defecto en la clase \texttt{book} de \LaTeX.

Aunque es m�s o menos reconocido que si \LaTeX\ utiliza esos m�rgenes
debe tener una raz�n de peso (y de hecho la tiene, se utilizan esos
para que el n�mero de letras por l�nea sea el id�neo para su lectura),
cuando se comienza a mirar el documento con los ojos del que quiere
verlo encuadernado, es cierto que parecen excesivos. Y empiezas a
abrir libros, regla en mano, para medir qu� m�rgenes utilizan. Y
reconoces que son mucho m�s peque�os (y razonables) que el de tu
maravilloso escrito. Al menos ese fue nuestro caso.

En ese momento, una soluci�n es \emph{reducir} esos m�rgenes para que
aquello quede mejor. Sin embargo nuestra opci�n no fue esa. Si tu
situaci�n te permite \emph{no} encuadernar el documento en formato
DIN-A4, entonces puedes ir a la reprograf�a de turno y pedir que, una
vez impreso, te guillotinen esos m�rgenes.

Tu escrito quedar� entonces en ``formato libro'', mucho m�s manejable
que el gran DIN-A4, y con unos m�rgenes mucho m�s razonables. La
figura~\ref{ap1:fig:encuadernacion} muestra el resultado, comparando
el tama�o final con el de un folio, que aparece superpuesto.

\figura{Bitmap/0A/encuadernacion}{width=0.7\textwidth}%
       {ap1:fig:encuadernacion}{Encuadernaci�n y m�rgenes guillotinados}

%-------------------------------------------------------------------
\section{En el d�a a d�a}
%-------------------------------------------------------------------
\label{ap1:cc}

Para terminar este breve ap�ndice, describimos ahora un modo de
trabajo que, si bien no utilizamos en su d�a para la escritura de la
tesis, s� hemos utilizado desde hace alg�n tiempo para el resto de
nuestros escritos de \LaTeX , incluidos \texis\ y �ste, su manual.

Estamos hablando de lo que se conoce en el mundo de la ingenier�a del
software como \emph{integraci�n cont�nua} \citep{Fowler06}. En
concreto, la integraci�n cont�nua consiste en aprovecharse del
servidor del control de versiones para realizar, en cada
\emph{commit} o actualizaci�n realizada por los autores, una
comprobaci�n de si los ficheros que se han subido son de verdad
correctos.

En el mundo del desarrollo software donde un proyecto puede involucrar
decenas de personas realizando varias actualizaciones diarias, la
integraci�n cont�nua tiene mucha importancia. Despu�s de que un
programador realice una actualizaci�n, un servidor dedicado comprueba
que el proyecto sigue compilando correctamente (e incluso ejecuta los
test de unidad asociados). En caso de que la actualizaci�n haya
estropeado algo, el servidor de integraci�n env�a un mensaje de correo
electr�nico al autor de ese \emph{commit} para avisarle del error y
que �ste lo subsane lo antes posible, de forma que se perjudique lo
menos posible al resto de desarrolladores.

Esa misma idea la hemos utilizado en la elaboraci�n de \texis\ y
de este manual. Cada vez que uno de los autores sub�a al SVN alg�n
cambio, el servidor comprobaba que el fichero maestro segu�a siendo
correcto, es decir, que se pod�a generar el PDF final sin errores.

No entraremos en m�s detalles de c�mo hacer esto. El lector interesado
puede consultar \citet{CCLatex}. Como se explica en ese art�culo
algunas ventajas del uso de esta t�cnica son:

\begin{figure}[t]
  \centering
  %
  \subfloat[][P�gina de descarga del documento generado]{
     \includegraphics[width=0.445\textwidth]%
                     {Imagenes/Bitmap/0A/dashboard}
     \label{ap1:fig:dashboard}
  }
  \qquad
  \subfloat[][M�tricas del proyecto]{
     \includegraphics[width=0.445\textwidth]%
                     {Imagenes/Bitmap/0A/metrics}
     \label{ap1:fig:metrics}
  }
 \caption{Servidor de integraci�n cont�nua\label{ap1:fig:cc}}
\end{figure}

\begin{itemize}
\item Se tiene la seguridad de que la versi�n disponible en el control
  de versiones es v�lida, es decir, es capaz de generar sin errores el
  documento final.

\item Se puede configurar el servidor de integraci�n cont�nua para que
  cada vez que se realiza un \emph{commit}, env�e un mensaje de correo
  electr�nico \emph{a todos los autores} del mismo. De esta forma
  todos los colaboradores est�n al tanto del progreso del mismo.

\item Se puede configurar para que el servidor haga p�blico (via
  servidor Web) el PDF del documento (ver
  figura~\ref{ap1:fig:dashboard}). Esto es especialmente �til para
  revisores del texto como tutores de tesis, que no tendr�n que
  preocuparse de descargar y compilar los \texttt{.tex}.
\end{itemize}

Por �ltimo, el servidor tambi�n permite ver la evoluci�n del proyecto.
La figura~\ref{ap1:fig:metrics} muestra una gr�fica que el servidor de
integraci�n cont�nua muestra donde se puede ver la fecha (eje
horizontal) y hora (eje vertical) de cada \emph{commit} en el
servidor; los puntos rojos representan commits cuya compilaci�n fall�.



% Variable local para emacs, para  que encuentre el fichero maestro de
% compilaci�n y funcionen mejor algunas teclas r�pidas de AucTeX
%%%
%%% Local Variables:
%%% mode: latex
%%% TeX-master: "../ManualTeXiS.tex"
%%% End:

...
\end{verbatim}

\medskip

Todos estos ficheros de cap�tulos y ap�ndices deben comenzar con el
comando \LaTeX\ \verb|\chapter|\footnote{Esto \emph{tambi�n} se
  cumple para los ap�ndices.}. El resto del fichero es un fichero
\LaTeX\ normal que tendr� secciones, subsecciones, figuras, tablas,
etc.

Al a�adir un nuevo fichero, es posible que tambi�n quieras a�adir su
nombre en el fichero \texttt{config.tex} para permitir la compilaci�n
r�pida de un �nico cap�tulo seg�n se cuenta en la
seccion~\ref{cap3:sec:compilacion-rapida}.

%-------------------------------------------------------------------
\subsection{Resumen del cap�tulo}
%-------------------------------------------------------------------

\texis\ permite incluir al comienzo de todos los cap�tulos un
breve resumen del mismo; este mismo manual lo hace. Para separarlo del
resto se utiliza un formato distinto.

En vez de cambiar el formato en todos y cada uno de los cap�tulos (y
ap�ndices), \texis\ proporciona un \emph{entorno} nuevo,
\texttt{resumen}, que lo hace por nosotros:

\begin{example}
\begin{resumen}
En este cap�tulo se describe...
\end{resumen}
\end{example}

El formato concreto est� definido en el fichero
\texttt{TeXiS/TeXiS.sty}, por lo que se puede cambiar a voluntad, lo
que provocar� el cambio en todas sus apariciones.

%-------------------------------------------------------------------
\subsection{Frases c�lebres}
%-------------------------------------------------------------------
\label{cap3:ssec:frases}

Como habr�s podido comprobar leyendo este manual, \texis\ permite
adem�s escribir en cada cap�tulo una ``frase c�lebre'' que es a�adida
inmediatamente despu�s del t�tulo del mismo, alineada a la derecha.

Para a�adir la frase (que est� formada por la cita en cuesti�n y su
autor), \texis\ define un nuevo entorno \texttt{FraseCelebre},
dentro del cual se especifican cada una de ellas con otros dos
entornos, \texttt{Frase} y \texttt{Fuente}:

\begin{example}
\begin{FraseCelebre}
\begin{Frase}
Nadie espere que yo diga algo.
\end{Frase}
\begin{Fuente}
Mafalda
\end{Fuente}
\end{FraseCelebre}
\end{example}

Evidentemente, las frases c�lebres pueden a�adirse en todos los
cap�tulos, incluidos los ``especiales'' (aquellos que no tienen
numeraci�n normal) como el cap�tulo de agradecimientos. Para hacerlo,
basta con utilizar los comandos anteriores.

Un cap�tulo donde es algo m�s complicado es el ``cap�tulo'' de
\emph{bibliograf�a}. Esto es debido a que la generaci�n del cap�tulo
completo consiste en una mera invocaci�n al comando
\verb+bibliography+

\begin{verbatim}
\bibliography{fichero1,fichero2}
\end{verbatim}

En el \verb|documentclass| que estamos utilizando (\texttt{book}) eso
significa que se crear� un nuevo \emph{cap�tulo} con la lista de
referencias. Si en ese cap�tulo se quiere a�adir una cita (como
hacemos por ejemplo en este manual), hay que realizar algunas tareas
adicionales. Naturalmente \texis\ las hace por nosotros, por lo que,
como se mencion� en la secci�n~\ref{cap2:sec:ficheros-auxiliares}, lo
�nico que tendremos que hacer es editar el fichero
\texttt{bibliografia.tex}, buscar la frase c�lebre del manual y
cambiarla a voluntad.

Antes de terminar, decir que, igual que en el caso del resumen, la
apariencia de la frase c�lebre se puede modificar en el fichero
\texttt{TeXiS/TeXiS.sty}.

%-------------------------------------------------------------------
\subsection{Secciones no numeradas}
%-------------------------------------------------------------------
\label{cap3:ssec:secciones-no-numeradas}

Como habr�s podido comprobar, en este manual todos los cap�tulos
terminan con dos secciones no numeradas, una de ellas con unas notas
bibliogr�ficas, y otra que tiene un peque�o resumen del siguiente
cap�tulo.

Aunque para el manual no son en realidad necesarias (especialmente la
de notas bibliogr�ficas, que en muchos cap�tulos nos ha costado
rellenar\ldots), las hemos puesto para que sirvan de ejemplo en el
\texttt{.tex}.

En principio, para poner una secci�n no numerada basta con utilizar la
``versi�n estrellada'' del comando \LaTeX\ correspondiente. Es decir,
utilizar \verb+\section*+ para a�adir una secci�n sin n�mero. El
problema en nuestro caso es que este comando no parece funcionar
correctamente con el paquete \texttt{fancyhdr}. \texis\ utiliza
ese paquete para configurar la cabecera y pie de p�gina; en concreto
para indicar que se desea que el n�mero de p�gina aparezca en las
esquinas ``externas'', mientras que en las esquinas internas debe
aparecer el nombre del cap�tulo (en las hojas pares o izquierdas) y
secci�n (en las impares o derechas). El mismo paquete es el que se
utiliza para que aparezca el n�mero de p�gina en la primera p�gina de
un cap�tulo y para cierta informaci�n que aparece cuando se genera el
documento en ``modo borrador'', seg�n aparece descrito en la
secci�n~\ref{cap3:sec:modos-compilacion}.

El problema aparece cuando una secci�n no numerada excede
el l�mite de la p�gina en la que empieza. En ese caso, la cabecera en
la que aparece el nombre de la secci�n en vez de contener el t�tulo de
esa secci�n sin numerar, seguir� mostrando la �ltima secci�n numerada.

La soluci�n es modificar a mano la cabecera, en concreto modificar la
configuraci�n de la cabecera donde aparece el t�tulo de la secci�n
actual (la parte izquierda de las p�ginas impares). Para eso, tras
consultar la documentaci�n del paquete, se aprende que hay que
utilizar el comando \verb+\markright+. Por ejemplo:

\begin{verbatim}
\section*{Notas bibliogr�ficas\markright{Notas bibliogr�ficas}}
\end{verbatim}

Como puede verse, en el propio comando \verb+\section*+, se incluye una
llamada a \verb+\markright+, que contiene el texto que debe aparecer a
en la cabecera. Con esto se soluciona el problema de las cabeceras.

Otro ``problema'' de las secciones sin numerar es que no se meten en
la tabla de contenidos que se incluye al principio del documento;
tampoco aparecen en el ``contenido'' del PDF listado por el visor que
mostrabamos en la figura~\ref{cap2:fig:pdf}\footnote{Ponemos
  \emph{problema} entre comillas porque normalmente se utiliza la
  versi�n con estrella de los comandos \texttt{section} precisamente
  para evitar que una secci�n aparezca en el �ndice.}. Sin embargo, en
nuestro caso prefer�amos que tambi�n las secciones aparecieran en el
�ndice (es decir, que la �nica diferencia entre las secciones
numeradas y las no numeradas fuera, precisamente, la ausencia de
numeraci�n). Para que aparezca, por lo tanto, se debe a�adir
expl�citamente la secci�n en la tabla de contenidos, con el comando:

\begin{verbatim}
\addcontentsline{toc}{section}{Notas bibliogr�ficas}
\end{verbatim}

\noindent que debe ejecutarse \emph{despu�s} del comando
\verb+\section*+. Por lo tanto, para a�adir una secci�n sin numerar
como la de ``Notas bibliogr�ficas'', el c�digo \LaTeX\ final que hay
que poner es:

\begin{verbatim}
%--------------------------------------------------------------
\section*{Notas bibliogr�ficas\markright{Notas bibliogr�ficas}}
%--------------------------------------------------------------
\addcontentsline{toc}{section}{Notas bibliogr�ficas}
\end{verbatim}

Entendemos que invocar a los comandos anteriores cada vez que se desea
una de estas secciones no numeradas es tedioso. Por ello \texis\
proporciona una serie de comandos (definidos en el fichero
\texttt{./TeXiS/TeXiS\_cab.tex}) que permiten a�adir f�cilmente cuatro
tipos de secciones sin numerar. Las secciones son los siguientes (ver
tabla~\ref{cap3:tab:seccionesnonumeradas}):

\begin{table}[t]
\footnotesize
\centering
\begin{tabular}{|l|c|c|}
\hline
Texto & Comando para \texttt{section} & Comando para �ndice \\
\hline
\hline
Conclusiones & \verb+\Conclusiones+ & \verb+\TocConclusiones+ \\
\hline
En el pr�ximo cap�tulo\ldots & \verb+\ProximoCapitulo+ & \verb+\TocProximoCapitulo+ \\
\hline
Notas bibliogr�ficas & \verb+\NotasBibliograficas+ & \verb+\TocNotasBibliograficas+ \\
\hline
Resumen & \verb+\Resumen+ & \verb+\TocResumen+ \\
\hline
\end{tabular}
\caption{Secciones no numeradas soportadas por \texis
   \label{cap3:tab:seccionesnonumeradas}}
\end{table}

\begin{itemize}
\item ``Conclusiones'': el manual no utiliza esta secci�n sin numerar,
  pero s� puede ser razonable utilizarlo a modo de resumen al final
  del cap�tulo de otro tipo de documentos. 

\item ``Notas bibliogr�ficas'': tambi�n utilizado en este documento,
  es �til para dar otras referencias bibliogr�ficas que por cualquier
  raz�n no se cit� en el texto.

\item ``En el pr�ximo cap�tulo...'': s� se ha utilizado en el manual,
  y puede servir para enlazar el contenido del cap�tulo con el
  siguiente.

\item ``Resumen'': con un objetivo parecido al de conclusiones pero
  con distinto t�tulo; tampoco lo utilizamos en el manual.
\end{itemize}

Como se puede ver en la tabla, para cada una de estas secciones
aparecen dos comandos, uno para el comando \verb+\section*+ y otro
para a�adir el �ndice, de forma que la definici�n de, por ejemplo, la
secci�n de ``En el pr�ximo cap�tulo...'' quedar�a:

\begin{verbatim}
%--------------------------------------------------------------
\section*{\ProximoCapitulo}
%--------------------------------------------------------------
\TocProximoCapitulo
\end{verbatim}

Somos conscientes de que los dos comandos podr�an haberse unificado en
uno s�lo, como \verb+\SeccionProximoCapitulo+ y que �l mismo hiciera todo el
trabajo (es decir, pusiera el \verb+\section*{...}+ as� como el
\verb+\addcontestline+). Sin embargo, esta soluci�n no es compatible
con la capacidad de los editores de resaltar secciones, ya que los
editores simplemente buscan la cadena ``\verb+\section+'' para
resaltarlo (ver figura~\ref{cap3:fig:emacs}).

\figura{Bitmap/03/SeccionesEmacs}{width=0.7\textwidth}%
       {cap3:fig:emacs}{Resaltado de secciones en emacs}

Es por ello que, a pesar de ser m�s tedioso, optamos por la
alternativa complicada: si se quiere meter una secci�n sin numerar, se
debe primero utilizar el comando \verb+\section*+, a�adiendo como
texto el comando que aparece en la segunda columna de la
tabla~\ref{cap3:tab:seccionesnonumeradas}, y posteriormente se utiliza
el otro comando para a�adirlo al �ndice. Separ�ndolo as�, adem�s,
permite al usuario de \texis\ decidir si quiere o no que la
secci�n aparezca en el �ndice.

%-------------------------------------------------------------------
\subsection{Cap�tulos especiales}
%-------------------------------------------------------------------
\label{cap3:ssec:capitulos-especiales}

Relacionado con las cabeceras de la secci�n anterior, \texis\
soporta (y este manual tiene) cap�tulos ``especiales'' que aparecen
sin numerar. Estos ``cap�tulos'' son, en concreto, la parte de
agradecimientos y resumen, los �ndices y la bibliograf�a.

Dado que todos ellos se caracterizan por la ausencia de secciones, no
tiene sentido mantener la cabecera utilizada en el resto del texto.
Por lo tanto, configuramos sus cabeceras para que en ambas p�ginas
aparezca directamente el t�tulo del cap�tulo (tambi�n sin n�mero).

Para hacerlo, \texis\ dispone del comando
\verb+\cabeceraEspecial+, que recibe como par�metro el nombre del
cap�tulo. De esta forma, el cap�tulo de agradecimientos comienza con:

\begin{verbatim}
\chapter{Agradecimientos}

\cabeceraEspecial{Agradecimientos}

\begin{FraseCelebre}
...
\end{verbatim}

que provoca un cambio en la cabecera que se debe utilizar.

Los cap�tulos sin numerar de este manual se encargan de configurar la
propia cabecera por lo que si partes de ellos para escribir tu
documento no deber�s preocuparte de nada (m�s all� de \emph{no} borrar
el comando).

Si incluyes nuevos cap�tulos sin numerar, has de saber que:

\begin{itemize}
\item No debes olvidar invocar el comando anterior al principio del
  cap�tulo sin numerar.

\item El comando anterior \emph{sobreescribe} el funcionamiento normal
  de la cabecera, por lo que se debe llamar al comando
  \verb+\restauraCabecera+ para reestablecerlo \emph{despu�s} del
  cap�tulo especial. Es importante resaltar el \emph{despu�s} pues
  debe hacerse cuando el cap�tulo \emph{ya ha terminado} y o bien se
  ha empezado el siguiente o bien se ha forzado el final de p�gina con
  un \verb+\newpage+. \texis\ ya hace esto autom�ticamente justo
  antes del primer cap�tulo (en \texttt{Tesis.tex}). Sin embargo, si
  incluyes alg�n cap�tulo especial m�s adelante en el documento, no
  debes olvidar restaurar la cabecera.
\end{itemize}

%-------------------------------------------------------------------
\subsection{Dividiendo el documento en partes}
%-------------------------------------------------------------------
\label{cap3:ssec:partes}

En ocasiones la estructura del documento tiene dos o m�s partes
claramente diferenciadas. Por ejemplo un libro puede tener una primera
parte de conceptos b�sicos con unos pocos cap�tulos y otra de
conceptos avanzados con el resto.

\LaTeX\ permite especificar distintas partes utilizando el comando
\verb|\part|. El resultado es la inserci�n de una
nueva hoja con el n�mero (en romanos) y t�tulo de la parte y la
adaptaci�n del �ndice de contenidos para incluir la informaci�n de esa
nueva parte.

Obviamente, \texis\ tambi�n permite la inclusi�n de distintas partes
(y este manual las tiene a modo de ejemplo). Sin embargo, en vez de
utilizar directamente el comando de \LaTeX, aconsejamos el uso de
comandos del propio \texis\ que tienen funcionalidad adicional.

En concreto, los comandos de \texis\ relacionados con las partes del
documento (y que describiremos a continuaci�n) permiten a�adir una
peque�a descripci�n de la parte que comienza en su hoja de t�tulo y
una  descripci�n m�s larga en la parte trasera (s�lo si el documento
est� configurado ``a dos caras'', especificando \texttt{twoside} en el
\texttt{documentclass} del  principio del documento).

\texis\ tambi�n se preocupa de que en el �ndice de contenidos del PDF
final la bibliograf�a (y en caso de existir la �ltima hoja con la
frase c�lebre) \emph{no} aparezcan ligados a la �ltima parte del
documento, sino que est�n en su mismo nivel.

Dicho todo esto, aconsejamos que, igual que se hace en el c�digo de
este manual, existan ficheros para definir cada una de las partes (en
el manual se llaman \texttt{Capitulos/Parte1.tex}, etc.). Estos
ficheros se incluyen desde el documento maestro justo antes del primer
cap�tulo de esa parte.

Los comandos de \texis\ relacionados con las partes del documento son
cuatro:

\begin{itemize}
\item \verb|\partTitle|: permite especificar el t�tulo de la parte que
  comenzar�.

\item \verb|\partDesc|: para indicar el texto descriptivo que
  aparecer� en la ``portada'' de esa parte. Es opcional; si no se
  indica, no aparecer� descripci�n.

\item \verb|\partBackText|: sirve para especificar el texto que
  aparecer� en la parte trasera de la hoja que delimita esa nueva
  parte. Es responsbilidad del autor asegurarse de que ese texto entra
  perfectamente en una �nica cara. Igual que el anterior, es opcional.

\item \verb|\makepart|: tras indicar el t�tulo y, opcionalmente,
  descripci�n y texto trasero, este comando construye la hoja que
  define esa parte del documento. Si se desea crear una parte sin
  numerar (lo que en \LaTeX\ suele conseguirse con la versi�n ``con
  estrella'' del comando), se puede utilizar \verb|\makespart| (la
  \texttt{s} solicita la versi�n \emph{starred}).
\end{itemize}

A modo de ejemplo este manual contiene tres partes; la primera de ella
cubre los tres primeros cap�tulo y tiene tanto descripci�n como texto
en la parte trasera. La segunda tiene �nicamente una descripci�n y la
tercera y �ltima, para los ap�ndices, no tiene ni descripci�n ni texto
trasero.

El c�digo \LaTeX\ para la definici�n de la primera parte es:

\begin{verbatim}
\partTitle{Conceptos b�sicos}

\partDesc{Esta primera parte del manual presenta los conceptos 
  b�sicos de \texis. Contiene un cap�tulo de introducci�n, 
  seguido de una descripci�n de la estructura de \texis\ y
  c�mo se genera el documento final, para terminar con un
  cap�tulo en el que se describe el proceso de edici�n sugerido
  y los comandos que \texis\ proporciona para facilitar dicho
  proceso.}

\partBackText{En realidad la divisi�n por partes del manual no
  aporta demasiado al lector; se ha dividido en varias partes
  debido a que, en la pr�ctica, el c�digo de este manual sirve
  como ejemplo de uso de \texis.

  En un contexto distinto, es posible que un manual de este
  tipo no habr�a tenido estas partes as� de diferenciadas.}

\makepart
\end{verbatim}


%-------------------------------------------------------------------
\section{Programando en \LaTeX}
%-------------------------------------------------------------------
\label{cap3:sec:programando}

Uno de los aspectos que diferencia a \LaTeX\ de los sistemas
ofim�ticos tradicionales para creaci�n de documentos es el modelo
subyacente que utiliza. En realidad, todo lo que el autor escribe en
sus ficheros \LaTeX\ es ``\emph{ejecutado}'' por el int�rprete de
\LaTeX\ hasta generar el documento final. Por lo tanto, se puede decir
que b�sicamente, cuando se escribe en \LaTeX\ se ``est� programando''
lo que posteriormente ser� un programa que generar� nuestro documento
final. Afortunadamente esa sensaci�n de ``programador'' no se tiene en
condiciones normales durante el proceso de autor�a. Sin embargo esta
peculiaridad s� se puede aprovechar para facilitar el proceso de
edici�n.

Ya hemos visto en el cap�tulo anterior un ejemplo de c�mo la
posibilidad de crear \emph{comandos} de \LaTeX\ nos permite establecer
``constantes'' que nos evitan tener que escribir palabras que
utilizaremos a menudo durante el texto. Sin embargo, profundizando un
poco m�s en el ``lenguaje'' que hay por debajo (por debajo de \LaTeX\
est� \TeX) se puede comprobar que pone a nuestra disposici�n algunas
estructuras conocidas por los programadores como los \texttt{if}.

%-------------------------------------------------------------------
\section{Modos de generaci�n del documento}
%-------------------------------------------------------------------
\label{cap3:sec:modos-compilacion}

Aprovechando esto, \texis\ est� preparada para admitir dos
\emph{configuraciones de generaci�n} o ``\emph{compilaci�n}''
distintas que, imitando los nombres tradicionales en el desarrollo
software, llamamos configuraci�n en modo ``release'' y en modo
``debug'' (o de depuraci�n):

\begin{itemize}
\item La configuraci�n en modo ``release'' est� pensada para la
  versi�n ``definitiva'', por lo que genera un fichero con la
  apariencia final del documento.

\item La configuraci�n en modo ``debug'' puede verse como una versi�n
  ``borrador''. En este caso el documento incluye ciertos elementos
  que no se desea incluir en la versi�n final, como comentarios en el
  propio texto.
\end{itemize}

La existencia de estos dos modos de compilaci�n puede sonar extra�a al
principio. En realidad, su utilidad depende del modo de escribir el
documento de cada uno. En nuestro caso, los cap�tulos de la tesis se
escribieron en un proceso ``iterativo'' de tal forma que inclu�amos
comentarios que quer�amos que aparecieran al imprimir ``la versi�n de
depuraci�n'', pero no quer�amos preocuparnos de tener que recordar
borrar llegado el momento de imprimir la versi�n final. Por otro lado,
cuando el documento es escrito por m�s de un autor (como este manual),
la posibilidad de poner comentarios f�cilmente descartables es
especialmente �til.

Los ficheros descargados est�n configurados para compilar la versi�n
definitiva; para cambiarla a la versi�n de ``depuraci�n'', basta con
cambiar el fichero \texttt{config.tex} del directorio ra�z. En cierto
momento al principio del fichero aparecen las l�neas siguientes.

\begin{verbatim}
% Comentar la l�nea si no se compila en modo release.
% TeXiS har� el resto
\def\release{1}
\end{verbatim}

Para generar el fichero con la configuraci�n de depuraci�n, basta con
comentar la l�nea en la que se ``define'' el s�mbolo
\texttt{release}\footnote{El comando recuerda a la orden del
  preprocesador de C/C++ ``\texttt{\#define release 1}''.}.

El primer efecto inmediato es que la plantilla a�ade autom�ticamente
como pie de p�gina el texto:

\medskip

{\small \sc Borrador -- \today}

\medskip

De esta forma, si tienes varias versiones imprimidas puedes estar
tranquilo de que no se te mezclar�n, pues adem�s de marcar que es un
borrador, aparece la fecha en la que se gener� el fichero.

En los tres apartados siguientes se describen tres comandos definidos
por \texis\ cuyo comportamiento depende del modo de compilaci�n.

\subsection{Comando \texttt{com}}
\label{cap3:subsec:comando-com}

El comando \verb|\com| permite a�adir un comentario que aparecer� (en
modo depuraci�n) en un p�rrafo aparte, con un ancho de l�nea algo
superior a lo normal y rodeado de un cuadro negro.

% En este caso no podemos utilizar el entorno example, porque para
% poner el comentario de verdad no podemos utilizar el comando real,
% \com, pues en la compilaci�n en release no saldr�a. He intenado hace
% que quede con la misma apariencia que en el ejemplo, pero no lo he
% logrado :(

Como ejemplo, el c�digo \LaTeX:

\begin{verbatim}
\com{Lo que sigue podr�a en realidad ser una secci�n distinta...}
\end{verbatim}

Se convierte en:

\comImpl{Lo que sigue podr�a en realidad ser una secci�n distinta...}

Hay que advertir que el recuadro anterior no tiene ning�n control
sobre los saltos de p�gina, por lo que ante comentarios demasiado
grandes (que no entran en lo que queda de p�gina), provoca que se
salte el resto de la misma y aparezca el comentario en la siguiente.

\subsection{Comando \texttt{comp}}
\label{cap3:subsec:comando-comp}

El comando anterior es muy �til pero debido a su tama�o puede no ser
recomendable para peque�os comentarios ``integrados'' dentro de un
p�rrafo. Para eso existe otro comando, \verb|\comp|, que hace
precisamente eso, permitir a�adir peque�os comentarios directamente en
el propio p�rrafo (\texttt{comp} viene de \textbf{COM}entario en \textbf{P}�rrafo).

% Igual que antes, no vale el "example".

El c�digo:

\begin{verbatim}
El juego ``Vampire: the Masquerade'', publicado en 1998,
requiri� 12 desarrolladores durante 24 meses, casi dos millones
de d�lares y unas 366.000 l�neas de c�digo.\comp{300.000 para
el juego, y 66.000 de scripts.}
\end{verbatim}

Se convierte en:

\smallskip

El juego ``Vampire: the Masquerade'', publicado en 1998,
requiri� 12 desarrolladores durante 24 meses, casi dos millones
de d�lares y unas 366.000 l�neas de c�digo.\compImpl{300.000 para
el juego, y 66.000 de scripts.}

\subsection{Comando \texttt{todo}}
\label{cap3:subsec:comando-todo}

Este comando permite a�adir comentarios para indicar tareas que a�n
faltan por hacer. Los inform�ticos solemos marcar esos comentarios en
nuestro c�digo fuente utilizando la ``palabra''
\texttt{TODO}\footnote{Que en realidad no tiene nada que ver con la
  palabra espa�ola, sino con las inglesas ``\emph{to do}'', que puede
traducirse aqu� a ``\emph{por hacer}''.}.

El comando \verb+\todo+ encierra el texto entre llaves y lo
antecede con la marca ``TODO'' en negrita, de forma que el
c�digo:

% Igual que antes, no vale el "example".

\begin{verbatim}
Existen autores que piensan que ense�ar programaci�n orientada
a objetos en el primer curso de programaci�n (CS1) es
beneficioso para los alumnos\todo{Meter referencias...}.
\end{verbatim}

se convierte en la versi�n de depuraci�n en:

\smallskip

Existen autores que piensan que ense�ar programaci�n orientada a
objetos en el primer curso de programaci�n (CS1) es beneficioso para
los alumnos\todoImpl{Meter referencias...}.

\medskip

Y, al igual que los anteriores, cuando se compila el documento en
``modo release'', el comando no tiene ning�n efecto.

\bigskip

Es importante destacar que en los dos comandos que van dentro de los
p�rrafos (\verb+\comp+ y \verb+\todo+) \emph{no
  se debe poner ning�n espacio antes del comando}. En caso de ponerse
el espacio, �ste \emph{aparecer�a} en la versi�n Release, cuando el
comando no tiene ning�n efecto:

\com{Aqu� utilizamos el entorno \texttt{example}, porque asumimos que
  la versi�n que estar� leyendo el usuario es la Release. Si est�s
  leyendo este texto, el ejemplo no lo entender�s, porque est�s
  manejando la versi�n de depuraci�n, y a la derecha ver�s tambi�n la
  secci�n ``\textbf{TODO}''.}

\begin{example}
... beneficioso para los
alumnos \todo{Meter 
referencias...}.
\end{example}

Para que cuando se genera el documento en modo depuraci�n quede bien,
el propio comando \emph{a�ade} el espacio de separaci�n entre el texto
que le precede y la apertura de la llave.

\medskip

Ten en cuenta, que al hacer uso de estos comandos para depuraci�n (\verb|\com|, \verb|\comp| o \verb|\todo|) el documento generado contendr� m�s texto que el final en \emph{release}. Eso significa que el n�mero de p�ginas variar�, y la maquetaci�n general tambi�n. Por tanto, \emph{no} debes utilizar el resultado de la generaci�n en depuraci�n para averiguar, por ejemplo, si una figura queda cerca del punto donde es referenciada, o si en una misma p�gina aparecen dos elementos flotantes.

%-------------------------------------------------------------------
\section{Acelerando la compilaci�n}
%-------------------------------------------------------------------
\label{cap3:sec:compilacion-rapida}

Cuando el documento va teniendo m�s y m�s p�ginas, compilarlo una y
otra vez hasta dar con el tama�o exacto que queremos darle a una
imagen, o para ver si una referencia queda bien generada a partir de
la entrada en el \texttt{.bib} puede llevar demasiado tiempo.

Para evitarlo, \texis\ permite, de manera f�cil, compilar un �nico
cap�tulo (o ap�ndice), que normalmente ser� aqu�l en el que se est� trabajando.

Para eso, simplemente hay que indicar qu� cap�tulo se quiere compilar
en el fichero \texttt{config.tex} utilizando el comando
\verb+\compilaCapitulo+\footnote{El comando s�lo puede invocarse una
  vez, por lo que no es v�lido si se quiere compilar un grupo
  determinado de cap�tulos.}. Si en vez de ser un cap�tulo lo que
queremos generar es un ap�ndice el procedimiento es el mismo, pero
utilizando el comando \verb+\compilaApendice+. Observa que \emph{no}
debe incluirse el nombre del directorio donde aparecen los ficheros
(es decir el ``\texttt{Capitulos}''), pues el propio comando lo hace
por nosotros.

Una vez que el cap�tulo se termina de escribir y se pasa al siguiente,
se querr� a�adir el \verb+\compilaCapitulo+ para el nuevo cap�tulo (y
anular el otro). En nuestro caso, en vez de eliminar el comando del
cap�tulo anterior, lo dejamos comentado por si es necesario en el
futuro. Es por ello que al final de la redacci�n del documento, se
tiene una l�nea por cada uno de los cap�tulos:

\begin{verbatim}
% Descomentar la l�nea para establecer el cap�tulo que queremos
% compilar

% \compilaCapitulo{01Introduccion}
% \compilaCapitulo{02EstructuraYGeneracion}
% \compilaCapitulo{03Edicion}
% \compilaCapitulo{04Imagenes}
% \compilaCapitulo{05Bibliografia}
% \compilaCapitulo{06Makefile}

% \compilaApendice{01AsiSeHizo}
\end{verbatim}

%-------------------------------------------------------------------
\section{Editores de \LaTeX\ y compilaci�n}
%-------------------------------------------------------------------
\label{cap3:sec:editores}

Existen numerosas alternativas para editar los ficheros de \LaTeX
\citep[ver][sec. 2.3]{Flynn05}, y
si has escrito ya alg�n art�culo, posiblemente ya tengas uno
``favorito''. Aunque el editor parezca poco importante (al fin y al
cabo lo importante es tu documento), en realidad pasar�s mucho tiempo
utiliz�ndolo, viendo sus colores, pulsando sus botones, y activando
sus teclas r�pidas.

Evidentemente \texis\ no obliga a utilizar ning�n editor en
concreto (faltar�a m�s), aunque es posible que necesites hacer algunos
cambios en los ficheros para que se adec�en a lo que espera el
editor. Esto es especialmente cierto si pretendes generar el documento
final utilizando alguna opci�n del editor.

En la secci�n~\ref{cap2:sec:compilacion} mostr�bamos c�mo compilar
todos los \texttt{.tex} desde la l�nea de comandos. Sin embargo,
reconocemos que esto no es lo m�s c�modo\footnote{\texis\ tiene un
  fichero \texttt{Makefile} para la compilaci�n en un �nico paso, que
  es explicado en el cap�tulo~\ref{cap:makefile}.}. Por lo tanto, si
el editor que tienes est� preparado para \LaTeX\ (no utilizas el Bloc
de notas...), es muy posible que tenga alg�n bot�n o tecla r�pida para
compilar el fichero abierto, ya sea con \texttt{latex} o
\texttt{pdflatex}.

Pues bien, en ese caso, debes comprobar c�mo funciona exactamente el
editor, ya que muy posiblemente, el fichero que estar�s editando
cuando quieras generar el documento no ser� el documento
\emph{maestro} (es decir, el que en la plantilla hemos llamado
\texttt{Tesis.tex}, y que contiene el punto de entrada e incluye todos
los dem�s). Por lo tanto, debes mirar de qu� manera puedes hacer que
el fichero que se env�a a \texttt{latex} sea el documento maestro.
Por ejemplo,
WinEdt\footnote{\url{http://www.winedt.com/}} permite crear
``proyectos'' donde se a�aden ficheros y se especifica cu�l es el
documento maestro; cuando se pulsa el bot�n de compilar,
independientemente del fichero activo en el editor, se manda compilar
el documento maestro.

Como se describe en la secci�n \ref{ap1:edicion}, nosotros utilizamos emacs \citep{emacsStallman} para crear nuestros ficheros \LaTeX. Como no pod�a ser de otro modo, \texis\ est� preparado para integrarse con �l, en particular
con el modo Auc\TeX\ que permite una edici�n c�moda de ficheros \TeX\ \citep{AtazLopezEmacs}.
En concreto, este modo dispone de una combinaci�n de teclas para
lanzar la generaci�n del documento final. En condiciones normales eso
implica enviar al programa \texttt{latex} el fichero que se est�
editando; sin embargo, en nuestro caso lo normal es que el fichero
\emph{maestro} que hay que utilizar no es el que se est� editando,
sino el fichero \texttt{Tesis.tex}. Para que funcione como queremos,
basta con a�adir al final de los ficheros \texttt{tex} unas
indicaciones para que Auc\TeX\ utilice ese fichero como fichero
maestro:

% ��Tendr� alguna repercusi�n que se meta en medio del fichero
% para el propio emacs!?
\begin{verbatim}
% Variable local para emacs, para  que encuentre el fichero
% maestro de compilaci�n y funcionen mejor algunas teclas
% r�pidas de AucTeX
%%%
%%% Local Variables:
%%% mode: latex
%%% TeX-master: "../Tesis.tex"
%%% End:
\end{verbatim}

Esta ``coletilla'' no es necesaria si utilizas cualquier otro editor.
Sin embargo \texis\ las tiene a�adidas en todos los ficheros (y
tambi�n en los ficheros de los cap�tulos y ap�ndices de este manual).
Las l�neas anteriores, adem�s, son utilizadas por otras combinaciones
de teclas de Auc\TeX, como las que permiten navegar por todas las
secciones del documento.

%-------------------------------------------------------------------
\section{Control de versiones}
%-------------------------------------------------------------------
\label{cap3:sec:control-versiones}

Como veremos en el cap�tulo~\ref{cap:makefile}, el fichero
\texttt{Makefile} contiene algunos objetivos para realizar copias de
seguridad de todos los ficheros del documento.

Sin embargo en el mundo de desarrollo software es habitual utilizar
sistemas de control de versiones. Estos sistemas gestionan las
distintas versiones por las que van pasando los ficheros durante todo
el proceso de desarrollo. La necesidad de estas herramientas est�
ampliamente reconocida, no s�lo porque sirven como medio de copia de
seguridad que permite \emph{volver hacia atr�s} ante alg�n fallo, sino
porque permite el trabajo simult�neo de dos o m�s
personas\footnote{Aunque esto en la redacci�n de una tesis no suele
  tener sentido, s� puede ser necesario en la elaboraci�n de manuales,
  cuadernillos de pr�cticas u otros documentos para los que
  \texis\ puede utilizarse.}.

Existen varias alternativas para el control de versiones, tanto
comerciales como bajo licencia \ac{GPL}
o similares. El sistema por excelencia dentro del software libre
fue durante muchos a�os \ac{CVS}
\citep{CVS}, aunque hoy por hoy ha sido desbancado por Subversion
\citep{Subversion}.  Entre las herramientas comerciales, destacan
SourceSafe de
Microsoft\footnote{\url{http://msdn.microsoft.com/ssafe/}},
Perforce\footnote{\url{http://www.perforce.com/}} y
AccuRev\footnote{\url{http://www.accurev.com/}}.

Aunque es una decisi�n que los autores del documento tendr�n que
tomar, aconsejamos el uso de uno de estos sistemas\footnote{En nuestro
  caso, utilizamos CVS para la escritura de las tesis, mientras que
  para la elaboraci�n de la plantilla (y manual), utilizamos
  Subversion.}. Una vez que se tiene configurada la m�quina servidora
que aloja el control de versiones (ver notas bibliogr�ficas), se suben
los ficheros \emph{fuente} del documento, que pasar�n a estar bajo el
control del servidor, lo que permitir� recuperar el estado del
documento en cualquier momento pasado (por lo que sirve tambi�n como
copia de seguridad).

Un punto importante es hacer que el sistema de control de versiones
\emph{ignore} los ficheros que son resultados de la generaci�n del
fichero final (el PDF). Cuando se compila el documento, \LaTeX\ genera
numerosos ficheros temporales (con extensiones como \texttt{.aux} o
\texttt{.bbl}) que \emph{no} deben subirse al sistema control de
versiones. Cuando se utiliza CVS se elimina el problema creando en los
directorios un fichero de texto llamado \texttt{.cvsignore} que
contiene todos los ficheros que deben ser ignorados. A pesar de que en
la elaboraci�n de la plantilla no utilizamos CVS, \texis\
incorpora esos ficheros para que puedan utilizarse en el proceso de
redacci�n de los documentos.

Si en vez de utilizar CVS est�s utilizando Subversion, puedes hacer
que �ste ignore los ficheros contenidos en el archivo
\texttt{.cvsignore} ejecutando la siguiente orden:

\begin{verbatim}
svn propset svn:ignore -F .cvsignore .
\end{verbatim}

\noindent en cada uno de los directorios que contengan el fichero. La
orden lo que hace es establecer la propiedad (\texttt{propset})
concreta para que el Subversion ignore (\texttt{svn:ignore}) los
ficheros que se indican en el fichero (\texttt{-F})
\texttt{.cvsignore}.

\com{
Para una versi�n futura de \texis, podr�amos incluir informaci�n
sobre la  revisi�n de la
plantilla en la p�gina posterior a la portada, igual que en el libro
del SVN. En ese caso, habr�a que contarlo aqu� para que los usuarios
puedan utilizarlo tambi�n.
}

%-------------------------------------------------------------------
\section*{\NotasBibliograficas}
%-------------------------------------------------------------------
\TocNotasBibliograficas

La idea de los dos modos de compilaci�n de la Tesis surgi� de forma
natural dada la experiencia en el proceso de desarrollo en C++, donde
los entornos integrados de desarrollo suelen proporcionar al menos
esas dos configuraciones posibles. La forma de hacerlo posible vino
despu�s de inspeccionar el c�digo \LaTeX\ del libro
\cite{ldesc2e}. La implementaci�n de los comandos no requiere un
conocimiento ni mucho menos extenso de las capacidades de \TeX; basta
con un poco de intuici�n al ver un ejemplo de \verb|\if|.

No obstante, el lector interesado en aprender \TeX\ a fondo puede
encontrar diversos manuales, como ``\TeX\ for the Impatient''
\citep{texImpatient}, aunque advertimos que se debe estar \emph{muy}
interesado para leerselo, ya que en condiciones normales no se
utilizar� nada de lo aprendido\footnote{A no ser que se quiera
  construir un paquete con una funcionalidad muy concreta...}. Tambi�n
se puede consultar \cite{texKnuth} o \cite{texByTopic}.

Con respecto a la utilizaci�n de control de versiones, dentro de las
opciones libres es muy utilizado el Subversion, cuyo libro de
referencia que ya se ha citado en el texto es
\cite{Subversion}. Para una descripci�n sencilla de c�mo instalar una
m�quina servidora puede consultarse \citet{LatexAndSVN} y
\citet{PaquetesSVNLatex}. En �ste �ltimo tambi�n aparece una somera
descripci�n de algunos paquetes de \LaTeX\ que pueden utilizarse para
incluir informaci�n relacionada directamente con las versiones de
Subversion. Aunque para m�s informaci�n al respecto recomendamos
\citet{SVNmulti} que dedica toda su atenci�n a la descripci�n de
\verb+svn-multi+, uno de los paquetes con m�s opciones disponibles
para ello.

%-------------------------------------------------------------------
\section*{\ProximoCapitulo}
%-------------------------------------------------------------------
\TocProximoCapitulo

En este cap�tulo hemos tratado los aspectos m�s importantes desde el
punto de vista de la edici�n de un documento realizado con \texis,
describiendo los comandos \LaTeX\ disponibles.

El pr�ximo cap�tulo aborda el tratamiento de las im�genes. Como se ver�,
soportar la generaci�n del documento
tanto con \texttt{latex} como \texttt{pdflatex} dificulta la gesti�n
de im�genes, pues cada uno utiliza un formato de fichero distinto. El
cap�tulo explica las distintas opciones que el usuario de \texis\
tiene para su manejo.

% Variable local para emacs, para  que encuentre el fichero maestro de
% compilaci�n y funcionen mejor algunas teclas r�pidas de AucTeX
%%%
%%% Local Variables:
%%% mode: latex
%%% TeX-master: "../ManualTeXiS.tex"
%%% End:

%%---------------------------------------------------------------------
%
%                          Parte 2
%
%---------------------------------------------------------------------
%
% Parte2.tex
% Copyright 2009 Marco Antonio Gomez-Martin, Pedro Pablo Gomez-Martin
%
% This file belongs to the TeXiS manual, a LaTeX template for writting
% Thesis and other documents. The complete last TeXiS package can
% be obtained from http://gaia.fdi.ucm.es/projects/texis/
%
% Although the TeXiS template itself is distributed under the 
% conditions of the LaTeX Project Public License
% (http://www.latex-project.org/lppl.txt), the manual content
% uses the CC-BY-SA license that stays that you are free:
%
%    - to share & to copy, distribute and transmit the work
%    - to remix and to adapt the work
%
% under the following conditions:
%
%    - Attribution: you must attribute the work in the manner
%      specified by the author or licensor (but not in any way that
%      suggests that they endorse you or your use of the work).
%    - Share Alike: if you alter, transform, or build upon this
%      work, you may distribute the resulting work only under the
%      same, similar or a compatible license.
%
% The complete license is available in
% http://creativecommons.org/licenses/by-sa/3.0/legalcode
%
%---------------------------------------------------------------------

% Definici�n de la segunda parte del manual

\partTitle{Conceptos avanzados}

\partDesc{Esta segunda parte del manual contiene cap�tulos que pueden
considerarse ``avanzados'', aunque cualquier documento a buen seguro
har� uso de los conceptos que en ellos se presentan.

Un primer cap�tulo explica la gesti�n de las im�genes que \texis\
espera que se utilice. El manual pasa despu�s a explicar c�mo a�adir
bibliograf�a y acr�nimos. Por �ltimo, se describe el fichero {\tt
Makefile} proporcionado, que ayuda en algunas de las tareas de
generaci�n de documentos.}

\makepart

%%---------------------------------------------------------------------
%
%                          Cap�tulo 4
%
%---------------------------------------------------------------------
%
% 04Imagenes.tex
% Copyright 2009 Marco Antonio Gomez-Martin, Pedro Pablo Gomez-Martin
%
% This file belongs to the TeXiS manual, a LaTeX template for writting
% Thesis and other documents. The complete last TeXiS package can
% be obtained from http://gaia.fdi.ucm.es/projects/texis/
%
% Although the TeXiS template itself is distributed under the 
% conditions of the LaTeX Project Public License
% (http://www.latex-project.org/lppl.txt), the manual content
% uses the CC-BY-SA license that stays that you are free:
%
%    - to share & to copy, distribute and transmit the work
%    - to remix and to adapt the work
%
% under the following conditions:
%
%    - Attribution: you must attribute the work in the manner
%      specified by the author or licensor (but not in any way that
%      suggests that they endorse you or your use of the work).
%    - Share Alike: if you alter, transform, or build upon this
%      work, you may distribute the resulting work only under the
%      same, similar or a compatible license.
%
% The complete license is available in
% http://creativecommons.org/licenses/by-sa/3.0/legalcode
%
%---------------------------------------------------------------------

\chapter{Gesti�n de las im�genes}
\label{cap4}
\label{cap:imagenes}


\begin{FraseCelebre}
\begin{Frase}
El alma nunca piensa sin una imagen mental.
\end{Frase}
\begin{Fuente}
Arist�teles
\end{Fuente}
\end{FraseCelebre}

\begin{resumen}
  Este cap�tulo describe todos los aspectos relacionados con las
  im�genes de los documentos. En particular, describe la estructura de
  directorios que \texis\ aconseja, as� como los aspectos
  relacionados con la diferencia entre los formatos esperados cuando
  se genera el documento final con \texttt{latex} y \texttt{pdflatex}.
\end{resumen}

%-------------------------------------------------------------------
\section{Introducci�n}
%-------------------------------------------------------------------
\label{cap4:sec:intro}

En este cap�tulo tratamos todos los aspectos relacionados con a�adir
im�genes al documento. Aunque en principio es algo bastante sencillo
(desde luego mucho m�s sencillo que a�adir una tabla compleja),
existen una serie de cosas a tener en cuenta que merecen un cap�tulo
entero en el manual.

En particular, lo que provoca que las im�genes requieran estas
explicaciones detalladas es el hecho de que, como ya dijimos en las
secciones~\ref{cap1:sec:que-es} y~\ref{cap2:sec:compilacion},
\texis\ te permite generar el documento utilizando tanto
\texttt{latex} como \texttt{pdflatex}.

Idealmente, el usuario final de \LaTeX\ no deber�a verse influenciado
por la aplicaci�n utilizada para generar sus ficheros. Sin embargo, en
cierto modo s� se ve afectado; no por el c�digo en s� contenido en los
\texttt{.tex} sino por los recursos a los que �stos hacen
referencia\footnote{En ciertas ocasiones tambi�n puede verse afectado
  el c�digo, si se utilizan paquetes que �nicamente funcionan con una
  de ellas.}. En concreto, si se utiliza \texttt{latex}, las im�genes
referenciadas con el comando \verb+\includegraphics+ se asume que
tienen formato \texttt{.eps}, mientras que en cuanto se utiliza
\texttt{pdflatex}, se admiten \texttt{.pdf}, \texttt{.png} y
\texttt{.jpg} (pero no \texttt{.eps}).

Por lo tanto, cuando utilizamos un c�digo como el siguiente (similar
al que hay en la portada para que aparezca el escudo):

\begin{verbatim}
\begin{figure}[t]
\begin{center}
\includegraphics[width=0.3\textwidth]%
                {Imagenes/Vectorial/escudoUCM}
\caption{Escudo de la Universidad Complutense}
\end{center}
\end{figure}
\end{verbatim}

\noindent cuando se genera con \texttt{latex}, se buscar� el fichero
\texttt{escudoUCM.eps} en el directorio \url{Imagenes/Vectorial},
mientras que al generarlo con \texttt{pdflatex}, se buscar� el fichero
en ese mismo directorio, con el mismo nombre, pero con extensi�n
\texttt{.pdf}, \texttt{.png} o \texttt{.jpg}.

Esto provoca que el programa utilizado para generar el documento final
es el que \emph{determina} qu� tipo de formato debe usarse para
almacenar las im�genes. Existen dos soluciones, que trataremos en las
secciones~\ref{cap4:formatoimagenes} y \ref{cap4:solTeXiS}. Antes de
eso, la siguiente secci�n explica la estructura de directorios que
\texis\ espera que se utilice.

%-------------------------------------------------------------------
\section{Gesti�n de im�genes}
%-------------------------------------------------------------------
\label{cap4:sec:gestion}

Los ficheros de im�genes pueden almacenarse donde el autor del
documento desee; al a�adir la referencia desde el \texttt{.tex},
deber� simplemente indicar la ruta correcta del archivo.

Sin embargo, \texis\ propone una estructura determinada que es la
que usa este documento. La estructura est� elegida de tal forma que
facilita la soluci�n del problema de la generaci�n utilizando tanto
\texttt{latex} como \texttt{pdflatex}, por lo que aunque pueda parecer
arbitraria, tiene cierto sentido.\comp{En realidad se puede discutir
  si tiene sentido la separaci�n entre vectorial y mapa de bits. La
  separaci�n tiene sentido para permitir, f�cilmente, cambiar el tipo
  de ficheros ``fuente'' que se utilizan. Si un usuario decide que
  quiere guardar los ficheros \texttt{eps} y convertirlos con los
  generadores respectivos al formato de \texttt{pdflatex}, es mejor
  tenerlos separados; de esta forma los de mapas de bits se
  transformar�n a \texttt{.png} y los vectoriales a \texttt{.pdf}.
  Teniendolos separados por tipo facilita esta tarea.}

La estructura que proponemos empieza con el directorio
\url{./Imagenes}, donde aparecen los siguientes directorios:

\begin{itemize}
\item \url{./Vectorial}: contiene los ficheros correspondientes a
  im�genes vectoriales.

\item \url{./Bitmap}: contiene los ficheros correspondientes a
  im�genes de mapas de bits.

\item \url{./Fuentes}: en este directorio aparecen los ``fuentes'' de
  las im�genes. As�, si se crean im�genes con Microsoft Visio, Power
  Point, o Corel, en este directorio ir�an los ficheros nativos de
  esos programas. Estos ficheros \emph{no} ser�n le�dos en el proceso
  de creaci�n del documento final.
\end{itemize}

Cada uno de los directorios anteriores, a su vez, contiene un
directorio por cap�tulo. De esta forma es f�cil encontrar los ficheros
si se quieren modificar. En el directorio ``ra�z'' se encuentran las
im�genes que no pertenecen a ning�n cap�tulo, como la del escudo de la
portada. Tambi�n pueden aparecer otras im�genes que se utilicen en
otras partes del documento que no sean los cap�tulos. Por ejemplo,
\texis\ proporciona una imagen que puede ser de utilidad, y que
est� colocada en ese directorio raiz por ser independiente del
cap�tulo. Es una figura ``dummy'' que sirve para marcar el lugar en el
que deber�a aparecer una figura o gr�fico que a�n est� por hacer
(figura~\ref{fig:todo}).

\figura{Vectorial/Todo}{width=.5\textwidth}{fig:todo}%
       {Figura utilizada para marcar una imagen por hacer.}

\texis\ incluye el comando \verb+\figura+ para facilitar la
inclusi�n de im�genes. En particular, el comando tiene cuatro
par�metros: el nombre del fichero (en el que no hay que indicar el
directorio \url{./Imagenes}), los argumentos pasados al
\verb+\includegraphics+ que suele tener informaci�n sobre el tama�o
deseado, la etiqueta con la que luego podr� referenciarse la
ilustraci�n, y por �ltimo el t�tulo que aparecer� en la parte
inferior.

Para incluir la figura~\ref{fig:todo} por lo tanto, el comando
es\footnote{Como se ha mencionado, observa que el primer par�metro
  donde se indica el nombre del fichero \emph{no incluye} ni el nombre
  del directorio \url{Imagenes} ni la extensi�n del fichero.}:

\begin{verbatim}
\figura{Vectorial/Todo}{width=.5\textwidth}{fig:todo}%
       {Figura utilizada para marcar una imagen por hacer.}
\end{verbatim}

que gracias al \verb+{fig:todo}+, luego puede citarse en el c�digo
\LaTeX\ con:

\begin{example}
La figura~\ref{fig:todo}
muestra\ldots
\end{example}

La figura se a�ade autom�ticamente al �ndice de figuras que aparece al
principio del documento, en el que se indica el n�mero de la figura,
el texto inferior y la p�gina en la que aparece. Es posible que el
texto sea lo suficientemente largo como para que ocupe m�s de una
l�nea en la entrada en el �ndice. Si se desea utilizar un texto m�s
corto para evitarlo, se puede utilizar el comando \verb+\figuraEx+ que
recibe un par�metro m�s con el ``t�tulo corto'' o lo que es lo mismo,
con el texto alternativo que aparecer� en el �ndice.

%-------------------------------------------------------------------
\section{Formato de las im�genes}
%-------------------------------------------------------------------
\label{cap4:formatoimagenes}

Recuperamos en esta secci�n el problema anteriormente comentado sobre
los formatos de las im�genes. Como ya dijimos en la secci�n de
introducci�n, el uso de \texttt{latex} o \texttt{pdflatex} determina
los formatos de los ficheros que deben utilizarse para las im�genes,
seg�n la tabla~\ref{cap4:tab:formatoImagenes}.

\begin{table}[t]
\centering
\begin{tabular}{|c|c|c|}
\hline
Programa & Mapa de bits & Vectoriales \\
\hline
\texttt{latex} & \texttt{.eps} & \texttt{.eps} \\
\hline
\texttt{pdflatex} & \texttt{.png} | \texttt{.jpg} & \texttt{.pdf} \\
\hline
\end{tabular}
\caption{Formatos de im�genes para \texttt{latex} y \texttt{pdflatex}
   \label{cap4:tab:formatoImagenes}}
\end{table}

Esto significa que, en principio, en el momento de a�adir la primera
imagen al documento, se debe decidir qu� programa se utilizar� para
generarlo, y utilizar el formato de imagen adecuado a �l. Si tienes
claro qu� programa utilizar�s, la soluci�n es as� de simple. Almacena
las im�genes en el formato adecuado seg�n la
tabla~\ref{cap4:tab:formatoImagenes}.

Desgraciadamente, lo habitual es no encontrarse en esa situaci�n.
Normalmente cuando se comienza a escribir, es muy dif�cil pronosticar
cu�l de los dos se utilizar�, y por lo tanto no quieres decantarte por
ninguno. No est�s seguro de cu�l quieres, o puede que quieras poder
generarlo de las dos formas, debido a alguna restricci�n del servicio
de publicaciones.

Por lo tanto, la mejor soluci�n es, simplemente, permitir ambas
alternativas. Para eso lo m�s f�cil es \emph{duplicar} las im�genes,
es decir, mantener tanto la copia que ser� le�da por \texttt{latex}
como la que utilizar� \texttt{pdflatex}.

Esta duplicaci�n, no obstante, no suele ser aconsejable, pues (adem�s
de consumir m�s espacio) es propensa a errores: si hay que cambiar una
imagen, lo habitual ser� abrir el fichero de \url{./Imagenes/Fuentes},
y luego ``exportarlo'' al formato nativo. En ese momento, es f�cil
olvidar generar \emph{los dos} ficheros.

Por lo tanto, hay dos soluciones r�pidas y f�ciles:

\begin{itemize}
\item Decidir al principio qu� programa se utilizar� y utilizar
  siempre los formatos que �ste espera, seg�n la
  tabla~\ref{cap4:tab:formatoImagenes}. Tiene la desventaja de que no
  se podr� (f�cilmente) cambiar el programa generador, pues se
  necesitar� crear las im�genes en el formato esperado por la nueva
  aplicaci�n.

\item No atarse al uso de ninguno de los dos, y duplicar los ficheros
  de forma que todas las im�genes se guardan dos veces, en cada uno de
  los formatos esperados por ambos programas. Su desventaja es la
  duplicaci�n de los ficheros, con los problemas de coherencia que eso
  puede provocar.
\end{itemize}

Ninguna de las dos alternativas es �ptima, por lo que \texis\
proporciona la soluci�n alternativa descrita en la secci�n siguiente.
Si decides utilizar alguna de las opciones f�ciles anteriores, puedes
omitir la lectura de la misma.

%-------------------------------------------------------------------
\section{Im�genes independientes del programa generador}
%-------------------------------------------------------------------
\label{cap4:solTeXiS}

En general conviene evitar duplicar los datos almacenados en disco
para evitar problemas de incoherencias. Por eso, cuando no se quiere
limitar la generaci�n del documento final a s�lo uno de los dos
programas, \texttt{latex} o \texttt{pdflatex}, \texis\ desaconseja
almacenar en disco cada imagen en los dos formatos exigidos por ellas.

\texis\ est� preparada para que se almacene en el directorio de
las im�genes �nicamente las soportadas por \texttt{pdflatex} (es
decir, ficheros \texttt{.pdf} para im�genes vectoriales y
\texttt{.png} o \texttt{.jpg} para mapas de bits). Obviamente, si se
hace as�, al utilizar \texttt{latex} para generar, dar� error al no
encontrar los ficheros de im�genes correspondientes. Y aqu� es donde
entra en acci�n el fichero \texttt{Makefile} incluido (que se explica
ampliamente en el cap�tulo~\ref{cap:makefile}) cuando se ejecuta con
el objetivo \texttt{latex}:

\begin{verbatim}
$ make latex
\end{verbatim}

\noindent antes de invocar a \texttt{latex}, convierte todos los
ficheros \texttt{.pdf} que hay en el directorio
\url{./Imagenes/Vectorial} a ficheros \texttt{.eps}, y todos los
\texttt{.jpg}  y \texttt{.png} de \url{./Imagenes/Bitmap} a
\texttt{.eps}, para que \texttt{latex} los encuentre.

Para realizar la conversi�n, se utilizan las aplicaciones
\texttt{pdftops} y \texttt{sam2p} que deben estar
accesibles en el PATH. Esa es la raz�n por la que, como mencion�bamos
en la secci�n~\ref{cap1:sec:que-es}, \texis\ anima al uso de
sistemas Linux: las aplicaciones anteriores est�n disponibles en este
sistema operativo (aunque puede que no se instalen directamente en
algunas distribuciones), mientras que en Windows normalmente no est�n.

%-------------------------------------------------------------------
\section{Gesti�n de im�genes y control de versiones}
%-------------------------------------------------------------------
\label{cap4:variaciones}

La soluci�n propuesta en la secci�n anterior hace que la plantilla
espere que dentro del directorio \url{./Imagenes/Vectorial} aparezcan
ficheros \texttt{.pdf} y en \url{./Imagenes/Bitmap} se encuentren
ficheros con extensi�n \texttt{.png} o \texttt{.jpg}.

En el proceso de generaci�n cuando se utiliza \texttt{latex}, se
convierten todos esos ficheros a \texttt{.eps} para que el programa
encuentre las im�genes en el formato que �ste espera, por lo que en
los directorios anteriores aparecer�n ficheros \texttt{.eps}
\emph{generados autom�ticamente}.

\texis\ est� configurado para que, en caso de utilizar un sistema
de control de versiones, �ste \emph{ignore} esos ficheros generados
(ver una explicaci�n detallada en la
secci�n~\ref{cap3:sec:control-versiones}). De esta forma, el usuario
no es ``molestado'' en los momentos de las actualizaciones con
mensajes indicando que hay nuevos ficheros en el directorio de las
im�genes que no han sido subidos al servidor.

Sin embargo, esta caracter�stica debe \emph{anularse} si se utiliza
una soluci�n distinta a la indicada en el apartado anterior. En
particular, se debe \emph{eliminar} el fichero
\texttt{.cvsignore}\footnote{O no establecer la propiedad
  \texttt{svn:ignore} con �l si se utiliza Subversion.} si:

\begin{itemize}
\item Se decide al principio de la redacci�n del documento que se va a
  utilizar \texttt{latex} para su generaci�n (y nunca
  \texttt{pdflatex}), y por lo tanto se usar�n siempre ficheros con
extensi�n \texttt{.eps}.

\item Se decide no utilizar la caracter�stica de conversi�n autom�tica
  de im�genes, y se duplican los ficheros, guardando siempre tanto la
  copia le�da por \texttt{latex} como por \texttt{pdflatex}.

\item Se decide que las im�genes se guardar�n siempre en \texttt{.eps}
  y se cambia el \texttt{Makefile} para que, en caso de utilizar
  \texttt{pdflatex} las convierta al formato utilizado por �ste
  (consulta la secci�n~\ref{cap6:subsec:compilacionImagenes} si est�s
  en este caso).
\end{itemize}

%-------------------------------------------------------------------
\section{Im�genes divididas}
%-------------------------------------------------------------------
\label{cap4:subfloat}

Somos conscientes de que esta secci�n incumple lo que comentamos al
principio del manual de lo que \texis\ \emph{no era}. Dec�amos en
la secci�n~\ref{cap1:sec:que-no-es} que esto \emph{no} era un manual
de \LaTeX, y lo seguimos manteniendo. Sin embargo, en este apartado
incumplimos moment�neamente esa promesa para explicar brevemente c�mo
incluir varias figuras dentro de un entorno flotante (t�picamente otra
figura).

En este manual ya ha aparecido un ejemplo. La
figura~\ref{cap2:fig:pdf} de la p�gina \pageref{cap2:fig:pdf} mostraba
en realidad dos capturas distintas, cada una de ellas con un subt�tulo
distinto.

El c�digo \LaTeX\ de esa figura era:

\begin{verbatim}
\begin{figure}[t]
  \centering
  %
  \subfloat[][Propiedades del documento]{
     \includegraphics[width=0.42\textwidth]%
                     {Imagenes/Bitmap/02/PropiedadesPDF}
     \label{cap2:fig:PropiedadesPDF}
  }
  \qquad
  \subfloat[][Tabla de contenidos]{
     \includegraphics[width=0.42\textwidth]%
                     {Imagenes/Bitmap/02/IndicePDF}
     \label{cap2:fig:TocPDF}
  }
 \caption{Capturas del visor de PDF\label{cap2:fig:pdf}}
\end{figure}
\end{verbatim}


La idea general es crear un entorno figura tradicional, pero no poner
en ella directamente el \verb+\includegraphics+, sino subdividir ese
entorno figura en varias partes. A cada una de ellas se le da una
etiqueta diferente para poder referenciarlas, una descripci�n,
etc�tera.  Un esquema general ser�a:

\begin{verbatim}
\begin{figure}[t]
  \centering
  %
  \subfloat[<ParaElIndice1>][<Caption1>]{ 
     % Contenido para este "subelemento" (podr� ser una
     % figura, una tabla, o cualquier otra cosa).
     \includegraphics[width=5cm]{ficheroSinExtension}
     \label{fig:etiqueta1}
  }
  \subfloat[<ParaElIndice2>][<Caption2>]{
     % Contenido para este "subelemento" (podr� ser una
     % figura, una tabla, o cualquier otra cosa).
     \includegraphics[width=5cm]{ficheroSinExtension}
     \label{fig:etiqueta2}
  }
 \caption{Descripci�n global para la figura}
 \label{Etiqueta para toda la figura}
\end{figure}
\end{verbatim}

El sistema autom�ticamente decide cu�ndo poner la siguiente figura al
lado, o en otra l�nea, en base a si entra o no. Es posible forzar a
que se ponga en una l�nea nueva si se deja una l�nea en blanco en el
\texttt{.tex}. Esto tiene la repercusi�n de que no deber�as dejar
l�neas en blanco en ning�n momento dentro del entorno flotante, para
evitar que se ``salte de linea'' en las figuras. Si quieres por
legilibidad dejar alguna l�nea, pon un comentario vacio (como hemos
hecho con el ejemplo anterior despu�s del \verb+\centering+).

La separaci�n entre dos figuras que se colocan en la misma fila puede
ser demasiado peque�a. Para separarlas un poco m�s, puedes poner
\verb+\qquad+ entre el cierre llaves de un \verb+\subfloat+ y el
siguiente. Ese era el cometido del \verb+\qquad+ que aparec�a en el
ejemplo de las figuras visto anteriormente.

Por otro lado, el \verb+\subfloat+ tiene dos ``par�metros'', que se
colocan entre corchetes justo despu�s. En realidad ambos son
opcionales. Podr�amos poner directamente:

\begin{verbatim}
\subfloat{ <comandos para el subelemento> }
\end{verbatim}

pero en ese caso no se etiquetar� con una letra.

El texto que se pone entre los primeros corchetes se utiliza para el
�ndice de figuras. En teor�a, se mostrar� en dicho �ndice primero la
descripci�n global de la figura, y luego la de cada subelemento. Si
no quieres que ocurra, deja en blanco el contenido del primer
corchete. En la pr�ctica, el �ndice de figuras no tiene ``niveles'',
por lo que si se ponde la descripci�n de la subfigura, �sta no
aparecer� en el mismo.

El segundo corchete recibe el texto con la descripci�n del
subelemento, es decir lo que aparecer� junto a la letra
identificativa. Si lo dejas vac�o (pero poniendo los corchetes),
saldr� la letra, sin texto. Si ni siquieras pones los corchetes, no
saldr� tampoco la letra.

Por �ltimo, debes saber que puedes referenciar de forma independiente
cada uno de los subelementos. Para eso, basta con
\verb+\ref{etiqueta}+, siendo la etiqueta una definida mediante
\verb+\label+ \emph{dentro} del \verb+\subfloat+. Se puede entonces
referenciar utilizando el comando \verb+\ref+ tradicional (que har�
que aparezca la letra correspondiente junto con el n�mero de figura),
o el comando \verb+\subref+:

\begin{example}
La figura~\ref{cap2:fig:pdf}
tiene dos partes. La
parte izquierda es la figura~
\ref{cap2:fig:PropiedadesPDF}
y a la derecha est� la
\subref{cap2:fig:TocPDF}.
\end{example}

Como ya hemos dicho, se pueden poner varias ``filas'' de im�genes en
la misma figura. Se puede forzar este comportamiento a�adiendo l�neas
en blanco dentro del entorno \verb+\figure+ (lo que ser� interpretado
por \LaTeX\ como un nuevo p�rrafo), aunque tambi�n se a�adir�n
autom�ticamente si \LaTeX\ detecta que no entra todo en una �nica
l�nea. La figura~\ref{fig:cap4:javy1} \citep[extra�da
de][]{GomezMartinMA2008PhD} es un ejemplo de esto. Como se ve en su
c�digo \LaTeX\ que se muestra a continuaci�n, cada una de las im�genes
ocupa el 45\% del ancho de la p�gina, por lo que �nicamente entran dos
figuras por fila. A pesar de que no existe ninguna l�nea en blanco en
el c�digo, las im�genes se colocan en dos filas distintas.

\begin{figure}[t]
  \centering
%
  \begin{SubFloat}
    {\label{fig:cap4:barrioclases}%
     Estudiante y Javy dirigi�ndose al barrio de clases}%
    \includegraphics[width=0.45\textwidth]%
                    {Imagenes/Bitmap/04/Javy1BarrioClases}%
  \end{SubFloat}
\qquad
  \begin{SubFloat}
    {\label{fig:cap4:framauro}%
     Estudiante enfrente de Framauro}%
    \includegraphics[width=0.45\textwidth]%
                    {Imagenes/Bitmap/04/Javy1Framauro}%
  \end{SubFloat}
  % La siguiente no entra; ira en otra 'linea'
  \begin{SubFloat}
    {\label{fig:cap4:pilaops}%
     Estudiante interactuando con la pila de operandos}%
    \includegraphics[width=0.45\textwidth]%
                    {Imagenes/Bitmap/04/Javy1PilaOperandos}%
  \end{SubFloat}
\qquad
  \begin{SubFloat}
    {\label{fig:cap4:estudianteyjavy}%
     Estudiante hablando con Javy}%
    \includegraphics[width=0.45\textwidth]%
                    {Imagenes/Bitmap/04/Javy1Javy}%
  \end{SubFloat}
\caption{Ejemplo de uso de \texttt{subfloat}.%
         \label{fig:cap4:javy1}}
\end{figure}


\begin{verbatim}
\begin{figure}[t]
  \centering
%
  \begin{SubFloat}
    {\label{fig:cap4:barrioclases}%
     Estudiante y Javy dirigi�ndose al barrio de clases}%
    \includegraphics[width=0.45\textwidth]%
                    {Imagenes/Bitmap/04/Javy1BarrioClases}%
  \end{SubFloat}
\qquad
  \begin{SubFloat}
    {\label{fig:cap4:framauro}%
     Estudiante enfrente de Framauro}%
    \includegraphics[width=0.45\textwidth]%
                    {Imagenes/Bitmap/04/Javy1Framauro}%
  \end{SubFloat}
  % La siguiente no entra; ira en otra 'linea'
  \begin{SubFloat}
    {\label{fig:cap4:pilaops}%
     Estudiante interactuando con la pila de operandos}%
    \includegraphics[width=0.45\textwidth]%
                    {Imagenes/Bitmap/04/Javy1PilaOperandos}%
  \end{SubFloat}
\qquad
  \begin{SubFloat}
    {\label{fig:cap4:estudianteyjavy}%
     Estudiante hablando con Javy}%
    \includegraphics[width=0.45\textwidth]%
                    {Imagenes/Bitmap/04/Javy1Javy}%
  \end{SubFloat}
\caption{Ejemplo de uso de \texttt{subfloat}.%
         \label{fig:cap4:javy1}}
\end{figure}
\end{verbatim}

%-------------------------------------------------------------------
\section*{\NotasBibliograficas}
%-------------------------------------------------------------------
\TocNotasBibliograficas

En este cap�tulo hemos descrito c�mo se gestionan las im�genes en
\texis\, por lo que no existe ninguna fuente relacionada adicional
de consulta.

Conviene, eso s�, indicar que las explicaciones al respecto del
entorno \verb+\subfloat+ dadas en la secci�n~\ref{cap4:subfloat}
distan mucho de estar completas, aunque es cierto que deber�an ser
suficientes para la mayor�a de los casos. El entorno, no obstante,
permite hacer muchas m�s cosas. Se puede consultar el manual oficial
para m�s informaci�n\footnote{Disponible en
  \url{ftp://tug.ctan.org/pub/tex-archive/macros/latex/contrib/subfig/subfig.pdf}.}.

%-------------------------------------------------------------------
\section*{\ProximoCapitulo}
%-------------------------------------------------------------------
\TocProximoCapitulo

El pr�ximo cap�tulo pasa a describir algunos aspectos sobre la
bibliograf�a. En concreto, veremos que \texis\ define un estilo de
bibliograf�a propio que, si bien no introduce demasiadas diferencias
con respecto al habitual, permite a�adir alg�n campo nuevo a las
citas, como por ejemplo la direcci�n Web y la fecha de la �ltima vez
que se visit� (o comprob� su existencia).

% Variable local para emacs, para  que encuentre el fichero maestro de
% compilaci�n y funcionen mejor algunas teclas r�pidas de AucTeX
%%%
%%% Local Variables:
%%% mode: latex
%%% TeX-master: "../ManualTeXiS.tex"
%%% End:

%%---------------------------------------------------------------------
%
%                          Cap�tulo 5
%
%---------------------------------------------------------------------
%
% 05Bibliografia.tex
% Copyright 2009 Marco Antonio Gomez-Martin, Pedro Pablo Gomez-Martin
%
% This file belongs to the TeXiS manual, a LaTeX template for writting
% Thesis and other documents. The complete last TeXiS package can
% be obtained from http://gaia.fdi.ucm.es/projects/texis/
%
% Although the TeXiS template itself is distributed under the 
% conditions of the LaTeX Project Public License
% (http://www.latex-project.org/lppl.txt), the manual content
% uses the CC-BY-SA license that stays that you are free:
%
%    - to share & to copy, distribute and transmit the work
%    - to remix and to adapt the work
%
% under the following conditions:
%
%    - Attribution: you must attribute the work in the manner
%      specified by the author or licensor (but not in any way that
%      suggests that they endorse you or your use of the work).
%    - Share Alike: if you alter, transform, or build upon this
%      work, you may distribute the resulting work only under the
%      same, similar or a compatible license.
%
% The complete license is available in
% http://creativecommons.org/licenses/by-sa/3.0/legalcode
%
%---------------------------------------------------------------------

\chapter{Bibliograf�a y acr�nimos}
\label{cap5}
\label{cap:bibliografia}


\begin{FraseCelebre}
\begin{Frase}
Como un ganso desplumado y escu�lido,\\
me preguntaba a m� mismo con voz indecisa\\
si de todo lo que estaba leyendo\\
har�a el menor uso alguna vez en la vida.
\end{Frase}
\begin{Fuente}
James Clerk Maxwell, sobre su educaci�n en Cambridge
\end{Fuente}
\end{FraseCelebre}

\begin{resumen}
Este cap�tulo aclara algunas cosas sobre la bibliograf�a
utilizada en \texis, y sobre la infraestructura para
la creaci�n de una lista de acr�nimos.
\end{resumen}

%-------------------------------------------------------------------
\section{Bibliograf�a}
%-------------------------------------------------------------------
\label{cap5:sec:intro}

Para hacer la bibliograf�a del documento, \texis\ hace uso, como no
pod�a ser de otra forma, de Bib\TeX. Esto permite una generaci�n
bastante sencilla de la misma, utilizando las entradas \verb+@bibitem+
correspondientes. El fichero \texttt{makefile} explicado en el
cap�tulo siguiente se encarga de invocar a \texttt{bibtex}, la
aplicaci�n responsable de la correcta creaci�n de la lista de
referencias. Si utilizas cualquier otro sistema para generar el
documento final (como por ejemplo las proporcionadas por el editor
\LaTeX\ que est�s utilizando), deber�s encargarte de averiguar c�mo
debes hacerlo.

La secci�n siguiente localiza el fichero m�s importante para la
construcci�n de la bibliograf�a, e indica d�nde cambiar los ficheros
\texttt{.bib} utilizados.

Por otro lado, para a�adir en el texto las referencias,
\texis\ (y este manual) hacen uso del formato utilizado por el
paquete \texttt{natbib} que, como se ha podido comprobar a lo largo de
estas p�ginas, se basa en indicar el nombre del autor y el a�o de la
publicaci�n en el propio texto. La secci�n~\ref{cap5:sec:natbib}
explica brevemente las capacidades del paquete.

\texis\ modifica ligeramente el formato de salida de cada
una de las referencias para adecuarlas m�s a nuestros gustos. Tambi�n
hemos a�adido algunas capacidades m�s que pueden a�adirse a las
mismas. La secci�n~\ref{cap5:sec:bibitem} las describe.

%El cap�tulo
La parte relativa a la bibliograf�a 
termina con una breve secci�n �til si se quiere cambiar el
tipo de bibliograf�a a utilizar.

%- - - - - - - - - - - - - - - - - - - - - - - - - - - - - - - - - -
\subsection{Ficheros involucrados}
%- - - - - - - - - - - - - - - - - - - - - - - - - - - - - - - - - -
\label{cap5:sec:ficheros}

El fichero responsable de la generaci�n de la bibliograf�a en \texis\
es \texttt{Cascaras/bibliografia.tex}.  Es el responsable de crear el
cap�tulo sin numerar, poner la cabecera especial para que en la parte
superior de todas sus p�ginas aparezca el texto
``\textsc{Bibliograf�a}'', etc. Para eso, al final del fichero aparece
la invocaci�n al comando \verb|\makeBib| definido en
\texttt{TeXiS/TeXiS\_bib.tex}.

Ya dijimos en la secci�n~\ref{cap3:ssec:frases} que para cambiar la
cita c�lebre que aparece en el cap�tulo sin numerar de la bibliograf�a
debemos editar el fichero \texttt{Cascaras/bibliografia.tex}. En
ese mismo fichero es donde se configuran los archivos donde se deben
buscar los \verb+@bibitem+ que se referencian en el texto. Para
hacerlo, se utiliza el comando \verb|\setBibFiles|:

\begin{verbatim}
\setBibFiles{%
nuestros,latex,otros%
}
\end{verbatim}

Como se puede ver, en este manual las referencias est�n organizadas en
tres ficheros distintos: referencias a trabajos propios
(\texttt{nuestros.bib}), referencias relacionadas con \LaTeX\
(\texttt{latex.bib}) y referencias varias que no entran en ninguna de
las dos anteriores (\texttt{otros.bib}).

Esos tres ficheros aparecen en el directorio ra�z del manual y
pueden/deben ser sustituidos por los utilizados en el nuevo documento.

%- - - - - - - - - - - - - - - - - - - - - - - - - - - - - - - - - -
\subsection{Referencias con \texttt{natbib}}
%- - - - - - - - - - - - - - - - - - - - - - - - - - - - - - - - - -
\label{cap5:sec:natbib}

Aunque en este manual no se haya hecho un uso demasiado extenso de la
bibliograf�a, es cierto que \texttt{natbib} proporciona opciones muy
interesantes para utilizarse en textos donde las referencias tengan un
peso importante (o al menos m�s importante que en este manual).

Cuando se utiliza \texttt{natbib}\footnote{El paquete es incluido por
  \texis\ en \texttt{TeXiS/TeXiS\_pream.tex}.} en vez de hacer uso de
\verb+\cite+, se pueden utilizar otras dos versiones distintas del
comando, en concreto \verb+\citet+ y \verb+\citep+. El primero est�
pensado para hacer referencias en el propio \texttt{t}exto y el �ltimo
para que las referencias aparezcan entre \texttt{p}ar�ntesis.

En vez de hacer una descripci�n de cada una de las variaciones, la
tabla~\ref{cap5:tab:natbib} contiene las distintas posibilidades. Las
opciones est�n cogidas directamente de la documentaci�n del paquete y
asume que existe un \texttt{@bibitem} con nombre \texttt{key} que
describe una referencia escrita por los autores Jones, Baker y Smith
en 1990. Notar que el resultado contiene la conjunci�n espa�ola ``y''
en vez de la inglesa ``and'' para separar el �ltimo autor.

\begin{table}[t]
\centering
\begin{tabular}{|c|c|}
\hline
Comando & Resultado \\
\hline
\hline
\verb+\citet{key}+ & Jones et al. (1990) \\
\hline
\verb+\citet*{key}+ & Jones, Baker y Smith (1990) \\
\hline
\verb+\citep{key}+ & (Jones et al., 1990) \\
\hline
\verb+\citep*{key}+ & (Jones, Baker y Smith, 1990) \\
\hline
\verb+\citep[cap. 2]{key}+ & (Jones et al., 1990, cap. 2) \\
\hline
\verb+\citep[e.g.][]{key}+ & (e.g. Jones et al., 1990) \\
\hline
\verb+\citep[e.g.][p. 32]{key}+ & (e.g. Jones et al., p. 32) \\
\hline
\verb+\citeauthor{key}+ & Jones et al. \\
\hline
\verb+\citeauthor*{key}+ & Jones, Baker y Smith \\
\hline
\verb+\citeyear{key}+ & 1990 \\
\hline
\end{tabular}
\caption{Distintas opciones de referencias con \texttt{natbib}%
         \label{cap5:tab:natbib}}
\end{table}

%- - - - - - - - - - - - - - - - - - - - - - - - - - - - - - - - - -
\subsection{Modificaciones en los \texttt{@bibitem}}
%- - - - - - - - - - - - - - - - - - - - - - - - - - - - - - - - - -
\label{cap5:sec:bibitem}

En \texis\ hemos definido nuestro propio estilo de bibliograf�a,
es decir, el formato de salida de las referencias en el listado que
aparece al final del documento. No es demasiado distinto del estilo
utilizado por defecto, pero s� tiene ligeras modificaciones.

Las modificaciones m�s obvias es que se utiliza la versi�n espa�ola
del formato, para que aparezca ``editor'', ``p�ginas'' o ``Informe
T�cnico'' entre otros.

Tambi�n hemos a�adido ``constantes'' (o, en nomenclatura de Bib\TeX,
``macros'') para una editorial y dos series muy utilizadas en nuestra
�rea, ``Springer-Verlag'' (\texttt{SV}), ``Lecture Notes in Computer
Science'' (\texttt{LNCS}) y ``Lecture Notes in Artificial Intelligence
(subserie de LNCS)'' (\texttt{LNAI}). De esta forma, una entrada de la
bibliograf�a puede ser:

\begin{verbatim}
@inproceedings{Ejemplo,
   author = { ... },
   title = { ... },
   ...
   publisher = SV,
   series = LNCS,
   ...
   year = 2009,
   month = jan
}
\end{verbatim}

Como se puede ver se han hecho uso de las macros para indicar la
editorial y la serie. Tambi�n se ha utilizado la macro para indicar el
mes (\texttt{jan} equivale a enero), de forma que la entrada sea
independiente del lenguaje utilizado posteriormente en la referencia.

\medskip

Mucho m�s interesante es la creaci�n de \emph{dos nuevos campos} que
pueden a�adirse en las entradas de Bib\TeX, y que permiten indicar la
p�gina Web en la que se puede encontrar la referencia.

En concreto, admitimos un nuevo campo, \verb+webpage+ que permite
indicar una p�gina Web, y un campo \verb+lastaccess+ permite indicar
la fecha del �ltimo acceso. De esta forma, la referencia, adem�s de
los campos e informaci�n habituales, incluye la localizaci�n.

Por ejemplo, el libro del Subversion citado en un cap�tulo anterior
\citep{Subversion} est� disponible en la Web; para que la URL o
direcci�n Web aparezca en la referencia, se pueden utilizar los nuevos
campos:

\begin{verbatim}
@Book{Subversion,
  author      = {Ben Collins-Sussman and
                 Brian W. Fitzpatrick and
                 C. Michael Pilato},
  title       = {Version Control with Subversion},
  publisher   = {O'Reilly},
  year        = 2004,
  isbn        = {0-596-00448-6},
  webpage     = {http://svnbook.red-bean.com/},
  lastaccess  = {Octubre, 2009}
}
\end{verbatim}

El estilo a�ade la cadena ``Disponible en'' antes de la url (que se
a�ade con el comando \verb+\url+ para su correcta divisi�n en l�neas)
y entre par�ntesis incluye la fecha del �ltimo acceso. El aspecto
final de la referencia del libro anterior es:

\medskip

\noindent{\sc Collins-Sussman, B.}, {\sc Fitzpatrick, B.~W.} y {\sc Pilato, C.~M.}
\newblock {\em Version Control with Subversion\/}.
\newblock O'Reilly, 2004.
\newblock ISBN 0-596-00448-6.
\newblock Disponible en \url{http://svnbook.red-bean.com/} (�ltimo acceso,
  Octubre, 2009).

\medskip

Tambi�n hemos a�adido un nuevo tipo de entrada para referenciar
art�culos de la Wikipedia. Puede ser discutible si es adecuado o no
utilizar citas de la Wikipedia en documentos acad�micos como tesis o
trabajos de fin de master, pero si eres de los que opinan que debe
citarse todo lo que uno utilice y utilizas la Wikipedia, puede que
quieras usar esta nueva entrada que hay en \texis. Por ejemplo, la
entrada de \LaTeX\ de la Wikipedia \citep{LaTeXWikipedia} se define
con:

\begin{verbatim}
@Wikipedia{LaTeXWikipedia,
   author     = {Wikipedia},
   title      = {{LaTeX}},
   wpentry    = {LaTeX},
   language   = {es},
   webpage    = {http://es.wikipedia.org/wiki/LaTeX},
   lastaccess = {Mayo, 2009},
}
\end{verbatim}

En el texto la referencia no aparece con el a�o sino que hace
alusi�n a que es una entrada de la Wikipedia, como se ha podido ver en
el parrafo anterior. El resultado final en la lista de referencias es:

\medskip

\noindent{\sc Wikipedia} (LaTeX).
\newblock Entrada: ``{LaTeX}''.
\newblock Disponible en \url{http://es.wikipedia.org/wiki/LaTeX} (�ltimo
  acceso,  Mayo, 2009).

%- - - - - - - - - - - - - - - - - - - - - - - - - - - - - - - - - -
\subsection{Cambio del estilo de la bibliograf�a}
%- - - - - - - - - - - - - - - - - - - - - - - - - - - - - - - - - -
\label{cap5:sec:cambios}

El estilo de las referencias de \texis\ est� definido en
\texttt{TeXiS/TeXiS.bst}, que se incluye en la parte final del fichero
\texttt{TeXiS\_bib.tex}:

\begin{verbatim}
...
\bibliographystyle{TeXiS/TeXiS}
...
\end{verbatim}

Si quieres utilizar un estilo distinto de los disponibles en \LaTeX\
basta con que cambies esa l�nea para definir ese otro estilo. Por
ejemplo:

\begin{verbatim}
...
\bibliographystyle{abbrv}
...
\end{verbatim}

\noindent configura la bibliograf�a para que queden numeradas y se
referencien desde el texto con el n�mero entre par�ntesis\footnote{Si
  quieres que queden entre corchetes ('[', ']'), debes quitar el
  \texttt{[round]} que aparece en la inclusi�n del paquete
  \texttt{natbib} en el fichero \texttt{TeXiS/TeXiS\_pream.tex}.}.  Ten
en cuenta, no obstante, que en ese caso el resultado de los comandos
\verb+\citep+ y \verb+citet+ es el mismo, por lo que puede que
necesites reescribir algunas frsaes; adem�s, \verb+\citeyear+ y
\verb+\citeauthor+ dejar�n de funcionar. Por �ltimo, cambiar el estilo
a un estilo predeterminado como ese elimina la posibilidad de incluir
referencias a p�ginas Web y a la Wikipedia explicadas antes.

\medskip

Otra posibilidad es que desees ajustar el funcionamiento del estilo
proporcionado por \texis. En ese caso, puedes editar el fichero
del estilo (el antes nombrado \texttt{TeXiS/TeXiS.bst}), respetando la
sint�xis que espera \texttt{bibtex}.



%-------------------------------------------------------------------
\section{Acr�nimos}
%-------------------------------------------------------------------
\label{capBiblio:sec:glosstex}

Los acr�nimos son las \emph{siglas} utilizadas a lo largo del
documento. \LaTeX\ permite facilitar su gesti�n, de manera que se
controle autom�ticamente el momento de la primera aparici�n de un
acr�nimo para poner su significado, o para que se genere
autom�ticamente una lista de acr�nimos a modo de resumen para ser
a�adida al final del documento.

Dado que el uso de acr�nimos no es tan conocido, de nuevo romperemos
nuestra promesa de no explicar aqu� aspectos concretos de \LaTeX, y
describiremos en la siguiente secci�n el funcionamiento del paquete
Gloss\TeX, que se encarga de la gesti�n de acr�nimos. Despu�s, nos
centraremos en el modo en el que \texis\ integra su uso.


%- - - - - - - - - - - - - - - - - - - - - - - - - - - - - - - - - -
\subsection{Acr�nimos con glosstex}
%- - - - - - - - - - - - - - - - - - - - - - - - - - - - - - - - - -
\label{cap5:sec:glosstex}

\figura{Bitmap/05/ListaAcronimos}{width=0.7\textwidth}%
       {cap5:fig:listaAcronimos}{Resultado de la lista de acr�nimos}

Al igual que ocurre con la bibliograf�a, el uso de acr�nimos en el documento supone dos cosas:

\begin{itemize}
  \item A lo largo del texto habr� que indicar en qu� punto se usa un acr�nimo, al igual que se indica cuando se utiliza una referencia bibliogr�fica.
  \item Al final del texto, tendr� que aparecer una lista con todos los acr�nimos utilizados (figura~\ref{cap5:fig:listaAcronimos}), al igual que ocurre con las citas.
\end{itemize}

Dada esta similitud con la bibliograf�a, no sorpende que para que sea \LaTeX qui�n nos gestione nuestros acr�nimos tendremos que hacer uso de una ``base de datos'' de acr�nimos, generalmente en ficheros con extensi�n \texttt{.gdf} (\emph{Glossary Data File}), que son conceptualmente similares a los ficheros \texttt{.bib} usados por Bib\TeX. Como ejemplo, a continuaci�n se indica el contenido (parcial) del fichero usado en este manual:

\begin{verbatim}
@entry{CVS, , \emph{Control Version System}, Sistema de Control
de Versiones}

@entry{GPL, , \emph{General Public License}, Licencia P�blica
General de GNU}
\end{verbatim}

Cada entrada se coloca dentro de un ``comando'' \texttt{@entry}, que tiene tres par�metros, separados por comas:

\begin{enumerate}
  \item \emph{Acr�nimo}: en el ejemplo, CVS y GPL. El acr�nimo hace tambi�n las veces de etiqueta identificativa, para referenciarlo dentro del texto.
  \item \emph{Representaci�n corta}: no aparece en ninguno de los dos ejemplos anteriores (se ha dejado en blanco). Es necesaria �nicamente si el acr�nimo \emph{tiene formato}. Veremos un ejemplo en un instante.
  \item \emph{Versi�n larga}: contiene la descripci�n completa del acr�nimo. Aunque estemos utilizando comas como separadores de cada uno de los tres elementos, al ser �ste el �ltimo campo, podr�a contener comas tal y como muestran los ejemplos.
\end{enumerate}


Una vez que se ha poblado el fichero con los acr�nimos, a lo largo del texto es posible a�adir comandos de Gloss\TeX\ para referirse a ellos, algo conceptualmente similar al uso de \verb+\cite+ respecto a la bibliograf�a. A continuaci�n se muestran los comandos disponibles\footnote{Para poder usarlos, ser� necesario haber inclu�do el paquete \texttt{glosstex} en el pre�mbulo del documento.}, y su resultado en el documento final\ifx\generaacronimos\undefined \footnote{Dado que \emph{no} est�s compilando el documento con el glosario, \emph{no} podr�s ver el resultado correcto aqu�. Consulta la secci�n~\ref{cap5:subsec:acronimosEnTeXiS} para m�s informaci�n.}\fi:

% Intento de hacerlo con una tabla. Queda mal porque ocupa demasiado
% en horizontal.
%
% \begin{center}
% \begin{tabular}{|c|c|c|}
% \hline
% Versi�n & Comando & Resultado \\
% \hline
% Corta (\emph{\textsf{s}hort}) & \verb+\acs{CVS}+ & \acs{CVS} \\
% Larga (\emph{\textsf{l}ong}) & \verb+\acl{CVS}+ & \acl{CVS} \\
% Completa (\emph{\textsf{f}ull}) & \verb+\acf{CVS}+ & \acf{CVS} \\
% \hline
% \end{tabular}
% \end{center}

\begin{itemize}
  \item \verb+\acs{CVS}+ (\emph{\textbf{s}hort}): \acs{CVS}
  \item \verb+\acl{CVS}+ (\emph{\textbf{l}ong}): \acl{CVS}
  \item \verb+\acf{CVS}+ (\emph{\textbf{f}ull}): \acf{CVS}
  \item \verb+\ac{CVS}+: es la m�s interesante de todas. Funciona como \verb+\acf+ la primera vez que aparece en el texto, y como \verb+\acs+ el resto de las veces.
\end{itemize}

Las tres primeras muestran diferentes partes de la entrada definida en el fichero \texttt{.gdf}. La m�s completa es la mostrada por \verb+\acf+, que combina la versi�n corta con la larga, que coloca entre par�ntesis. No obstante, el comando m�s interesante es \verb+\ac+, que ``recuerda'' si ya se introdujo un acr�nimo anteriormente en el documento, mostrando su descripci�n completa �nicamente la primera vez\footnote{El uso manual de \texttt{acf} \emph{no} hace que Gloss\TeX\ considere que el acr�nimo ya se ha ``presentado'', por lo que si se utiliza primero \texttt{acf} y m�s adelante \texttt{ac}, aparecer� de nuevo la versi�n completa.}. El modo en el que se construye esta descripci�n (con la versi�n larga entre par�ntesis) es personalizable, si bien el modo de hacerlo queda fuera del alcance de este documento.

Por tanto, el modo de uso recomendado de los acr�nimos es a�adir en el fichero \texttt{.gdf} todos los acr�nimos usados en el documento, y luego hacer \emph{siempre} referencia a ellos a trav�s del comando \verb+\ac+. \LaTeX\ incluir� la descripci�n completa la primera vez, y dejar� �nicamente el acr�nimo todas las dem�s.

\medskip

Hemos visto que en los ejemplos descritos, el propio acr�nimo \emph{hace las veces de identificador}, de modo que pare referenciarlo basta con un mero \verb+\ac{CVS}+ o similar. Esto ser� suficiente la mayor parte de las veces, pero en ocasiones tendremos acr�nimos m�s complejos, como por ejemplo PC$^2$ (\emph{Programming Contest Control}).
% Para no tener que volver a decirlo, aqu� lo hemos puesto a mano sin usar GlossTeX, para que lo vean bien tambi�n aquellos que no generen el documento con �l.
El problema, es que para que se muestre correctamente, el c�digo \LaTeX\ del acr�nimo es en realidad \verb+PC$^2$+, por lo que tendr�amos que referirnos a �l como \verb+\ac{PC$^2$}+. Esto no s�lo resulta inc�modo, sino que adem�s genera un error de compilaci�n, dado que no podemos a�adir comandos \LaTeX\ dentro de un identificador.

La soluci�n es hacer uso del segundo campo que dej�bamos vac�o en las entradas del fichero \texttt{.gdf} de ejemplo:

\begin{verbatim}
@entry{PC2, PC$^2$, \emph{Programming Contest Control}}
\end{verbatim}

Cuando se indica dicho campo, Gloss\TeX\ har� uso de �l para mostrar el acr�nimo, y utilizar� el primero s�lo como identificador. Ahora, para referirnos a �l bastar� con un \verb+\ac{PC2}+ que, la primera vez que se usa, se convierte en un \acf{PC2}, y las siguientes en un m�s corto \acs{PC2}\ifx\generaacronimos\undefined \footnote{De nuevo, como \emph{no} est�s compilando el documento con el glosario, \emph{no} podr�s ver el resultado correcto aqu�.}\fi.

\medskip

Aparte de la ayuda proporcionada por Gloss\TeX\ para evitarnos tener que escribir la versi�n completa de nuestros acr�nimos, tambi�n nos sirve para a�adir en la parte final un listado de acr�nimos donde se muestra el acr�nimo y su descripci�n larga (tal y como mostraba la figura~\ref{cap5:fig:listaAcronimos}). Al igual que ocurre con la bibliograf�a, s�lo aparecer�n en la lista aquellos acr�nimos que realmente se hayan usado a lo largo del documento. El comando para conseguirlo es:

\begin{verbatim}
\printglosstex(acr)
\end{verbatim}

No obstante, y al igual que ocurre con la bibliograf�a, para que toda esta infraestructura funcione es necesario ejecutar programas externos (aparte del propio \texttt{latex} o \texttt{pdflatex}). El proceso completo es el siguiente:

\begin{enumerate}
  \item El uso de comandos \verb+\ac?+ genera la inclusi�n de anotaciones dentro de los ficheros \texttt{.aux} generados al compilar los \texttt{.tex}.
  \item Al igual que se hace con la aplicaci�n \texttt{bibtex} para la bibliograf�a, en este caso usamos el programa \texttt{glosstex}\footnote{Este programa deber� estar instalado en el sistema en el que se est� generando el documento.} que los recorre y genera un nuevo fichero con extensi�n \texttt{.gxs} con informaci�n sobre las entradas referenciadas, y un \texttt{.gxg} con informaci�n de registro (\emph{log}).
  \item El fichero \texttt{.gxs} debe volverse a procesar usando otro programa, en este caso \texttt{makeindex}, que ordena alfab�ticamente las entradas utilizadas, y les aplica un formato concreto. Este proceso genera un nuevo fichero, con extensi�n \texttt{.glx}, que, finalmente, contiene todo correcto.
  \item El fichero anterior es utilizado por el comando \LaTeX\ \verb+printglosstex+ que debe ser inclu�do dentro de alg�n fichero \texttt{.tex} del documento, y que es conceptualmente similar al \verb+\bibliography{ficheros}+ usado para Bib\TeX.
\end{enumerate}

Todo esto complica el proceso de construcci�n del documento, que ahora
requiere pasos adicionales\footnote{Como siempre, tambi�n es v�lido el
  uso de \texttt{latex} en lugar de \texttt{pdflatex}, pero el fichero
  generado (\texttt{.dvi}) deber� despu�s ser convertido a PDF.}:

\begin{verbatim}
$ pdflatex Tesis
$ bibtex Tesis
$ glosstex Tesis acronimos.gdf
$ makeindex Tesis.gxs -o Tesis.glx -s glosstex.ist
$ pdflatex Tesis
\end{verbatim}

En la ejecuci�n de \texttt{glosstex} es necesario proporcionarle el nombre del fichero (o ficheros) con la ``base de datos'' de acr�nimos. En el caso de \texttt{bibtex} esto no es necesario, porque ya se lo proporcionamos directamente en los \texttt{.tex} con el comando \verb+\bibliography{ficheros}+. Adem�s, en la ejecuci�n a \texttt{makeindex} proporcionamos el par�metro \texttt{glosstex.ist} que indica el formato que queremos aplicar a nuestra lista de acr�nimos; ten en cuenta que \texttt{makeindex} se utiliza para la generaci�n de otros �ndices, por lo que para mantener la generalidad mantiene fuera dicho formato para que pueda ser adaptado a cada caso.

Para poder hacer uso, por tanto, de los acr�nimos, \emph{es necesario} disponer no s�lo del paquete \LaTeX\ \texttt{glosstex}, sino tambi�n de las \emph{aplicaciones} \texttt{glosstex} y \texttt{makeindex}. En la secci�n~\ref{cap2:sec:compilacion} ve�amos un modo simplificado de compilar el presente documento que no inclu�a las �rdenes para procesar los acr�nimos. Si no se ejecutan estas aplicaciones, el documento se generar� correctamente, salvo por la lista de acr�nimos que aparecer� vac�a.



%- - - - - - - - - - - - - - - - - - - - - - - - - - - - - - - - - -
\subsection{Acr�nimos en \texis}
%- - - - - - - - - - - - - - - - - - - - - - - - - - - - - - - - - -
\label{cap5:subsec:acronimosEnTeXiS}

Afortunadamente, al utilizar \texis\, pr�cticamente de lo �nico que hay que preocuparse es de crear la ``base de datos'' de acr�nimos y de hacer uso del comando \verb+\ac+ cuando corresponda. El resto de tareas son gestionadas autom�ticamente\footnote{Salvo, naturalmente, la instalaci�n de las propias aplicaciones \texttt{glosstex} y \texttt{makeindex} que deber� haber realizado el usuario.}.

Para dar soporte al uso de acr�nimos, \texis\ se apoya en varios ficheros:

\begin{itemize}
   \item \texttt{TeXiS/TeXiS\_acron.tex}: contiene el comando de Gloss\TeX\ que a�ade al documento un nuevo cap�tulo sin numeraci�n con la lista de acr�nimos. De manera predefinida, este cap�tulo se a�adir� al documento \emph{unicamente en modo ``release''} (consulta la secci�n~\ref{cap3:sec:modos-compilacion} para m�s informaci�n). En modo borrador (\emph{debug}) los acr�nimos no se incluyen en el documento, de modo que se ahorra algo de tiempo.

   \item \texttt{acronimos.gdf}: �ste es el fichero donde se recomienda a�adir los acr�nimos que se usen. Hace las veces de ``base de datos'' de acr�nimos. Siendo realistas, los ficheros \texttt{.tex} de \texis\ \emph{no} dependen de que se utilice este fichero en concreto. Tal y como se explic� en la secci�n anterior, es la ejecuci�n externa de las herramientas de Gloss\TeX\ la que recibir� el nombre en concreto. Sin embargo, \texis\ proporciona un modo autom�tico de construcci�n, descrito en el cap�tulo~\ref{cap:makefile}, que s� asume este nombre de fichero. La construcci�n se apoya en la herramienta \texttt{make}, de ah� que al principio del documento mencion�ramos que resultaba m�s c�modo hacer uso de plataformas GNU/Linux (que disponen de �l) en lugar de Windows.
\end{itemize}

En la versi�n en borrador, adem�s de no generarse la lista de acr�nimos como un cap�tulo m�s, \emph{tampoco} se a�aden las descripciones largas de los acr�nimos. Es decir, si se utiliza en alg�n lugar del documento las �rdenes de Gloss\TeX\ mencionadas previamente (\verb+\ac+, \verb+\acl+, etc�tera), �stas se ``puentear�n'' de modo que en el documento final tan s�lo aparecer� la propia etiqueta.
\ifx\generaacronimos\undefined
Por ejemplo, dado que esta versi�n se ha compilado \emph{sin} acr�nimos, aparecer� tan s�lo \acl{CVS} aunque en el c�digo \texttt{.tex} subyacente haya sido escrito \verb+\acl{CVS}+ (puedes comprobarlo si te parece).
\else
Por ejemplo, ante \verb+\acl{CVS}+, en lugar de aparecer ``\acl{CVS}'' se mostrar� �nicamente ``CVS''.
\fi

\medskip

Dado que no todos los documentos har�n uso de acr�nimos, y que, despu�s de todo, generarlos supone un esfuerzo en la fase de compilaci�n no despreciable\footnote{Este esfuerzo es real tan s�lo si no se hace uso del \texttt{Makefile} proporcionado por \texis.}, es posible que en ocasiones no se desee que se incluyan los acr�nimos tampoco en la versi�n final (modo \emph{release}). En ese caso, basta con que se modifique el fichero \texttt{config.tex} (el mismo en el que se escog�a qu� versi�n se quer�a compilar) y comentar la linea siguiente:

\begin{verbatim}
\def\acronimosEnRelease{1}
\end{verbatim}

Si se comenta, \texis\ asumir� que no se desean acr�nimos tampoco en la versi�n final, y no se incluir� el cap�tulo sin numeraci�n. De nuevo, ten en cuenta que en ese caso los comandos de Gloss\TeX\ se puentear�n tambi�n\footnote{Aunque en este caso no deber�a ser un problema porque si no se quieren los acr�nimos en la versi�n final ser� porque no se han usado.}.

\medskip

En el improbable caso en el que se quiera hacer uso de los acr�nimos para que sea el propio sistema el que se encargue de escribirnos el significado la primera vez que se usan, pero no se quiere que aparezca el listado final de los acr�nimos, entonces ser� necesario modificar directamente el fichero \texttt{Tesis.tex}, y dejar de incluir \texttt{TeXiS/TeXiS\_acron}, cuya inclusi�n es ahora mismo condicional en funci�n de si se usan o no los acr�nimos.

\medskip

Por �ltimo, hay que tener en cuenta que cuando se modifica el modelo de compilaci�n (por ejemplo, indicando versi�n final o borrador, o pidiendo que se a�adan o quiten los acr�nimos en \emph{release}) es necesario \emph{borrar los ficheros intermedios} generados durante la compilaci�n\footnote{Las razones que ocasionan esto quedan explicadas en la secci�n~\ref{cap6:subsec:glosstexYCambioModo}.}. Si se est� generando el documento haciendo uso de la infraestructura proporcionada por \texis\ a trav�s de su \texttt{Makefile} (descrita en el cap�tulo siguiente), bastar� con un sencillo:

\begin{verbatim}
make clean
\end{verbatim}

%- - - - - - - - - - - - - - - - - - - - - - - - - - - - - - - - - -
\subsection{M�s all� de \texis}
%- - - - - - - - - - - - - - - - - - - - - - - - - - - - - - - - - -

El nombre Gloss\TeX\ proviene en realidad de \emph{glossary}, por lo
que su objetivo inicial era realizar \emph{glosarios}, no meras listas
de acr�nimos. Para ser honestos, en las entradas de los ficheros
\texttt{.gdf} podemos a�adir una \emph{descripci�n completa} con una
descripci�n de la entrada:

\begin{verbatim}
@entry{PC2, PC$^2$, \emph{Programming Context Control}}
Software de gesti�n utilizado en los concursos de programaci�n
impulsados por ACM, a trav�s del cual los concursantes env�an
sus soluciones, y los jueces acceden a ellas y las valoran.
\end{verbatim}

Esa descripci�n \emph{est� fuera} de la entrada \texttt{@entry}, y
resulta �til en el caso de que quisieramos mantener un glosario de
palabras. \texis\ \emph{no} proporciona soporte para esto, por lo que
si lo quieres utilizar, tendr�s que a�adir la infraestructura
necesaria por tu cuenta.

%-------------------------------------------------------------------
\section*{\NotasBibliograficas}
%-------------------------------------------------------------------
\TocNotasBibliograficas

Existen numerosas publicaciones relacionadas con BibTeX; para una
descripci�n de los tipos de entradas que se soportan, etc., se puede
consultar \citet{BibTexOriginal} o \citet{latexCompanion}. En esta
�ltima referencia tambi�n puede encontrarse informaci�n sobre las
posibilidades del paquete \texttt{natbib}.

Por otro lado, es tambi�n f�cil encontrar informaci�n sobre c�mo
cambiar el estilo utilizado (es decir, lo que hemos hecho en
\texis\ para a�adir el \verb+@lastaccess+ o ajustar las cadenas
que aparecen). Por ejemplo, una descripci�n en espa�ol es
\citet{AtazLopez}. Tambi�n se puede consultar \citet{latexCompanion} o
\citet{patashnik1988a}.

\medskip

Respecto a Gloss\TeX, la fuente principal de informaci�n es la
disponible en el cat�logo de paquetes de \TeX. Puedes encontrarla en
\url{http://www.ctan.org/tex-archive/help/Catalogue/entries/glosstex.html}.

%-------------------------------------------------------------------
\section*{\ProximoCapitulo}
%-------------------------------------------------------------------
\TocProximoCapitulo

Con este cap�tulo terminan los cap�tulos m�s importantes del manual,
donde se ha contado lo que se debe saber para utilizar \texis.  El
pr�ximo cap�tulo describe el fichero \texttt{Makefile} proporcionado.
El fichero permite generar de forma f�cil el documento final,
utilizando la utilidad \texttt{make} disponible en virutalmente todas
las plataformas. Somos conscientes de que no todo el mundo querr�
utilizar este mecanismo para generar el documento final (muchos
usuarios preferir�n utililizar las opciones del editor que
utilicen); por eso lo hemos puesto al final.

% Variable local para emacs, para  que encuentre el fichero maestro de
% compilaci�n y funcionen mejor algunas teclas r�pidas de AucTeX
%%%
%%% Local Variables:
%%% mode: latex
%%% TeX-master: "../ManualTeXiS.tex"
%%% End:

%%---------------------------------------------------------------------
%
%                          Cap�tulo 6
%
%---------------------------------------------------------------------
%
% 06Makefile.tex
% Copyright 2009 Marco Antonio Gomez-Martin, Pedro Pablo Gomez-Martin
%
% This file belongs to the TeXiS manual, a LaTeX template for writting
% Thesis and other documents. The complete last TeXiS package can
% be obtained from http://gaia.fdi.ucm.es/projects/texis/
%
% Although the TeXiS template itself is distributed under the 
% conditions of the LaTeX Project Public License
% (http://www.latex-project.org/lppl.txt), the manual content
% uses the CC-BY-SA license that stays that you are free:
%
%    - to share & to copy, distribute and transmit the work
%    - to remix and to adapt the work
%
% under the following conditions:
%
%    - Attribution: you must attribute the work in the manner
%      specified by the author or licensor (but not in any way that
%      suggests that they endorse you or your use of the work).
%    - Share Alike: if you alter, transform, or build upon this
%      work, you may distribute the resulting work only under the
%      same, similar or a compatible license.
%
% The complete license is available in
% http://creativecommons.org/licenses/by-sa/3.0/legalcode
%
%---------------------------------------------------------------------

\chapter{Makefile}
\label{cap6}
\label{cap:makefile}

\begin{FraseCelebre}
\begin{Frase}
A fuerza de construir bien, se llega a buen arquitecto.
\end{Frase}
\begin{Fuente}
Arist�teles
\end{Fuente}
\end{FraseCelebre}

\begin{resumen}
Este cap�tulo describe la infraestructura de creaci�n del
documento final, apoyada en la herramienta \texttt{make}
de GNU.
\end{resumen}

%-------------------------------------------------------------------
\section{Introducci�n}
%-------------------------------------------------------------------
\label{cap6:sec:intro}

Ya se ha esbozado a lo largo de este manual que la generaci�n del
documento final requiere varias etapas, algo que es de hecho inherente
al propio \LaTeX\ (o Pdf\LaTeX). En la
secci�n~\ref{cap2:sec:compilacion} vimos que era necesario la
invocaci�n a \texttt{pdflatex} tres veces, junto con el uso de
\texttt{bibtex}. En la secci�n~\ref{cap5:sec:glosstex} se a�adi� la
necesidad de invocar a \texttt{glosstex} y a \texttt{makeindex} para
a�adir el listado de acr�nimos. El resultado es una generaci�n
bastante laboriosa que requiere dar varios pasos en un orden concreto.

El mundo del desarrollo del software ha lidiado con un problema
similar (m�s complejo, de hecho) desde hace d�cadas, y que se ha ido
resolviendo con diferentes herramientas, siendo \texttt{make} una de
las m�s extendidas. \texis\ proporciona un fichero \texttt{Makefile}
para ser utilizado con ella, y simplificar la generaci�n del
documento, as� como otras labores rutinarias. Dado que la
infraestructura \texttt{make} no est� disponible nativamente en
plataformas Windows, s�lo podr� utilizarse sobre GNU/Linux y otras
variantes de Unix\footnote{Windows dispone de Cygwin que proporciona
  muchas de las herramientas habituales en Unix. \emph{No} hemos
  probado \texis\ sobre �l; si lo haces, �no dudes en contarnos la
  experiencia!}. Incluso aunque se utilizara \texttt{nmake}, una
herramienta similar a \texttt{make} proporcionada con Visual Studio,
el \texttt{Makefile} de \texis\ no funcionar�a dado que hace uso de
comandos que s�lo est�n disponibles en Unix/Linux. Es por ello que en
el primer cap�tulo recomend�bamos el uso de dicho sistema.

En la secci�n siguiente, se describen los diferentes \emph{objetivos}
que este \texttt{Makefile} proporciona. La
secci�n~\ref{cap6:sec:funcionamiento} describe brevemente el
funcionamiento de algunas de sus partes.

%-------------------------------------------------------------------
\section{Objetivos del \texttt{Makefile}}
%-------------------------------------------------------------------
\label{cap6:sec:objetivos}

En la terminolog�a de \texttt{make}, un objetivo es un \emph{grupo de
  tareas} que se ejecutan en conjunto. En el entorno del desarrollo
del software, esas tareas suelen estar enfocadas a la compilaci�n del
proyecto, aunque tambi�n se incluyen objetivos para, por ejemplo,
borrar los ficheros intermedios, o instalar el programa reci�n
compilado.

El \texttt{Makefile} de \texis\ sigue esa misma filosof�a,
proporcionando los siguientes objetivos:

\begin{itemize}
\item \texttt{pdflatex}: es el objetivo por defecto. Genera el
  documento utilizando Pdf\LaTeX. Se encarga de generar la
  bibliograf�a, los acr�nimos, y de compilar varias veces el documento
  para que se actualicen correctamente las referencias.

\item \texttt{latex}: genera el documento utilizando \LaTeX, y
  convierte el \texttt{.dvi} resultante en \texttt{.pdf}. Tiene en
  cuenta las necesidades en cuanto al formato de las im�genes
  descritas en la secci�n~\ref{cap4:solTeXiS}, por lo que se
  convierten autom�ticamente a formato \texttt{.eps}.

\item \texttt{imagenes}: se encarga de convertir las im�genes (tanto
  vectoriales como de mapas de bits) a formato
  \texttt{.eps}. Normalmente este objetivo no necesitar� ser lanzado
  manualmente nunca; el objetivo \texttt{latex} anterior lo hace por
  nosotros.

\item \texttt{imagenesvectoriales} e \texttt{imagenesbitmap}:
  convierte o bien las im�genes vectoriales, o bien las de mapas de
  bits a formato \texttt{.eps}. Igual que antes, normalmente no se
  necesitan invocar manualmente.

\item \texttt{fast}: genera el documento de manera r�pida utilizando
  Pdf\LaTeX. Se limita a generarlo una �nica vez, sin invocar a
  Bib\TeX\ ni a Gloss\TeX. Est� pensada para la compilaci�n ``del d�a
  a d�a'' cuando se a�ade algo de texto y se quiere ver r�pidamente el
  resultado, sin preocuparnos de que las referencias queden
  correctamente actualizadas. Si este objetivo se combina con el
  comando \verb+\compilaCapitulo+ descrito en la secci�n
  \ref{cap3:sec:compilacion-rapida}, la compilaci�n puede resultar muy
  r�pida.

\item \texttt{fastlatex}: similar al anterior, pero realiza la
  generaci�n utilizando \LaTeX. En este caso, el \texttt{.dvi} sigue
  convirti�ndose a \texttt{.pdf}.

\item \texttt{clean}: elimina todos los ficheros intermedios creados
  durante la generaci�n del documento. Este objetivo resulta
  interesante cuando se quiere reconstruir el documento completamente,
  sin basarse en informaci�n previa. Es, de hecho, necesario lanzarlo
  cuando se compila el documento con la lista de acr�nimos por primera
  vez (o cuando deja de hacerse). Si no se ha modificado la definici�n
  de la constante \verb+\acronimosEnRelease+ (consulta la
  secci�n~\ref{cap5:subsec:acronimosEnTeXiS} para informaci�n sobre
  ella), esto ocurrir� siempre que se cambie el modo de configuraci�n
  de borrador a versi�n final
  (secci�n~\ref{cap3:sec:modos-compilacion}). Ten en cuenta que este
  objetivo \emph{borra tambi�n} los ficheros \texttt{.eps}, por lo que
  si has modificado el modelo de gesti�n de im�genes (para que los
  originales sean \texttt{.eps} que se convierten a \texttt{.pdf} si
  se usa Pdf\LaTeX, consulta la
  secci�n~\ref{cap6:subsec:compilacionImagenes} para m�s informaci�n)
  entonces tendr�s que modificar tambi�n este objetivo.

\item \texttt{distclean}: similar a la anterior, pero tambi�n borra el
  fichero \texttt{.pdf} generado, y los ficheros de copia de seguridad
  de los \texttt{.tex} creados por los editores de texto m�s
  habituales (con extensiones \verb+.tex~+ y \texttt{.backup}).

\item \texttt{crearZip}: genera el documento (con Pdf\LaTeX) y crea un
  fichero \texttt{.zip} con �l y todos los fuentes (inclu�do
  im�genes). El fichero es �til para distribuirlo a revisores que
  tengan intenci�n de cambiar y regenerar el documento.

\item \texttt{crearVersion}: similar al anterior, pero copia el .zip
  en el subdirectorio \texttt{VersionesPrevias} incluyendo en el
  nombre la fecha y hora actuales. Es �til para realizar copias de
  seguridad locales o para ``congelar versiones'' en ``hitos''
  concretos de la escritura.

\item \texttt{crearBackup}: genera el documento, y hace una copia de
  \emph{todo} el directorio (incluyendo el subdirectorio
  \texttt{VersionesPrevias} mencionado antes) comprimi�ndola en un
  fichero \texttt{.zip} que copia en el \emph{directorio padre} del
  actual. Est� pensado para hacer una copia de seguridad completa que
  luego sea guardada en alg�n otro lugar.

\item \texttt{ayuda} o \texttt{help}: muestra una descripci�n de todos
  los objetivos anteriores.

\end{itemize}

En el d�a a d�a, los m�s �tiles son \texttt{pdflatex}, \texttt{fast} y
\texttt{clean}. Para invocar a cualquiera de ellos en un entorno
GNU/Linux donde \texttt{make} est� disponible bastar� con:

\begin{verbatim}
$ make <objetivo>
\end{verbatim}
% AucTeX (en emacs) cree hemos abierto una ecuaci�n en el verbatim
% por el signo dolar, y las teclas r�pidas se vuelven un poco locas.
% La "cerramos" con otro $

En el caso de que se quiera realizar la generaci�n usando Pdf\LaTeX,
al ser el objetivo por defecto (el primero que aparece en el
\texttt{Makefile}) no es necesario especificar nada:

\begin{verbatim}
$ make
pdflatex Tesis
This is pdfTeXk, Version 3.141592-1.40.3 (Web2C 7.5.6)
[...]
\end{verbatim}
% $

Si decides que tu herramienta de generaci�n por defecto sea \LaTeX,
seguramente quieras colocar el objetivo \texttt{latex} delante para
convertirlo en el que se ejecute por defecto.


%-------------------------------------------------------------------
\section{Funcionamiento interno}
%-------------------------------------------------------------------
\label{cap6:sec:funcionamiento}

En principio, el fichero \texttt{Makefile} deber�a funcionar por s�
solo si se siguen los convenios de \texis\ descritos en los cap�tulos
previos, por lo que la secci�n anterior deber�a ser suficiente para un
usuario normal. Aqu� describiremos algunos detalles internos que
pueden ser �tiles en algunos (idealmente pocos) casos.



%- - - - - - - - - - - - - - - - - - - - - - - - - - - - - - - - - -
\subsection{La compilaci�n de las im�genes}
%- - - - - - - - - - - - - - - - - - - - - - - - - - - - - - - - - -
\label{cap6:subsec:compilacionImagenes}

En el cap�tulo~\ref{cap:imagenes} se mencionaba que \texis\ es capaz
de independizarse de la diferencia entre los formatos de im�genes
aceptados por Pdf\LaTeX\ y \LaTeX\ siempre que el autor siga un
determinado convenio en el modo de gestionarlas y utilice la
infraestructura de generaci�n, es decir el fichero \texttt{Makefile}
al que se refiere este cap�tulo.

Dado que nosotros hacemos normalmente uso de Pdf\LaTeX, \texis\ asume
que de manera nativa se querr� utilizar �ste en lugar de \LaTeX, por
lo que muestra una clara tendencia hacia el formato de im�genes que
\texttt{pdflatex} soporta de manera nativa.

En concreto, el \texttt{Makefile} asume que las im�genes se
proporcionan en formato \texttt{.pdf} para el caso de las vectoriales,
y en formato \texttt{.jpg} o \texttt{.png} para los mapas de bits. Si
las im�genes se desarrollaron originalmente con Kivio, Corel, Visio,
Gimp o Photoshop, es responsabilidad del propio autor convertirlas a
los formatos soportados por Pdf\LaTeX. De esa manera, el
\texttt{Makefile} \emph{no} necesitar� realizar ning�n tipo de
transformaci�n de im�genes en el objetivo por defecto
\texttt{pdflatex}.

En el caso de que se desee utilizar \LaTeX, la situaci�n es m�s
complicada. El \texttt{Makefile} tendr� que buscar las im�genes (tanto
vectoriales como de mapas de bits) y convertirlas a \texttt{.eps}. De
eso se encargan los objetivos \texttt{imagenesbitmap} y
\texttt{imagenesvectoriales} respectivamente.

Ambos se apoyan en el convenio de directorios propuesto por \texis, en
el que se asume que las im�genes vectoriales estar�n en el directorio
\texttt{Imagenes/Vectorial}, y las de mapas de bits en
\texttt{Imagenes/Bitmap}, con un subdirectorio dentro por
cap�tulo\footnote{\emph{No} se soportan subdirectorios adicionales
  dentro de los directorios de cada cap�tulo. Esta restricci�n se debe
  al modo en el que los \emph{scripts} buscan los ficheros de im�genes
  que hay que convertir. Consulta cualquiera de los ficheros
  \texttt{updateAll.sh} dentro de \texttt{Imagenes/Vectorial} o
  \texttt{Imagenes/Bitmap} para ver los detalles.}. El
\texttt{Makefile} encadena una serie de invocaciones a otros
\texttt{Makefile} y algunas llamadas a \emph{scripts} del \emph{shell}
(\texttt{bash}) para realizar la conversi�n. En �ltima instancia, el
responsable de dicha conversi�n es un \emph{script} llamado
\texttt{update-eps.sh}, del que, en realidad, existen dos versiones,
uno dentro de \texttt{Imagenes/Vectorial} y otro en
\texttt{Imagenes/Bitmap}, espec�ficos para cada uno de los dos tipos
de im�genes. Para convertir los \texttt{.pdf} vectoriales a
\texttt{.eps}, se hace uso de la aplicaci�n \texttt{pdftops}, que
deber� estar instalada. Por su parte, para convertir las im�genes de
mapas de bits \texttt{.jpg} o \texttt{.png} se usa \texttt{sam2p}. Los
scripts terminan con error (deteniendo por tanto la generaci�n del
documento) si estas herramientas no est�n disponibles. Se han
desarrollado de tal manera que el error �nicamente se lance si
realmente se necesita convertir alguna imagen. Adem�s, se evita
regenerar los ficheros \texttt{.eps} si los originales (\texttt{.pdf},
\texttt{.jpg} o \texttt{.png}) no han sufrido cambios desde la �ltima
generaci�n. Como ya se dijo previamente, se debe tener en cuenta que
el \texttt{Makefile} considera a esos ficheros \texttt{.eps} como
\emph{ficheros generados}, por lo que son eliminados por el objetivo
\texttt{clean}, y se anima a que se configure el control de versiones
(CVS, SVN, etc�tera) para que \emph{los ignore}.

\medskip

Si por alguna raz�n se prefiere \LaTeX\ a Pdf\LaTeX, entonces
resultar� m�s c�modo utilizar el formato \texttt{.eps} como
predefinido, de manera que a partir de las im�genes originales
(creadas con cualquier programa de dibujo) se generen los
\texttt{.eps} en lugar de los \texttt{.pdf} mencionados antes. Si,
adem�s, se quiere mantener la posibilidad de generar el documento con
\texttt{pdflatex} (aunque sea a costa de trabajar m�s convirtiendo las
im�genes), deber� adaptarse la infraestructura de generaci�n en varios
puntos:

\begin{itemize}
\item Modificar los ficheros \texttt{update-eps.sh} que se encuentran
  en los directorios \texttt{Imagenes/Bitmap} y
  \texttt{Imagenes/Vectorial} para que conviertan los \texttt{.eps}
  ``fuente'' en \texttt{.pdf}. En este caso no merece la pena
  convertir a \texttt{.jpg} o \texttt{.png} las im�genes vectoriales;
  resulta m�s c�modo convertir todo a \texttt{.pdf}. Para eso, se
  puede utilizar \texttt{epstopdf} (que deber� estar instalado). Una
  ventaja extra es que ambas versiones de \texttt{update-eps.sh} ser�n
  iguales, no como en el caso anterior en el que hab�a que diferenciar
  entre vectoriales y de im�genes de bits.

  Por mantener la coherencia, los ficheros deber�an renombrarse a algo
  como \texttt{update-pdf.sh}, en cuyo caso habr� que ajustar los
  \emph{scripts} \texttt{updateAll.sh} para modificar su invocaci�n.

\item Modificar el fichero \texttt{Makefile} principal (en el
  directorio ``ra�z'' de \texis) en varios puntos:
    \begin{itemize}
    \item Poner como objetivo por defecto a \texttt{latex} en lugar de
      a \texttt{pdflatex}. Esto es una mera cuesti�n de comodidad, y
      para llevarlo a cabo basta colocar en primer lugar el objetivo
      de \texttt{latex}.
    \item Renombrar \texttt{fast} a \texttt{fastpdflatex} y
      \texttt{fastlatex} a \texttt{fast}. De nuevo, esto es una
      cuesti�n de comodidad.
    \end{itemize}

  \item Modificar el gui�n (\emph{script}) \texttt{cleanAll.sh}
    situado tanto en el directorio \texttt{Imagenes/Vectorial} como en
    \texttt{Imagenes/Bitmap}. Ambos son invocados durante la ejecuci�n
    del objetivo \texttt{clean} del \texttt{Makefile} principal. En la
    versi�n inicial de \texis, \emph{se borran} los ficheros
    \texttt{.eps}, dado que son producto de la compilaci�n. Al usar
    \LaTeX, hay que conservarlos, y borrar los ficheros \texttt{.pdf}
    generados en los directorios de las im�genes.

  \item Modificar el fichero \texttt{.cvsignore} o el equivalente en
    el sistema de control de versiones que se est� usando (si hay
    alguno) para que se ignoren los nuevos tipos de ficheros generados
    (\texttt{.pdf}) en lugar de los \texttt{.eps} que vienen
    predefinidos en \texis. Consulta la secci�n~\ref{cap4:variaciones}
    para m�s informaci�n.

\end{itemize}


%- - - - - - - - - - - - - - - - - - - - - - - - - - - - - - - - - -
\subsection{Makefile, Gloss\TeX, y cambio de modo de generaci�n}
%- - - - - - - - - - - - - - - - - - - - - - - - - - - - - - - - - -
\label{cap6:subsec:glosstexYCambioModo}


En la secci�n \ref{cap5:subsec:acronimosEnTeXiS} se comentaba que al
cambiar el modo de generaci�n de documento desde ``depuraci�n''
(\emph{debug}) a versi�n final (\emph{release}) o viceversa, era
necesario realizar un \texttt{make clean} previo debido al uso de
Gloss\TeX. Aunque en la pr�ctica es suficiente con recordar hacerlo,
en esta secci�n se explican las causas de esta necesidad.

\medskip

Como se recordar� de la secci�n~\ref{cap5:sec:glosstex}, cuando se
utilizan acr�nimos, tras varias etapas termina consigui�ndose un
fichero con extensi�n \texttt{.glx} con la descripci�n de aqu�llos que
se usaron. En la siguiente compilaci�n del documento, el comando
\verb+\printglosstex+ se encargar� de recoger el contenido de dicho
fichero e incrustarlo en esa posici�n.

Ten en cuenta que la primera vez que se compila el documento,
\emph{no} existir� ning�n fichero \texttt{.glx}; esto es perfectamente
legal, dado que \verb+\printglosstex+ no se quejar� si el fichero
\emph{no} existe. Por desgracia, s� generar� un error si existe
\emph{pero est� vac�o}.

Cuando en \texis\ el uso de acr�nimos est� desactivado (lo que ocurre
en la generaci�n de depuraci�n), los comandos de Gloss\TeX\ como
\verb+\acs+, \verb+\acl+ etc�tera se puentean y no se tienen en
cuenta, por lo que en los ficheros intermedios no se informar� del uso
de ning�n acr�nimo. Sin embargo, durante el proceso de compilaci�n del
documento, el \texttt{Makefile} insistir� en realizar los pasos
necesarios para la generaci�n de acr�nimos (es decir, invocar al
ejecutable \texttt{glosstex} y a \texttt{makeindex}). �stos
desembocar�n en la creaci�n de un fichero \texttt{.glx} vac�o, al no
haber ning�n acr�nimo usado. En principio, esto significa que
\verb+\printglosstex+ fallar�. Afortunadamente, \texis\ \emph{no}
a�ade dicho comando \LaTeX\ al documento cuando los acr�nimos est�n
desactivados, lo que evita el problema.

Sin embargo, cuando se activa su generaci�n (al pasar a modo de
compilaci�n \emph{release}), \texis\ comienza a incluir el comando. En
la primera compilaci�n del documento, por tanto, \verb+\printglosstex+
se encontrar� que el fichero \texttt{.glx} est� vac�o (de la
compilaci�n en \emph{debug} anterior), y fallar�. Para solucionarlo,
basta en realidad con eliminar dicho fichero; sin embargo, resulta
mucho m�s f�cil de recordar (y por tanto pr�ctico) limitarse a
realizar un \texttt{make clean} que borra, entre otros, el
\texttt{.glx} por nosotros.

\medskip

Por otro lado, cuando el documento se ha generado correctamente con
los acr�nimos (normalmente en modo \emph{release}) y se vuelve a
generar sin ellos (en modo \emph{debug}), en la primera compilaci�n
\LaTeX\ (o Pdf\LaTeX) har� uso tanto de los fuentes del documento como
de los ficheros auxiliares (\texttt{.aux}) generados en la compilaci�n
anterior para acelerar el proceso. En esos ficheros se encuentran
entradas relativas a los acr�nimos que se utilizaron en la �ltima
compilaci�n (en \emph{release}). Al hacer ahora la compilaci�n sin
acr�nimos, no se habr�n inclu�do los paquetes \LaTeX\ que comprenden
los comandos de Gloss\TeX, por lo que \LaTeX\ fallar� al no
comprenderlos. De nuevo, para solucionar este problema es suficiente
con borrar los ficheros \texttt{.aux} generados en la compilaci�n
anterior; sin embargo resulta mucho m�s simple realizar un
\texttt{make clean} con la infraestructura de generaci�n proporcionada
por \texis.


%-------------------------------------------------------------------
%\section*{\ProximoCapitulo}
%-------------------------------------------------------------------
%\TocProximoCapitulo


%-------------------------------------------------------------------
\section*{\NotasBibliograficas}
%-------------------------------------------------------------------
\TocNotasBibliograficas

Sobre la utilidad \texttt{make} hay una gran cantidad de informaci�n
en Internet. Quiz� el lugar de referencia es la p�gina oficial del
proyecto de GNU (\url{http://www.gnu.org/software/make/}), aunque
contiene mucha m�s informaci�n de la necesaria para comprender el
\texttt{Makefile} proporcionado con \texis. Por el mero hecho de
proporcionar tambi�n una referencia impresa, puede consultarse tambi�n
\cite{libroMake}.



% Variable local para emacs, para  que encuentre el fichero maestro de
% compilaci�n y funcionen mejor algunas teclas r�pidas de AucTeX
%%%
%%% Local Variables:
%%% mode: latex
%%% TeX-master: "../ManualTeXiS.tex"
%%% End:


% Ap?ndices
\appendix
%---------------------------------------------------------------------
%
%                          Parte 3
%
%---------------------------------------------------------------------
%
% Parte3.tex
% Copyright 2009 Marco Antonio Gomez-Martin, Pedro Pablo Gomez-Martin
%
% This file belongs to the TeXiS manual, a LaTeX template for writting
% Thesis and other documents. The complete last TeXiS package can
% be obtained from http://gaia.fdi.ucm.es/projects/texis/
%
% Although the TeXiS template itself is distributed under the 
% conditions of the LaTeX Project Public License
% (http://www.latex-project.org/lppl.txt), the manual content
% uses the CC-BY-SA license that stays that you are free:
%
%    - to share & to copy, distribute and transmit the work
%    - to remix and to adapt the work
%
% under the following conditions:
%
%    - Attribution: you must attribute the work in the manner
%      specified by the author or licensor (but not in any way that
%      suggests that they endorse you or your use of the work).
%    - Share Alike: if you alter, transform, or build upon this
%      work, you may distribute the resulting work only under the
%      same, similar or a compatible license.
%
% The complete license is available in
% http://creativecommons.org/licenses/by-sa/3.0/legalcode
%
%---------------------------------------------------------------------

% Definici�n de la �ltima parte del manual, los ap�ndices

\partTitle{Ap�ndices}

\makepart

%---------------------------------------------------------------------
%
%                          Ap�ndice 1
%
%---------------------------------------------------------------------
%
% 01AsiSeHizo.tex
% Copyright 2009 Marco Antonio Gomez-Martin, Pedro Pablo Gomez-Martin
%
% This file belongs to the TeXiS manual, a LaTeX template for writting
% Thesis and other documents. The complete last TeXiS package can
% be obtained from http://gaia.fdi.ucm.es/projects/texis/
%
% Although the TeXiS template itself is distributed under the 
% conditions of the LaTeX Project Public License
% (http://www.latex-project.org/lppl.txt), the manual content
% uses the CC-BY-SA license that stays that you are free:
%
%    - to share & to copy, distribute and transmit the work
%    - to remix and to adapt the work
%
% under the following conditions:
%
%    - Attribution: you must attribute the work in the manner
%      specified by the author or licensor (but not in any way that
%      suggests that they endorse you or your use of the work).
%    - Share Alike: if you alter, transform, or build upon this
%      work, you may distribute the resulting work only under the
%      same, similar or a compatible license.
%
% The complete license is available in
% http://creativecommons.org/licenses/by-sa/3.0/legalcode
%
%---------------------------------------------------------------------

\chapter{As� se hizo...}
\label{ap1:AsiSeHizo}

\begin{FraseCelebre}
\begin{Frase}
Pones tu pie en el camino y si no cuidas tus pasos, nunca sabes a donde te pueden llevar.
\end{Frase}
\begin{Fuente}
John Ronald Reuel Tolkien, El Se�or de los Anillos
\end{Fuente}
\end{FraseCelebre}

\begin{resumen}
Este ap�ndice cuenta algunos aspectos pr�cticos que nos planteamos en
su momento durante la redacci�n de la tesis (a modo de ``as� se hizo
nuestra tesis''). En realidad no es m�s que una excusa para que �ste
manual tenga un ap�ndice que sirva de ejemplo en la plantilla.
\end{resumen}

%-------------------------------------------------------------------
\section{Edici�n}
%-------------------------------------------------------------------
\label{ap1:edicion}

Ya indicamos en la secci�n~\ref{cap3:sec:editores} (p�gina
\pageref{cap3:sec:editores}) que \texis\ est� preparada para
integrarse bien con emacs, en particular con el modo Auc\TeX.

Eso era en realidad un s�ntoma indicativo de que en nuestro trabajo
cotidiano utilizamos emacs para editar ficheros \LaTeX. Es cierto que
inicialmente utilizamos otros editores creados expresamente para la
edici�n de ficheros en \LaTeX, pero descubrimos emacs y ha llegado
para quedarse (la figura~\ref{cap3:fig:emacs} mostraba una captura del
mismo mientras cre�bamos este manual). Ten en cuenta que si utilizas
Windows, tambi�n puedes usar emacs para editar; no lo consideres como
algo que s�lo se utiliza en el mundo Unix. Nosotros lo usamos a diario
tanto en Linux como en Windows.

No obstante, hay que reconocer que emacs \emph{no} es f�cil de
utilizar al principio (el manual de referencia de \cite{emacsStallman}
tiene m�s de 550 p�ginas); su curva de aprendizaje es empinada,
especialmente si quieres sacarle el m�ximo partido, o al menos
beneficiarte de algunas de sus combinaciones de teclas. Pero una vez
que consigues \emph{no} mover las manos para desplazar el cursor sobre
el documento, manejas las teclas r�pidas para a�adir los comandos
\LaTeX\ m�s utilizados y conoces las combinaciones de Auc\TeX\ para
moverte por el documento o buscar las entradas de la bibliograf�a, no
cambiar�s f�cilmente a otro editor.

Si quieres aprovechar emacs, no debes dejar de leer el documento que
nos introdujo a nosotros en el modo Auc\TeX, ``\emph{Creaci�n de
  ficheros \LaTeX\ con GNU Emacs}'' \citep{AtazLopezEmacs}.

%-------------------------------------------------------------------
\section{Encuadernaci�n}
%-------------------------------------------------------------------
\label{ap1:encuadernacion}

Si has mirado con un poco de atenci�n este manual, habr�s visto que
los m�rgenes que tiene son bastante grandes. \texis\ no configura
los m�rgenes a unos valores concretos sino que, directamente, utiliza
los que se establecen por defecto en la clase \texttt{book} de \LaTeX.

Aunque es m�s o menos reconocido que si \LaTeX\ utiliza esos m�rgenes
debe tener una raz�n de peso (y de hecho la tiene, se utilizan esos
para que el n�mero de letras por l�nea sea el id�neo para su lectura),
cuando se comienza a mirar el documento con los ojos del que quiere
verlo encuadernado, es cierto que parecen excesivos. Y empiezas a
abrir libros, regla en mano, para medir qu� m�rgenes utilizan. Y
reconoces que son mucho m�s peque�os (y razonables) que el de tu
maravilloso escrito. Al menos ese fue nuestro caso.

En ese momento, una soluci�n es \emph{reducir} esos m�rgenes para que
aquello quede mejor. Sin embargo nuestra opci�n no fue esa. Si tu
situaci�n te permite \emph{no} encuadernar el documento en formato
DIN-A4, entonces puedes ir a la reprograf�a de turno y pedir que, una
vez impreso, te guillotinen esos m�rgenes.

Tu escrito quedar� entonces en ``formato libro'', mucho m�s manejable
que el gran DIN-A4, y con unos m�rgenes mucho m�s razonables. La
figura~\ref{ap1:fig:encuadernacion} muestra el resultado, comparando
el tama�o final con el de un folio, que aparece superpuesto.

\figura{Bitmap/0A/encuadernacion}{width=0.7\textwidth}%
       {ap1:fig:encuadernacion}{Encuadernaci�n y m�rgenes guillotinados}

%-------------------------------------------------------------------
\section{En el d�a a d�a}
%-------------------------------------------------------------------
\label{ap1:cc}

Para terminar este breve ap�ndice, describimos ahora un modo de
trabajo que, si bien no utilizamos en su d�a para la escritura de la
tesis, s� hemos utilizado desde hace alg�n tiempo para el resto de
nuestros escritos de \LaTeX , incluidos \texis\ y �ste, su manual.

Estamos hablando de lo que se conoce en el mundo de la ingenier�a del
software como \emph{integraci�n cont�nua} \citep{Fowler06}. En
concreto, la integraci�n cont�nua consiste en aprovecharse del
servidor del control de versiones para realizar, en cada
\emph{commit} o actualizaci�n realizada por los autores, una
comprobaci�n de si los ficheros que se han subido son de verdad
correctos.

En el mundo del desarrollo software donde un proyecto puede involucrar
decenas de personas realizando varias actualizaciones diarias, la
integraci�n cont�nua tiene mucha importancia. Despu�s de que un
programador realice una actualizaci�n, un servidor dedicado comprueba
que el proyecto sigue compilando correctamente (e incluso ejecuta los
test de unidad asociados). En caso de que la actualizaci�n haya
estropeado algo, el servidor de integraci�n env�a un mensaje de correo
electr�nico al autor de ese \emph{commit} para avisarle del error y
que �ste lo subsane lo antes posible, de forma que se perjudique lo
menos posible al resto de desarrolladores.

Esa misma idea la hemos utilizado en la elaboraci�n de \texis\ y
de este manual. Cada vez que uno de los autores sub�a al SVN alg�n
cambio, el servidor comprobaba que el fichero maestro segu�a siendo
correcto, es decir, que se pod�a generar el PDF final sin errores.

No entraremos en m�s detalles de c�mo hacer esto. El lector interesado
puede consultar \citet{CCLatex}. Como se explica en ese art�culo
algunas ventajas del uso de esta t�cnica son:

\begin{figure}[t]
  \centering
  %
  \subfloat[][P�gina de descarga del documento generado]{
     \includegraphics[width=0.445\textwidth]%
                     {Imagenes/Bitmap/0A/dashboard}
     \label{ap1:fig:dashboard}
  }
  \qquad
  \subfloat[][M�tricas del proyecto]{
     \includegraphics[width=0.445\textwidth]%
                     {Imagenes/Bitmap/0A/metrics}
     \label{ap1:fig:metrics}
  }
 \caption{Servidor de integraci�n cont�nua\label{ap1:fig:cc}}
\end{figure}

\begin{itemize}
\item Se tiene la seguridad de que la versi�n disponible en el control
  de versiones es v�lida, es decir, es capaz de generar sin errores el
  documento final.

\item Se puede configurar el servidor de integraci�n cont�nua para que
  cada vez que se realiza un \emph{commit}, env�e un mensaje de correo
  electr�nico \emph{a todos los autores} del mismo. De esta forma
  todos los colaboradores est�n al tanto del progreso del mismo.

\item Se puede configurar para que el servidor haga p�blico (via
  servidor Web) el PDF del documento (ver
  figura~\ref{ap1:fig:dashboard}). Esto es especialmente �til para
  revisores del texto como tutores de tesis, que no tendr�n que
  preocuparse de descargar y compilar los \texttt{.tex}.
\end{itemize}

Por �ltimo, el servidor tambi�n permite ver la evoluci�n del proyecto.
La figura~\ref{ap1:fig:metrics} muestra una gr�fica que el servidor de
integraci�n cont�nua muestra donde se puede ver la fecha (eje
horizontal) y hora (eje vertical) de cada \emph{commit} en el
servidor; los puntos rojos representan commits cuya compilaci�n fall�.



% Variable local para emacs, para  que encuentre el fichero maestro de
% compilaci�n y funcionen mejor algunas teclas r�pidas de AucTeX
%%%
%%% Local Variables:
%%% mode: latex
%%% TeX-master: "../ManualTeXiS.tex"
%%% End:


\backmatter

%
% Bibliograf?a
%

%---------------------------------------------------------------------
%
%                      configBibliografia.tex
%
%---------------------------------------------------------------------
%
% bibliografia.tex
% Copyright 2009 Marco Antonio Gomez-Martin, Pedro Pablo Gomez-Martin
%
% This file belongs to the TeXiS manual, a LaTeX template for writting
% Thesis and other documents. The complete last TeXiS package can
% be obtained from http://gaia.fdi.ucm.es/projects/texis/
%
% Although the TeXiS template itself is distributed under the 
% conditions of the LaTeX Project Public License
% (http://www.latex-project.org/lppl.txt), the manual content
% uses the CC-BY-SA license that stays that you are free:
%
%    - to share & to copy, distribute and transmit the work
%    - to remix and to adapt the work
%
% under the following conditions:
%
%    - Attribution: you must attribute the work in the manner
%      specified by the author or licensor (but not in any way that
%      suggests that they endorse you or your use of the work).
%    - Share Alike: if you alter, transform, or build upon this
%      work, you may distribute the resulting work only under the
%      same, similar or a compatible license.
%
% The complete license is available in
% http://creativecommons.org/licenses/by-sa/3.0/legalcode
%
%---------------------------------------------------------------------
%
% Fichero  que  configura  los  par�metros  de  la  generaci�n  de  la
% bibliograf�a.  Existen dos  par�metros configurables:  los ficheros
% .bib que se utilizan y la frase c�lebre que aparece justo antes de la
% primera referencia.
%
%---------------------------------------------------------------------


%%%%%%%%%%%%%%%%%%%%%%%%%%%%%%%%%%%%%%%%%%%%%%%%%%%%%%%%%%%%%%%%%%%%%%
% Definici�n de los ficheros .bib utilizados:
% \setBibFiles{<lista ficheros sin extension, separados por comas>}
% Nota:
% Es IMPORTANTE que los ficheros est�n en la misma l�nea que
% el comando \setBibFiles. Si se desea utilizar varias l�neas,
% terminarlas con una apertura de comentario.
%%%%%%%%%%%%%%%%%%%%%%%%%%%%%%%%%%%%%%%%%%%%%%%%%%%%%%%%%%%%%%%%%%%%%%
\setBibFiles{%
nuestros,latex,otros%
}

%%%%%%%%%%%%%%%%%%%%%%%%%%%%%%%%%%%%%%%%%%%%%%%%%%%%%%%%%%%%%%%%%%%%%%
% Definici�n de la frase c�lebre para el cap�tulo de la
% bibliograf�a. Dentro normalmente se querr� hacer uso del entorno
% \begin{FraseCelebre}, que contendr� a su vez otros dos entornos,
% un \begin{Frase} y un \begin{Fuente}.
%
% Nota:
% Si no se quiere cita, se puede eliminar su definici�n (en la
% macro setCitaBibliografia{} ).
%%%%%%%%%%%%%%%%%%%%%%%%%%%%%%%%%%%%%%%%%%%%%%%%%%%%%%%%%%%%%%%%%%%%%%
\setCitaBibliografia{
\begin{FraseCelebre}
\begin{Frase}
  Y as�, del mucho leer y del poco dormir, se le sec� el celebro de
  manera que vino a perder el juicio.
\end{Frase}
\begin{Fuente}
  Miguel de Cervantes Saavedra
\end{Fuente}
\end{FraseCelebre}
}

%%
%% Creamos la bibliografia
%%
\makeBib

% Variable local para emacs, para  que encuentre el fichero maestro de
% compilaci�n y funcionen mejor algunas teclas r�pidas de AucTeX

%%%
%%% Local Variables:
%%% mode: latex
%%% TeX-master: "../Tesis.tex"
%%% End:


%
% ?ndice de palabras
%

% S?lo  la   generamos  si  est?   declarada  \generaindice.  Consulta
% TeXiS.sty para m?s informaci?n.

% En realidad, el soporte para la generaci?n de ?ndices de palabras
% en TeXiS no est? documentada en el manual, porque no ha sido usada
% "en producci?n". Por tanto, el fichero que genera el ?ndice
% *no* se incluye aqu? (est? comentado). Consulta la documentaci?n
% en TeXiS_pream.tex para m?s informaci?n.
\ifx\generaindice\undefined
\else
%%---------------------------------------------------------------------
%
%                        TeXiS_indice.tex
%
%---------------------------------------------------------------------
%
% TeXiS_indice.tex
% Copyright 2009 Marco Antonio Gomez-Martin, Pedro Pablo Gomez-Martin
%
% This file belongs to TeXiS, a LaTeX template for writting
% Thesis and other documents. The complete last TeXiS package can
% be obtained from http://gaia.fdi.ucm.es/projects/texis/
%
% This work may be distributed and/or modified under the
% conditions of the LaTeX Project Public License, either version 1.3
% of this license or (at your option) any later version.
% The latest version of this license is in
%   http://www.latex-project.org/lppl.txt
% and version 1.3 or later is part of all distributions of LaTeX
% version 2005/12/01 or later.
%
% This work has the LPPL maintenance status `maintained'.
% 
% The Current Maintainers of this work are Marco Antonio Gomez-Martin
% and Pedro Pablo Gomez-Martin
%
%---------------------------------------------------------------------
%
% Contiene  los  comandos  para  generar  el �ndice  de  palabras  del
% documento.
%
%---------------------------------------------------------------------
%
% NOTA IMPORTANTE: el  soporte en TeXiS para el  �ndice de palabras es
% embrionario, y  de hecho  ni siquiera se  describe en el  manual. Se
% proporciona  una infraestructura  b�sica (sin  terminar)  para ello,
% pero  no ha  sido usada  "en producci�n".  De hecho,  a pesar  de la
% existencia de  este fichero, *no* se incluye  en Tesis.tex. Consulta
% la documentaci�n en TeXiS_pream.tex para m�s informaci�n.
%
%---------------------------------------------------------------------


% Si se  va a generar  la tabla de  contenidos (el �ndice  habitual) y
% tambi�n vamos a  generar el �ndice de palabras  (ambas decisiones se
% toman en  funci�n de  la definici�n  o no de  un par  de constantes,
% puedes consultar modo.tex para m�s informaci�n), entonces metemos en
% la tabla de contenidos una  entrada para marcar la p�gina donde est�
% el �ndice de palabras.

\ifx\generatoc\undefined
\else
   \addcontentsline{toc}{chapter}{\indexname}
\fi

% Generamos el �ndice
\printindex

% Variable local para emacs, para  que encuentre el fichero maestro de
% compilaci�n y funcionen mejor algunas teclas r�pidas de AucTeX

%%%
%%% Local Variables:
%%% mode: latex
%%% TeX-master: "./tesis.tex"
%%% End:

\fi

%
% Lista de acr?nimos
%

% S?lo  lo  generamos  si  est? declarada  \generaacronimos.  Consulta
% TeXiS.sty para m?s informaci?n.


\ifx\generaacronimos\undefined
\else
%---------------------------------------------------------------------
%
%                        TeXiS_acron.tex
%
%---------------------------------------------------------------------
%
% TeXiS_acron.tex
% Copyright 2009 Marco Antonio Gomez-Martin, Pedro Pablo Gomez-Martin
%
% This file belongs to TeXiS, a LaTeX template for writting
% Thesis and other documents. The complete last TeXiS package can
% be obtained from http://gaia.fdi.ucm.es/projects/texis/
%
% This work may be distributed and/or modified under the
% conditions of the LaTeX Project Public License, either version 1.3
% of this license or (at your option) any later version.
% The latest version of this license is in
%   http://www.latex-project.org/lppl.txt
% and version 1.3 or later is part of all distributions of LaTeX
% version 2005/12/01 or later.
%
% This work has the LPPL maintenance status `maintained'.
% 
% The Current Maintainers of this work are Marco Antonio Gomez-Martin
% and Pedro Pablo Gomez-Martin
%
%---------------------------------------------------------------------
%
% Contiene  los  comandos  para  generar  el listado de acr�nimos
% documento.
%
%---------------------------------------------------------------------
%
% NOTA IMPORTANTE:  para que la  generaci�n de acr�nimos  funcione, al
% menos  debe  existir  un  acr�nimo   en  el  documento.  Si  no,  la
% compilaci�n  del   fichero  LaTeX  falla  con   un  error  "extra�o"
% (indicando  que  quiz�  falte  un \item).   Consulta  el  comentario
% referente al paquete glosstex en TeXiS_pream.tex.
%
%---------------------------------------------------------------------


% Redefinimos a espa�ol  el t�tulo de la lista  de acr�nimos (Babel no
% lo hace por nosotros esta vez)

\def\listacronymname{Lista de acr�nimos}

% Para el glosario:
% \def\glosarryname{Glosario}

% Si se  va a generar  la tabla de  contenidos (el �ndice  habitual) y
% tambi�n vamos a  generar la lista de acr�nimos  (ambas decisiones se
% toman en  funci�n de  la definici�n  o no de  un par  de constantes,
% puedes consultar config.tex  para m�s informaci�n), entonces metemos
% en la  tabla de contenidos una  entrada para marcar  la p�gina donde
% est� el �ndice de palabras.

\ifx\generatoc\undefined
\else
   \addcontentsline{toc}{chapter}{\listacronymname}
\fi


% Generamos la lista de acr�nimos (en realidad el �ndice asociado a la
% lista "acr" de GlossTeX)

\printglosstex(acr)

% Variable local para emacs, para  que encuentre el fichero maestro de
% compilaci�n y funcionen mejor algunas teclas r�pidas de AucTeX

%%%
%%% Local Variables:
%%% mode: latex
%%% TeX-master: "../Tesis.tex"
%%% End:

\fi

%
% Final
%
%---------------------------------------------------------------------
%
%                      fin.tex
%
%---------------------------------------------------------------------
%
% fin.tex
% Copyright 2009 Marco Antonio Gomez-Martin, Pedro Pablo Gomez-Martin
%
% This file belongs to the TeXiS manual, a LaTeX template for writting
% Thesis and other documents. The complete last TeXiS package can
% be obtained from http://gaia.fdi.ucm.es/projects/texis/
%
% Although the TeXiS template itself is distributed under the 
% conditions of the LaTeX Project Public License
% (http://www.latex-project.org/lppl.txt), the manual content
% uses the CC-BY-SA license that stays that you are free:
%
%    - to share & to copy, distribute and transmit the work
%    - to remix and to adapt the work
%
% under the following conditions:
%
%    - Attribution: you must attribute the work in the manner
%      specified by the author or licensor (but not in any way that
%      suggests that they endorse you or your use of the work).
%    - Share Alike: if you alter, transform, or build upon this
%      work, you may distribute the resulting work only under the
%      same, similar or a compatible license.
%
% The complete license is available in
% http://creativecommons.org/licenses/by-sa/3.0/legalcode
%
%---------------------------------------------------------------------
%
% Contiene la �ltima p�gina
%
%---------------------------------------------------------------------


% Ponemos el marcador en el PDF al nivel adecuado, dependiendo
% de su hubo partes en el documento o no (si las hay, queremos
% que aparezca "al mismo nivel" que las partes.
\ifpdf
\ifx\tienePartesTeXiS\undefined
   \pdfbookmark[0]{Fin}{fin}
\else
   \pdfbookmark[-1]{Fin}{fin}
\fi
\fi

\thispagestyle{empty}\mbox{}

\vspace*{4cm}

\small

\hfill \emph{--�Qu� te parece desto, Sancho? -- Dijo Don Quijote --}

\hfill \emph{Bien podr�n los encantadores quitarme la ventura,}

\hfill \emph{pero el esfuerzo y el �nimo, ser� imposible.}

\hfill 

\hfill \emph{Segunda parte del Ingenioso Caballero} 

\hfill \emph{Don Quijote de la Mancha}

\hfill \emph{Miguel de Cervantes}

\vfill%space*{4cm}

\hfill \emph{--Buena est� -- dijo Sancho --; f�rmela vuestra merced.}

\hfill \emph{--No es menester firmarla -- dijo Don Quijote--,}

\hfill \emph{sino solamente poner mi r�brica.}

\hfill 

\hfill \emph{Primera parte del Ingenioso Caballero} 

\hfill \emph{Don Quijote de la Mancha}

\hfill \emph{Miguel de Cervantes}


\newpage
\thispagestyle{empty}\mbox{}

\newpage

% Variable local para emacs, para  que encuentre el fichero maestro de
% compilaci�n y funcionen mejor algunas teclas r�pidas de AucTeX

%%%
%%% Local Variables:
%%% mode: latex
%%% TeX-master: "../Tesis.tex"
%%% End:


\end{document}
