%---------------------------------------------------------------------
%
%                          Capítulo 1
%
%---------------------------------------------------------------------
%
% 00IntroduccionP.tex
% Copyright 2019 Alfonso Soria Muñoz, Pedro Domenech
%
% This file belongs to the TeXiS manual, a LaTeX template for writting
% Thesis and other documents. The complete last TeXiS package can
% be obtained from http://gaia.fdi.ucm.es/projects/texis/
%
% Although the TeXiS template itself is distributed under the 
% conditions of the LaTeX Project Public License
% (http://www.latex-project.org/lppl.txt), the manual content
% uses the CC-BY-SA license that stays that you are free:
%
%    - to share & to copy, distribute and transmit the work
%    - to remix and to adapt the work
%
% under the following conditions:
%
%    - Attribution: you must attribute the work in the manner
%      specified by the author or licensor (but not in any way that
%      suggests that they endorse you or your use of the work).
%    - Share Alike: if you alter, transform, or build upon this
%      work, you may distribute the resulting work only under the
%      same, similar or a compatible license.
%
% The complete license is available in
% http://creativecommons.org/licenses/by-sa/3.0/legalcode
%
%---------------------------------------------------------------------

\chapter{Introducción}

\begin{FraseCelebre}
\begin{Frase}
Púsose don Quijote delante de dicho carro, y haciendo en su fantasía
uno de los más desvariados discursos que jamás había hecho, dijo en
alta voz:
\end{Frase}
\begin{Fuente}
  Alonso Fernández de Avellaneda, El Ingenioso Hidalgo Don Quijote de
  la Mancha
\end{Fuente}
\end{FraseCelebre}

\begin{resumen}
  Este capítulo presenta una breve introducción a \texis.  El
  lector podrá hacerse una idea de qué es y para qué sirve. También se
  encuentra aquí una descripción del resto de capítulos del manual.
\end{resumen}


%-------------------------------------------------------------------
\section{Introducción}
%-------------------------------------------------------------------
\label{cap1:sec:introduccion}


Si estás leyendo estas líneas es muy posible que haya llegado la hora
de ponerte a escribir la tesis, después de mucho tiempo dando vueltas
al área de investigación concreta en el que estás inmerso. O puede que
estés a punto de empezar a escribir la memoria del proyecto de fin de
carrera, fin de master, o cualquier otro documento de cierta
envergadura.

Sea lo que sea lo que te traes entre manos, lo más probable es que no
sea fácil hacerlo. Muy posiblemente no tengas aún muy claro qué vas a
escribir, pero tu tutor/director/profesor te ha dicho que vayas
empezando a plasmar esas ideas sobre el papel para tener algo firme, y
sentir que vas avanzando.

Y entonces viene el problema de cómo escribirlo. Muy posiblemente
habrás escrito algún artículo en \LaTeX\ y estés convencido de que esa
es la vía a seguir para hacer un documento que superará las 10 páginas
y que tendrá bibliografía. O puede, simplemente, que alguien te haya
dicho que lo mejor es que escribas el proyecto en \LaTeX\ porque la
apariencia final es mejor, porque es más cómodo, o cualquier otra
razón.

Sea como fuere, parece que estás más o menos decidido a escribir tu
documento en \LaTeX. Bien hecho. Pero, ¿cómo?. Al contrario de lo que
suele ocurrir en congresos y en revistas, no tienes disponible ninguna
página en la que descargarte las ``instrucciones para los autores'',
con la cómoda plantilla en \LaTeX\ que tú, sufrido autor, simplemente
tienes que rellenar. No. Ahora las cosas son más complicadas.

Así que te vas a la guía de \LaTeX\ con la que empezaste (apostamos
que es la misma con la que hemos empezado todos), y ves las distintas
posibilidades que te ofrece en su ``\texttt{documentclass}'':
\texttt{article}, \texttt{report}, \texttt{book}, ... Y te quedas con
la última. Pero te asaltan muchas preguntas. ¿Cómo organizo todo esto?
o ¿cómo hago la portada? o incluso ¿qué hago para que no ponga
``Chapter'', sino ``Capítulo''?. En ese punto, es de suponer, has
pedido ayuda a la gente de alrededor y/o a tu buscador de Internet
favorito. Y de alguna forma, te has encontrado leyendo estas líneas.

Tenemos que decir que exactamente esa fue nuestra situación cuando por
fin nos decidimos a escribir nuestras tesis. Desgraciadamente, ni la
gente que teníamos alrededor ni nuestro buscador favorito supieron
contestarnos de forma satisfactoria, por lo que tuvimos que invertir
\emph{mucho tiempo} hasta conseguir que el resultado que salía de
nuestros \texttt{.tex} nos gustara, hasta que nos sentimos cómodos con
la estructura de los ficheros, con las macros disponibles y con el
modo de compilación.

Y para que nadie más pueda utilizar como excusa el no saber cómo
personalizar la clase \texttt{book} para retrasar el comienzo de su
tesis, para que nadie más se decida por Word u otro paquete ofimático
en vez de \LaTeX\ porque lo ve mucho más sencillo, en definitiva, para
que nadie pierda tanto tiempo como perdimos nosotros creando la
estructura, decidimos hacer público el esqueleto básico que
construimos nosotros para hacerlas. Ese esqueleto básico o plantilla
es \texis.

En vez de hacer disponible la plantilla o ficheros \texttt{.tex} sin
ningún contenido, proporcionamos un manual en formato PDF que (a no
ser que estés leyendo directamente el código \LaTeX), será lo que
estás leyendo. Este manual ha sido creado \emph{con la propia
  plantilla}. Por lo tanto, la distribución de \texis\ es en
realidad el código fuente de \emph{su propio manual}. Con su código
fuente entre tus manos, lo único que tienes que hacer es borrar su
contenido (\emph{este texto}), y rellenarlo con tu gran contribución
al mundo.  Como podrás comprobar, la estructura del propio manual
sigue el esquema de lo que podría ser una tesis, trabajo de
investigación o proyecto de fin de carrera, precisamente para que sea
fácil quitar el contenido textual y sustituirlo por el nuevo.

En los capítulos que siguen encontrarás toda la información necesaria
para poder utilizar los ficheros \LaTeX\ para crear tus propios
documentos. Además, el propio código fuente está lleno de comentarios
(especialmente en los ficheros que definen el estilo), por lo que
también en ellos encontrarás una buena fuente de información. Eso es
especialmente importante en caso de que quieras modificar en algo el
aspecto final de tu documento.

Esperemos que te sea de utilidad. Si es así, nos gustaría que lo
reconocieras en la sección de agradecimientos. Si durante tu proceso
de escritura has añadido algún aspecto que crees que puede ser
interesante para otros, no dudes en decírnoslo para intentar incluirlo
en siguientes versiones de la propia plantilla; tampoco dudes en
enviarnos sugerencias sobre las explicaciones de este manual para
poder mejorarlo con el tiempo. Por último, también puedes enviarnos el
resultado final para poner una referencia a él en la página de
descarga, donde, por cierto, puedes ver otros documentos creados con
la plantilla, lo que te permitirá coger ideas de cosas que puedes
variar. Recuerda que la versión más reciente de \texis\ está
disponible en \url{http://gaia.fdi.ucm.es/projects/texis/}.

%-------------------------------------------------------------------
\section{Qué es \texis}
%-------------------------------------------------------------------
\label{cap1:sec:que-es}

La plantilla que tienes entre las manos es, como hemos dicho, el
esqueleto del código fuente de las Tesis Doctorales de los dos autores
\citep{GomezMartinMA2008PhD, GomezMartinPP2008PhD}. Por tanto, sirve
para escribir otras Tesis Doctorales u otros documentos con estructura
similar de forma fácil.

\texis\ te permite además generar el fichero utilizando tanto el
comando \texttt{latex} (que genera de forma nativa ficheros
\texttt{dvi} que luego se convierten a ficheros \texttt{ps} o
\texttt{pdf}), como \texttt{pdflatex}. De esta forma el usuario final
puede elegir entre cualquiera de las dos herramientas\footnote{Esto es
  útil por ejemplo cuando quieres utilizar \texttt{pdflatex} pero
  finalmente el servicio de publicaciones sólo admite el uso de
  \texttt{latex}.}.  Aconsejamos, no obstante, la utilización de este
último, debido a que \texis\ contiene ciertos comandos para dotar al
PDF final de marcadores que permiten una navegación cómoda por el
fichero utilizando los visores tradicionales.

\medskip

Como explicaremos en el capítulo siguiente, la plantilla se aprovecha
mejor en sistemas GNU/Linux. Nota que hemos dicho que la plantilla
``\emph{se aprovecha mejor}'' en sistemas GNU/Linux, no que \emph{no
pueda utilizarse} en Windows o Mac; es evidente que \LaTeX\ es
multiplataforma, y por lo tanto puede compilarse en cualquier sistema
que tenga instalada una distribución del mismo.

La razón por esta ``desviación positiva'' hacia Linux estriba en que
para hacer más cómodo el proceso de edición y compilación, \texis\
proporciona ficheros que facilitan el proceso de generación del
fichero PDF final, tal y como se describe en el
capítulo~\ref{cap:makefile}.  Esos ficheros adicionales sólo funcionan
correctamente si son ejecutados en Linux.

%-------------------------------------------------------------------
\section{Qué no es}
%-------------------------------------------------------------------
\label{cap1:sec:que-no-es}

Esta plantilla \emph{no} es un manual de \LaTeX, ni una guía de
referencia, ni un compendio de preguntas frecuentes. De hecho, no nos
consideramos expertos en \LaTeX, por lo que no tendríamos fuerzas para
escribir algo así. Si necesitas un manual de \LaTeX, puedes encontrar
muchos y muy buenos en Internet. Al final de este capítulo aparece una
lista con algunos de ellos.

La plantilla tampoco es \emph{una clase} de \LaTeX. Si miras el código
fuente podrás comprobar que el documento comienza con
\verb+\documentclass{book}+\footnote{Personalizado, eso sí, para que
  utilice DIN A-4, a doble cara y con letra de 11 puntos.}, por lo que
se basa en la clase \texttt{book}.

La plantilla tampoco te ayudará a gestionar tu bibliografía. Los
\texttt{.bib} los tendrás que crear y organizar tú ya sea de forma
manual o con alguna herramienta diseñada para ello.

\medskip

Queremos una vez más insistir antes de terminar que no somos expertos
en \LaTeX.  Durante el proceso de escritura de nuestras Tesis nos
tuvimos que enfrentar a problemas de formato que tuvimos que
solucionar buscando en Internet o preguntando a personas cercanas. Y
podemos decir que prácticamente todos los problemas a los que nos
hemos enfrentado en nuestra vida como usuarios de \LaTeX\ están
resueltos aquí, pues sendas Tesis han sido los documentos más extensos
que hemos escrito.

Por lo tanto, si tienes alguna duda concreta de \LaTeX, en vez de
preguntarnos a nosotros, busca en foros de Internet o en la
documentación del paquete que estás utilizando. A buen seguro
encontrarás ahí la respuesta. Si la duda que tienes es relativa a la
plantilla, revisa los comentarios que encontrarás en el código fuente,
hay ciertas cosas de demasiado bajo nivel que hemos preferido no
contar en el texto. Y sólo como último recurso, preguntanos a
nosotros, aunque ya te advertimos que puede que no sepamos
responderte. Querríamos poder animarte a escribirnos tus dudas, pero
preferimos no hacerlo para no decepcionarte.


%-------------------------------------------------------------------
\section{Estructura de capítulos}
%-------------------------------------------------------------------
\label{cap1:sec:estructura}

El manual está estructurado en los siguientes capítulos:

\begin{itemize}
\item El capítulo~\ref{cap2} describe a vista de pájaro los distintos
  ficheros que forman \texis. Además da una primera aproximación
  a cómo generar el documento final (\texttt{.pdf}).

\item El capítulo~\ref{cap3} se centra en el proceso de
  edición. Aunque aparentemente la tarea de escribir el texto es
  trivial, \texis\ proporciona una serie de comandos que pueden
  ser útiles durante la escritura (al menos a nosotros nos lo
  parecieron). Este capítulo se centra en la explicación de esos
  comandos.

\item El capítulo~\ref{cap4} pasa a describir cómo se estructuran las
  imágenes en \texis. Igual que antes, esto puede parecer
  superfluo a un usuario medio de \LaTeX, pero \texis\ contiene
  algunos comandos que esperan esa estructura. Es el usuario el último
  que decide si utiliza esos comandos (y por lo tanto esa estructura)
  u opta por otra completamente distinta.

\item El capítulo~\ref{cap5} aborda la bibliografía y la gesión de los
  acrónimos. Como se verá, \texis\ dispone de algunas opciones de
  personalización que merecen un pequeño capítulo.

\item El capítulo~\ref{cap6} pone fin al manual, detallando las
  opciones del fichero \texttt{Makefile} que permiten una generación
  cómoda del documento final en entornos Linux.
\end{itemize}

El manual tiene, por último, un apéndice que, si bien no es
interesante desde el punto de vista del usuario, nos sirve de excusa
para proporcionar el código \LaTeX\ necesario para su creación: a modo
de ``así se hizo'', comenta brevemente cómo fue el proceso de
escritura de nuestras tesis.


%-------------------------------------------------------------------
\section*{\NotasBibliograficas}
%-------------------------------------------------------------------
\TocNotasBibliograficas

El ``libro'' por el que la mayoría de la gente empieza sus andaduras
con \LaTeX\ es \cite{ldesc2e} pues es relativamente corto, fácil de
leer y de acceso público (licencia GPL), por lo que se puede
conseguir la versión electrónica fácilmente. Un libro algo más
completo que éste y que suele ser el segundo en orden de preferencia
es \cite{notsoshort} con la misma licencia. Dentro de los libros
dedicados a \LaTeX\ de libre distribución, también se puede contar con
\cite{latexAPrimer}.

No obstante, los libros de \LaTeX\ más conocidos son ``The \LaTeX\
Companion'' \citep{latexCompanion} y ``\LaTeX: A Document Preparation
System'' \citep{LaTeXLamport}.

%-------------------------------------------------------------------
\section*{\ProximoCapitulo}
%-------------------------------------------------------------------
\TocProximoCapitulo

Una vez hecha una descripción de \texis, el próximo capítulo
describe los ficheros que componen tanto la plantilla como el manual
que estás leyendo. También se explicará cómo se puede generar o
compilar el manual a partir de los \texttt{.tex} proporcionados. Por
lo tanto, el capítulo sirve como una primera aproximación rápida al
trabajo con \texis; al final del mismo seremos capaces de entender
la estructura de directorios propuesta y dónde se encuentran los
ficheros que hay que editar para cambiar el contenido del documento
final.

No obstante, el capítulo siguiente debe verse únicamente como una
primera aproximación. El capítulo~\ref{cap:edicion} da más detalles
sobre el proceso de edición del documento, y el
capítulo~\ref{cap:makefile} dará una alternativa al modo de
compilación explicado.

% Variable local para emacs, para  que encuentre el fichero maestro de
% compilación y funcionen mejor algunas teclas rápidas de AucTeX
%%%
%%% Local Variables:
%%% mode: latex
%%% TeX-master: "../ManualTeXiS.tex"
%%% End:
